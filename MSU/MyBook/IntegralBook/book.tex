\documentclass{book} % Класс документа для книги

\usepackage[utf8]{inputenc} % Поддержка UTF-8
\usepackage[russian]{babel} % Поддержка русского языка
\usepackage{amsmath, amssymb} % Пакеты для работы с математикой
\usepackage{amsthm} % Пакет для определения теорем
\usepackage{graphicx}

\title{Интегралы: учебно-развлекательное пособие}
\author{Васильченко Дмитрий}
\date{} % Дата, автоматически выводит сегодняшнюю

\newtheorem{theorem}{Теорема}[chapter] % Нумерация теорем по главам
\newtheorem{lemma}{Лемма}[chapter] % Нумерация лемм по главам
\newtheorem{example}{Пример}[chapter] % Нумерация примеров по главам
\newtheorem{corollary}{Следствие}[chapter] % Нумерация следствий по главам
\newtheorem{notabene}{Замечание}[chapter]

% Определение окружений с нестандартным шрифтом (например, определение и замечание)
\theoremstyle{definition}
\newtheorem{definition}{Определение}[chapter]

\usepackage{titlesec}

% Настройка нумерации разделов с символом §
\titleformat{\section}
{\normalfont\Large\bfseries} % шрифт заголовка
{\S\thesection.} % нумерация с символом § и точкой
{1em} % отступ
{} % текст заголовка

\theoremstyle{remark}
\newtheorem{remark}{Замечание}[chapter]
\begin{document}
	
	
	\maketitle % Создание титульной страницы
	
	\tableofcontents
	
	\chapter{Введение}
	\section{Требования к читателю}
		\par
		Для успешного прочтения и понимания этой работы читателю необходимо обладать некоторыми начальными сведениями. \textbf{Необходимо} знать и понимать понятие предела и производной действительной функции одной переменной, не лишним будет иметь знания из начал линейной алгебры, а также необходимо знать хоть сколько-нибудь теорию множеств, хотя всю необходимую информацию из этого раздела я буду напоминать. 
		
		Когда мы перейдём к изучению интеграла Лебега возникнут вопрос из теории меры, впрочем, если автор не поленится, то непременно опишет и докажет все необходимые факты.
	\section{О функциях}
		\par
		Самым основным и важным объектом математического и функционального анализа являятся функция, со школы нам известно, что функции можно складывать, умножать, подставлять друг в друга и даже рисовать их графики! Ни для кого не секрет, что человек, знающий понятие функции, наверняка представляет себе в голове именно график функции, а не аналитическое описание и свойство этого объекта. По графику можно о функции понять очень многое: участки возрастания и убывания, гладкость, непрерывность, можно даже прикинуть где находятся корни. 
		
		Давайте немного порассуждаем теперь о свойствах функций, которые вам наверняка известны. Начнем мы с чётности.
		\begin{definition}
			Функция $f(x): \mathbb{R} \to \mathbb{R}$ называется чётной, если $\forall x \in \mathbb{R} \, f(x) = f(-x)$
		\end{definition}
		Говоря по-человечески, четная функция - это функция симметричная относительно оси OY, за примерами далеко ходить не придётся, самая простая чётная функция: $f(x) = |x|$. Совершенно понятно, что чётной будет являтся любая функция вида $f(x) = x^n$, где $|n|$ -чётное, обращу внимание, что это верно и для отрицательных $n$. Одной из важнейших чётных функций является $\cos{x}$, в дальнейшем мы часто будем использовать эту функцию для "опробывания" интегралов.
		
		Наряду с чётными функциями существую и нечётные
		\begin{definition}
			Функция $f(x): \mathbb{R} \to \mathbb{R}$ называется чётной, если $\forall x \in \mathbb{R} \, f(x) = -f(-x)$
		\end{definition}
		Примерами могут послужить линейные функции с $c= 0$, $\sin{x}$ и $x^n$, где $|n|$ - нечётное число.
		
		Ещё существуют периодические функции, которые также очень удобны для построения графиков
		\begin{definition}
				Функция $f(x): \mathbb{R} \to \mathbb{R}$ называется периодической, если $\exists T \in \mathbb{R}: \ \forall x \in \mathbb{R} \Rightarrow f(x) = f(x+T)$. $T$ в этом случае называется периодом
		\end{definition}
		Понятно, что если функция периодическая, то ее график состоит из повторяющихся частей, примером может послужить $\sin{x}$ или константа.
		
		Важнейшим классом функций являются непрерывные функции
		\begin{definition}
				Функция $f(x): \mathbb{R} \to \mathbb{R}$ называется непрерывной в т. $x_0$, если $\lim\limits_{t \to x_0} f(t) = f(x_0)$. Функция непрерывна на множестве, если непрерывна в каждой точке этого множества.
		\end{definition}
		В школе нас учили, что непрерывная функция это такая функция, график которой можно нарисовать не отрывая руки. Можно переформулировать, что у функции нет слишком резких скачков, приводящих к вертикальным разрывам графика. Примеров непрерывных функций пруд пруди, все линейные функции, все функции вида $x^n$, где $n \geq 1$, $\sin{x}, \cos{x}$ и т.д. Найти примеры разрывных функций труднее, но все же удаётся: $\dfrac{1}{x}$ будет разрывной в точке 0.
				\begin{example}
			Рассмотрим один из самых часто использующихся контрпримеров в мат анализе: функцию Дирихле, а именно
			\begin{eqnarray*}
				D(x) = \begin{cases}
					1, \, x \in \mathbb{Q} \\
					0, \, x \in \mathbb{R} \backslash \mathbb{Q}
				\end{cases}
			\end{eqnarray*}
			Как видно из определения в рациональных точках функция Дирихле принимает значения 1, а в иррациональных 0, понятно, что визуализировать это можно нарисовав две сплошные линии $y=1, \ y = 0$, но даст ли нам что-то такая визуализация?
		\end{example}
		\begin{example}
			Функция Хевисайда:
			\begin{equation*}
				\theta(x) = \begin{cases}
					1, \ x > 0 \\
					0, \ x \leq 0
				\end{cases}
			\end{equation*}
			Очевидно притерпевает разрыв в точке $x= 0$. Эту функцию ещё называют функцией бесконечной кривизны.
		\end{example}
		
		Ещё более приятным свойством функции будет дифференцируемость.
		\begin{definition}
				Функция $f(x): \mathbb{R} \to \mathbb{R}$ называется дифференцируемой в точке $x_0$, если в этой точке справедливо представление $f(x) = Ax + \overline{o}(x)$. Функция дифференцируема на множестве, если дифференцируема в каждой точке.
		\end{definition}
		Дифференцируемость ещё называют гладкостью, а под гладкостью понимается, что к любой точке графика можно провести касательную. Посмотрев на определение, можно понять, что изначально дифференцируемость в точке означает, что в близи рассматриваемой точки график функции стремится к прямой, это и позволяет построить касательную.
		
		Математики люди странные, напридумывают всяких определений, а потом пытаются понять кто из кого следует, этим займёмся и мы. Мы привели много примеров четных, нечетных и периодических функций и из них можно заметить, что из чётности/нечётности не следует периодичность, но и из периодичности не следует четность/нечётность, поэтому делаем вывод, что эти понятия вообще о разном. Что же касается непрерывности и дифференцируемости, очевидно, что дифференцируемость требует больше всяких условий, поэтому логично проверить следует ли из неё непрерывность. Предоставляю это вам, чтобы это сделать нужно осознать, что понятие дифференцируемости $f(x)$ в точке $x_0$ эквивалентно существованию предела $\lim\limits_{\Delta x \to 0} \dfrac{f(x_0+\Delta x) - f(x_0)}{\Delta x}$.
		
		Хорошо, непрерывность следует из дифференцируемости, но может быть верно и обратное? Тогда вообще получится, что непрерывность = дифференцируемость. В таких случаях нужно или доказывать утверждение или искать контрпримеры, дабы не тратить время займёмся вторым. Простейшим примером непрерывной, но не дифференцируемой функции будет $f(x) = |x|$, в точке $x = 0$. Если вы нуждаетесь в строгом доказательстве не дифференцируемости этой функции в точке $x = 0$, то рассмотрите односторонние пределы и убедитесь, что они имеют противополжные знаки. Что же это означает на наивном языке графиков? До точки 0 мы движемся вниз со скоростью 1, а после движемся вверх с такой же по модулю скоростью, но что происходит в точке 0? Мы движеся вверх или вниз? На этот вопрос ответ найти не получится, поэтому это функция и не является гладкой.
		
		Этот пример здорово показывает, что из непрерывности на множестве не следует дифференцируемость на этом же множестве. Но может быть нам удастся показать, что всегда существует чуть меньшее множество, на котором функция будет дифференцируема?
		\begin{example}
			Давайте рассмотрим забавный пример, хотим построить всюду непрерывную функцию, но не имеющую производной в некоторых точках ${x_n}$. Понятно, что можно просто задать кусочно линейную функцию, но хочется что-нибудь более приятное для глаз. Вспомним, что сумма непрерывных функций - функция непрерывная, а значит мы может брать две непрерывные функции, которые будет недифференцируемы каждая в своей точке и суммировать их, результат будет удовлетворять описанным выше требованиям, в итоге получим: 
			\begin{equation*}
				f(x) = \sum\limits_{n=1}^{N} |x - x_n|, 
			\end{equation*}
			однако этот пример справедлив лишь для конечного множества точек, для обобщения на случай множества нужно лишь сделать ряд сходящимся:
			\begin{equation*}
				f(x) = \sum\limits_{n=1}^{\infty} \dfrac{1}{2^n}|x - x_n|, 
			\end{equation*}
		\end{example}
		
		Отлично, мы получили пример непрерывной функции, которая недифференцируема на счётном множестве, но даже на интервале $(0,1)$ точек больше. Здесь мы прибегнем к двум мощнейшим примерам, перевернувшим в своё время представление о непрерывности и дифференцируемости.
		
		\begin{example}
		Построим функцию непрерывную всюду, но не имеющей нигде производной, пример ван дер Вардена:
			\begin{equation}
				f(x) = \sum\limits_{n=0}^{\infty} \dfrac{\{10^n x\}}{10^n}, 
			\end{equation}
			здесь $\{x\}$ - расстояние от $x$ до ближайшего целого числа. Давайте покажем, что эта функция непрерывна и не дифференцируема.
			\par
			Непрерывность $f(x)$ следует из непрерывности каждой $\dfrac{\{10^n x\}}{10^n}$ и из оценки $\dfrac{\{10^n x\}}{10^n} \leq 10^{-n}$. Попробует теперь вычислить производную $f(x)$ в какой-нибудь точке $x \in [0, 1)$. Запишем $x$ в виде $x = 0,a_1a_2\dots a_n\dots$, если дробь конечна, то дополняем нулями.
			\begin{equation*}
				\{10^n x\} = \begin{cases} 
					0,a_{n}a_{n+1}\dots, \text{если }	0,a_{n}a_{n+1}\dots \leq \dfrac12 \\
					1 - 0,a_{n}a_{n+1}\dots, \text{если } 0,a_{n}a_{n+1}\dots > \dfrac12
				\end{cases}
			\end{equation*}
			Положим приращение аргумента $h_m = - 10^{-m}$, если $a_m = 4$ или $a_m = 9$ и $h_m = 10^{-m}$ иначе. Рассмотрим отношение
			\begin{equation*}
				\dfrac{f(x+h_m) - f(x)}{h_m} = 
			\end{equation*}
			\begin{equation*}
				= 10^m \sum\limits_{n=0}^{\infty} \pm \dfrac{\{10^n(x\pm 10^{-m})\} - \{10^nx\}}{10^n}
			\end{equation*}
			Рассмотрим числитель при $n \geq m$:
			\begin{equation*}
				\{10^n(x\pm 10^{-m})\} - \{10^nx\} = \{ 10^nx + 10^b\} - \{10^nx\} = 0, 
			\end{equation*}
			т.к. в этом случае $b \in \mathbb{N}_0$, т.е. $10^b$ не влияет на результат операции потому что является целым. Пусть теперь $ n < m$
			\begin{equation*}
				\{10^n(x\pm 10^{-m})\} - \{10^nx\} = \{0,a_na_{n+1}\dots (a_{n-m} \pm 1) a_{n-m+1} \dots\} - \{0,a_na_{n+1}a_{n+2}\dots\}
			\end{equation*}
			В данном случае первое выражение может "перескочить"  $\dfrac12$ и стать ближе к верхнему числу, тогда выражение будет равняться $-10^{n-m}$, а если оно останется меньше $\dfrac12$, то выражение будет равно $10^{n-m}$. 
			
			Давайте визуализируем это на примере $x = 0,1492$. Пусть $n=0, m=1$, тогда
			\begin{equation*}
				\{10^n(x \pm 10^{-m})\} - \{10^nx\} = \{0,2492\} - \{0,1492\} = 10^{-1} = 10^{n-m}
			\end{equation*}
			или 
			\begin{equation*}
				\{10^n(x \pm 10^{-m})\} - \{10^nx\} = \{0,0492\} - \{0,1492\} = -10^{-1} = -10^{n-m}
			\end{equation*}
			если же $n = 1, m = 2$, то
			\begin{equation*}
				\{10^n(x \pm  10^{-m})\} - \{10^nx\} = \{0,502\} - \{0,492\} = -10^{-1} = -10^{n-m}
			\end{equation*}
			или аналогично может быть $10^{n-m}$.
			
			Таким образом получили, что числители дробей могут обращаться в 0 или обращаться в целые положительные или отрицательные значения, получаем, что не выполнено необходимое условие сходимости числового ряда, следовательно производная не существует ни в одной точке.
		\end{example}
		\begin{example}
		Рассмотрим более интересный пример, а именно функцию Вейерштрасса:
		\begin{equation*}
			f(x) = \sum\limits_{n=0}^{\infty} a^n \cos{b^n \pi x}, \ 0 < a < 1, \ ab > 1 + \dfrac{3\pi}2.
		\end{equation*}
		Это функция является непрерывной всюду, но нигде не дифференцируемой, то есть ни к одной точке графика этой функции нельзя провести касательную, в таком случае едва ли у нас получится построить этот самый график.
		\end{example}
		
		
		Однако это все удобно лишь когда мы находимся в школе, выйдя за ее пределы, все становится гораздо и гораздо хуже и непонятнее. Появляются функции визуализировать, которые не представляется никакой возможности, чтобы не быть голосословным приведу пример:
	\section{О физических и иных приложениях изучаемого понятия}
		Если вкратце, то интегралы применяются везде, от физики до машинного обучения, но поговорим о некоторых приложениях более подробно
		
		Рассмотрим простейший пример, пусть у нас есть следующий график: по оси x день недели, по оси y количество денег которое мы потратили за день. Задача известна каждому: посчитать количество денег, потраченных за неделю. На первый взгляд пример никакого интереса представлять не может, но давайте посмотрим на бытовую задачу с точки зрения графика. Заметим, что если за каждый день мы потратили одинаковую сумму, то нам необязательно складывать 7 чисел, достаточно умножить сумму, потраченную за день, на 7. Если приглядеться, можно заметить, что сумма, потраченная за неделю равняется площадью под графиком наших трат. По сути это есть и есть интеграл. 
		
		Интегралы, как и многие другие математические операции, пришли к нам из физики. Пусть функция $N(t)$ описывает как мощность двигателя меняется во времени, тогда чтобы найти работу нам нужно будет проинтегрировать эту функцию, простейший пример это когда $ N = const$, тогда работа будет $A = NT$, где $T$ - прошедшее время.	Аналогично можно вычислять пройденный путь, зная скорость и ещё много подобных примеров.
		
		Вторым важнейшим физическим применением интегралов будет нахождение массы тела, когда мы знаем его плотность. Дело в том, что плотность может не быть постоянной, например, в атмосфере земли, плотность воздуха непостоянна и меняется при изменении высоты. Для вычисления такой массы используюся интегралы.
		
		Помимо этого интегралы используются для нахождения центра масс сложных объектов, генерации случайных чисел, в 3D-графике, моделировании популяций животных и даже расчёта необходимого освещения в архитектуре.
		
	\chapter{Неопределенный интеграл}
		Наконец мы вступаем во владения математического анализа, в этой главе мы очень подробно познакомимся с понятием неопределённого интеграла, разберёмся в чем его "ощутимый смысл" и разберём множество примеров.
	\section{Строгое определение}
		\begin{definition}
			$F(x): M \to \mathbb{R}$ и $F(x)$ - дифференцируема на множестве $M$ называется первообразной функции $f(x): M \to \mathbb{R}$, если $F'(x) = f(x), \ \forall x \in M$.
		\end{definition}
		Сразу обратим наше внимание, что если $F(x)$ первообразная для $f(x)$, то и $F(x) + C$ будет первообразной $f(x)$. Это следует из линейности операции дифференцирования и факта $\mathrm{const}' = 0$ (этот факт в свою очередь следует из определения производной, или на пальцах, функция не меняется, значит ее скорость изменения (производная) равна нулю).
		\begin{lemma}
			Пусть $F(x), G(x)$ первообразные функции $f(x)$ на связном множестве $M$, тогда $F(x) = G(x) + C$.
		\end{lemma}
		\begin{proof}[Доказательство:]
			Рассмотрим функцию $H(x) = F(x) - G(x)$, её производная тождественно равна 0, значит $H'(x) = 0$, следовательно $H(x) = const$.
		\end{proof}
		\begin{notabene}
			Обратим внимание на важность связности множества $M$. В случае интервала, луча или прямой наше утверждение верно, но если в качестве $M$ взять $(0,1) \cup (3,5)$. Тогда $F(x) \equiv 1$ и $G(x) = \begin{cases} a\neq0,\ x \in (0,1) \\ b\neq 0, \ x \in (3,5) \end{cases}$ являются первообразными, но не отличаются на константу.
			
			Почему так нельзя сделать в случае связного множества? Потому дифференцируемая функция обязана быть непрерывной.
		\end{notabene}
		\begin{notabene}
			Актуален вопрос, любая ли функция имеет первообразную? Нет, существуют функции, не являющиеся производной никакой функции, например
			$$
				f(x) = \begin{cases}
					1, \ x \geq 0 \\
					-1, \ x < 0.
				\end{cases}
			$$
			Вопрос о том, когда все-таки существуют первообразные будет более подробно обсуждён позднее, в главе определённый интеграл.
		\end{notabene}
		\begin{definition}
			 Операция перехода от функции $f(x)$ к её первообразной называется неопределённым интегрированием. Совокупность всех первообразных $\{F(x) + C \vert C \in \mathbb{R}, F'(x) = f(x)\}$ функции $f(x)$ называется неопределённым интегралом функции $f(x)$. Обозначается как $\int f(x) dx$.
		\end{definition}
		Формально, когда мы говорим о первообразной функции $f(x)$ мы обязаны записать это в следующим виде
		$$
		\int f(x) dx = \{F(x) + C \vert C \in \mathbb{R}\},
		$$
		но на практике мы будем опускать скобки и писать следующее
		$$
		\int f(x) dx = F(x) + C.
		$$
	\section{Свойства неопределённого интеграла}
		Введём обозначение
		$$
		\int d g(x) = \int g'(x) dx
		$$
		\begin{theorem}
			(Основные свойства неопределенного интеграла)\newline
			\begin{enumerate}
				\item $\left(\int f(x) dx\right)' = f(x);$
				\item $\int d F(x) = \int F'(x) dx = F(x) + C$
				\item Пусть $\exists \int f_1(x) dx, \int f_2(x) dx$, тогда  $\exists \int \alpha f_1(x) \pm \beta f_2(x) dx = \alpha \int f_1(x) \pm \beta \int f_2(x) dx$
			\end{enumerate}
		\end{theorem}
		\begin{proof}
			\textbf{1}. По сути следует из определения, но покажем подробнее
			неопределённый интеграл от функции представляет из себя множество, операция дифференцирования множества неопределена. Поэтому будем воспринимать это как дифференцирование каждого элемента множества, тогда 
			$$\left(\int f(x) dx \right)' = \{F'(x) + C' \vert C \in \mathbb{R}\} = \{f(x) \}$$
			В силу того, что производная от константы 0 и определения $F(x)$.\newline
			\textbf{2}. Аналогично следует из определения, читатель может в этом убедиться самостоятельно.\newline
			\textbf{3}. Покажем, что функция, стоящая справа, является первообразной функции $\alpha f_1(x) \pm \beta f_2(x)$.\newline
			Пусть $F(x) = \alpha \int f_1(x) \pm \beta \int f_2(x) dx$. Тогда $F'(x) = \alpha \left(\int f_1(x) dx\right)' \pm \beta \left(\int f_2(x) dx\right)'$. По свойству 1 $F'(x) = \alpha f_1(x) \pm \beta f_2(x)$. Используя свойство 2, все доказано.
		\end{proof}
		\begin{notabene}
			Заметим, что свойства 1, 2 показывают, что операция неопределённого интегрирования является обратной к операции дифференцирования. Поэтому зная табличные производные мы можем получить табличные интегралы. Например, 
			$$
				\left(\ln{|x|}\right)' = \mathrm{sgn}x\dfrac{1}{|x|} = \dfrac{1}{x} \Rightarrow \int \dfrac{1}{x} dx = \ln{x} + C,
			$$
			$$
				\left(x^n\right)' = n x^{n-1} \Rightarrow \int x^n dx = \dfrac{x^{n+1}}{n+1} + C
			$$
		\end{notabene}
		\begin{theorem} (интегрирование по частям)\newline
			Пусть на некотором промежутке функции $u(x), v(x)$ дифференцируемы и существует один из интегралов $\int u'(x) v(x) dx , \int u(x) v'(x) dx$, тогда существует и второй и справедливо
			$$
				\int u'(x) v(x) dx = u(x) v(x) - \int v'(x) u(x) dx
			$$
		\end{theorem}
		\begin{proof}
			Рассмотрим 
			
			$$
				\left(u(x) v(x) - \int u'(x) v(x) dx\right)' = u'v +uv' - u'v = uv'.
			$$
			По второму свойству получим $\int u'(x) v(x) dx = u(x) v(x) - \int v'(x) u(x) dx$
		\end{proof}
		\begin{theorem}
			(замена переменной в неопределенном интеграле)\newline
			Если на некотором промежутке $I_x$ выполнено: $\int f(x) dx = F(x) + C$, а $\varphi: I_t \to I_x$ - дифференцируемая функция, тогда на $I_t$ выполнено
			$$
				\int f(\varphi(t)) \varphi'(t) dt  = F(\varphi(t)) + C
			$$
		\end{theorem}
		\begin{proof}
			Используем теорему о производной сложной функции 
			$$
				\left(F(\varphi(t)) + C \right)' = F'(\varphi(t)) * \varphi'(t) = f(\varphi(t)) * \varphi'(t),
			$$
			откуда по второму свойству получим требуемое.
		\end{proof}
		Рассмотрим, теперь, несколько примеров.
		\begin{example}
			$$
				\int \dfrac{1}{1 + x^2} dx = \arctg{x} + C,
			$$
			так как производная $\arctg x$ равняется $\dfrac{1}{1+x^2}$.
		\end{example}
		\begin{example}
			Пример использования интегрирования по частям
			$$
			\int \sin{x} x dx = \int (- \cos{x})' x dx = -x \cos{x} + \int x' \cos{x} dx = -x \cos{x} + \int \cos{x} dx = -x \cos{x} + \sin{x} + C
			$$
		\end{example}
		\begin{example}
			Пример использования формулы замены переменной.
			$$
				\int \dfrac{1}{a^2 + x^2} dx = \int \limits \dfrac{1}{a^2\left(1 + \dfrac{x^2}{a^2}\right)}dx = \{y = \dfrac{x}{a}, dx  = a dy\} = \dfrac{1}{a^2} \int \dfrac{a}{1 + y^2} dy = \dfrac{1}{a} \arctg{y} = \dfrac{1}{a} \arctg{\dfrac{x}{a}}$$. 
			
		\end{example}
		Очень важно понимать какими свойствами должна обладать первообразная, иначе можно получать фантастически неверные результаты интегрирования, как минимум первообразная должна быть непрерывна, следующая теорема даёт чуть больше информации о связи первообразной с её "порождающей" функцией.
		
		\begin{theorem}
			Пусть $F(x) \in C(a,b)$ и $c \in (a,b)$, а функция $f: (a,b) \to \mathbb{R}$ непрерывна в точке $c$. Пусть $F(x)$ - первообразная функции $f(x)$ на $(a,c), (c,b)$. Тогда $F(x)$ - первообразная для $f(x)$ на $(a,b)$.
		\end{theorem}
		\begin{proof}
			По условию
			$$
			F'(x) = f(x), \ x \in (a,c) \cup (c,b).
			$$
			Нужно показать, что $F(x)$ дифференцируема в точке $c$ и $F'(c) = f(c)$. \newline
			Рассмотрим последовательность $\{x_n\} \to c - 0$. На отрезке $[x_n, c]$ для $F(x)$ выполнены все условия теоремы Лагранжа о конечных приращениях, следовательно $\exists \xi \in (x_n, c):$
			$$
			\dfrac{F(x_n) - F(c)}{x_n - c} = F'(\xi_n) = f(\xi_n).
			$$
			$x_n < \xi_n< c$, по теореме о двух эвольвентах $\xi_n \to c - 0$. Поэтому в силу непрерывности $f(x)$ в точке $c$, получим $F'(c) = f(c)$. Аналогично можем показать для $\{x_n\} \to c + 0$.
		\end{proof}
		\begin{example}
			Рассмотрим
			$$
			 \int |x| dx $$
			 Распишем на два случая:
			 $$
			 \int |x| dx  = \begin{cases}
			 		\dfrac{x^2}{2} + C_1 , \ x \geq 0\\
			 		-\dfrac{x^2}{2} + C_2, \ x < 0
			 \end{cases}
			 $$		
			 Обозначим $f(x) = |x|$, тогда $f(0-0) = f(0) = f(0+0) = 0$ из этого получаем, что $C_1 = C_2$, и окончательно
			 $$
			 \int |x| dx  = \dfrac{x|x|}{2} + C
			 $$
		\end{example}
		\begin{example}
			$$
			\int e^{|x|}dx = \begin{cases}
				e^{x} + C_1, \ x \geq 0\\
				-e^{x} + C_2, \ x < 0
			\end{cases} = \{\text{чтобы первообразная была непрерывна }\} =
						 \begin{cases}
							e^{x} + C_1, \ x \geq 0\\
							-e^{x} + C_1 + 2, \ x < 0
						\end{cases}
			$$
		\end{example}
	\chapter{Определенный интеграл Римана}
		В основе понятия определенного интеграла Римана лежит задача нахождения площади криволинейной трапеции. То есть нам дана функция, надо определить площадь между графиком функции и осью Ox.
		
		\begin{definition}
			Разбиением отрезка $[a,b]$ называется множество точек 
			$$
			\tau = \{x_k\}_{k=0}^{n} = \{a = x_0 < x_1 < \dots < x_{n-1} < x_n = b\},
			$$
			$\Delta_k = x_{k+1}-x_k$ - длина k-го отрезка, $d = \max\limits_{0 \leq x \leq n-1 } \Delta_k$ - диаметр разбиения.
		\end{definition}
		\begin{definition}
			Размеченным разбиением $[a,b]$ называется пара $(\tau, \xi)$, где $\xi = (\xi_1, \dots, \xi_{n})$, $\xi_k \in [\tau_k, \tau_{k-1}]$.
		\end{definition}
		\begin{definition}
			Пусть $f: [a,b] \to \mathbb{R}$, тогда сумма
			$$
			\sigma = \sigma_\tau(f, \xi) = \sum\limits_{k=1}^{n} f(\xi_k) \Delta_k
			$$
			называется интегральной суммой Римана функции $f(x)$, отвечающей разбиению $(\tau, \xi)$ отрезка $[a,b]$.
		\end{definition}
		\begin{definition}
			Пусть $f: [a,b] \to \mathbb{R}$. Число $I$ называют пределом интегральных сумм при диаметере разбиения стремящемся к нулю, и пишут $I = \lim\limits_{d\tau \to 0} \sigma_\tau(f, \xi)$, если:
			$$
				\forall \varepsilon > 0 \ \exists \delta(\varepsilon) > 0: \ \forall \tau, d_\tau < \delta(\varepsilon), \forall \xi \Rightarrow |\sigma_\tau(f, \xi) - I| < \varepsilon.
			$$
			В этом случае функция $f$ называется интегрируемой по Риману на $[a,b]$, а число $I$ называется определённым интегралом Римана от функции $f$ по отрезку $[a,b]$. Обозначение $I = \int\limits_a^b f(x) dx$.
		\end{definition}
	\chapter{Интеграл Лебега}
	
	
\end{document}