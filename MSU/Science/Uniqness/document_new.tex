\documentclass[12pt, a4paper]{article}

% --- PREAMBLE ---
\usepackage[russian]{babel}
\usepackage[utf8]{inputenc}
\usepackage{amsmath}
\usepackage{amssymb}
\usepackage{geometry}

% Page layout
\geometry{
	a4paper,
	left=2.5cm,
	right=2.5cm,
	top=2.5cm,
	bottom=2.5cm
}

\author{Васильченко Д.Д.}
% --- DOCUMENT ---
\begin{document}
	
	\begin{center}
		\Large\bfseries
		Доказательство единственности решения задачи для уравнения Лапласа в случае граничных условий I-I  и II-II.
	\end{center}
	
		\begin{center}
		\large Васильченко Д.Д.
	\end{center}
	\vspace{1cm}
	
	\section{Задача I-I}
	Рассмотрим в области $D^+ = \{(x,y): x \in (0,\pi), y > 0\}$ следующую задачу.
	\begin{equation}
		\dfrac{\partial^2 u}{\partial x^2}(x,y) + \dfrac{\partial^2 u}{\partial y^2}(x,y) = 0
	\end{equation}
	с граничными условиями 
	\begin{equation}
		u(0,y) = 0, \ u(\pi, y) = 0, \ 0 < y < +\infty, 
	\end{equation}
	\begin{equation}
		\dfrac{1}{k} \dfrac{\partial u}{\partial y}(x,0+0) - \dfrac{\partial u}{\partial x}(x,0+0) = \varphi(x), \,\varphi(x) \in L_2(0,\pi),
	\end{equation}
	\begin{equation}
		u(x,y) \rightrightarrows 0, \ y \to +\infty 
	\end{equation}
	
	\textbf{Теорема 1.} \textit{Пусть функция $u(x,y)$ является гармонической в области $\Omega \subset \mathbb{R}^2$. Пусть $P \in \Omega$ и $R > 0$ такие, что шар $\overline{B(P, R)} \subset \Omega$. Тогда справедлива следующая оценка
		\begin{equation*}
			|\nabla u(P)| \leq \dfrac{4\sqrt2}{\pi R} \max\limits_{z \in \partial B(P,R)} |u(z)|.
	\end{equation*}}
	\textbf{Доказательство.}
	Производная гармонической функцией сама является гармонической и для гармонических функций справедливо свойство среднего. Применим его к производной функции $u_x(x,y)$ в точке $P$, для просты положим $P = (0,0)$.
	\begin{equation*}
		u_x(0,0) = \dfrac{1}{\pi R^2} \iint\limits_{B(0,R)} u_x(x,y) dx dy.
	\end{equation*}
	
	Применим теорему Гаусса-Отстроградского для векторного поля $\vec{F} = (u, 0)$, тогда $\nabla\vec{F} = u_x$:
	\begin{equation*}
		\iint\limits_{B(0,R)} u_x dx dy = \oint\limits_{\partial B(0,R)} u \cdot \eta_x ds.
	\end{equation*}
	Тогда 
	\begin{equation*}
		u_x(0,0) = \dfrac{1}{\pi R^2} \oint\limits_{\partial B(0,R)} u \cdot \eta_x ds.
	\end{equation*}
	Теперь оценим интеграл
	\begin{equation*}
		|u_x(0,0)| \leq \dfrac{1}{\pi R^2} \oint\limits_{\partial B(0,R)} |u(s)| \cdot |\eta_x(s)| ds \leq \dfrac{1}{\pi R^2} \oint\limits_{\partial B(0,R)}  \max\limits_{z \in \partial B(P,R)} |u(z)| \cdot |\eta_x(s)| ds.
	\end{equation*}
	Вектор нормали имеет вид $\vec{n} = (\cos\alpha, \sin\alpha)$, элемент длины дуги $ds = R\,d\alpha$.
	\begin{equation*}
		|u_x(0,0)| \leq \dfrac{1}{\pi R^2}  \max\limits_{z \in \partial B(P,R)} |u(z)|\int\limits_{0}^{2\pi}  |\cos \alpha| R d\alpha = \dfrac{4}{\pi R^2}  \max\limits_{z \in \partial B(P,R)} |u(z)|.
	\end{equation*}
	Аналогичную оценку можем получить и для $u_y(0,0)$, тогда оценка для градиента выглядит следующим образом:
	\begin{equation*}
		|\nabla u(0,0)| \leq \dfrac{4\sqrt2}{\pi R^2}  \max\limits_{z \in \partial B(P,R)} |u(z)|.
	\end{equation*}
	Теорема доказана.
	
	\textbf{Лемма 1.} \textit{Пусть $u(x,y)$ - решение задачи (1)-(4), тогда $u_y(x,y) \rightrightarrows 0$ при $ y \to \infty$}.
	\textbf{Доказательство.}По условию (4)
	\begin{equation*}
		\forall \varepsilon > 0 \exists y^* > 0 \colon \forall x_0 \in (0,\pi), y_0 > y^* \ \, \, |u(x_0,y_0)| < \varepsilon.
	\end{equation*}
	Используем теорему 1, сперва выберем шар с центром в точке $P = (x_0, y_0)$ и радиусом $R = \min\{x_0, R - x_0\}$, то есть $B(P, R)$. По теореме 1 $u_y(P) \leq  \dfrac{4\sqrt2}{\pi R^2}  \max\limits_{z \in \partial B(P,R)} |u(z)| <  \dfrac{4\sqrt2}{\pi R^2} \varepsilon$. Получаем следующее:
	\begin{equation*}
		\forall \varepsilon^* > 0 \ \exists y^* > 0 \colon \forall x_0 \in (0,\pi) y_0 > y^* \,\, |u'_y(x_0,y_0)| < \varepsilon^* = \dfrac{4\sqrt2}{\pi R^2} \varepsilon.
	\end{equation*}
	Лемма доказана.
	
	
	\textbf{Теорема 2.} \textit{Решение задачи (1)-(4) единственно для $k \in \mathbb{R}  \backslash \{0\}$}.
	
	\textbf{Доказательство.} Используем энергетический метод.
	Пусть $u(x,y)$ - решение однородной задачи (1)-(4) ($\varphi(x) \equiv 0$). Рассмотрим прямоугольник $D_{\varepsilon R} = (0, \pi) \times (\varepsilon, R) \subset D^+$. Справедливо следующее равенство.
	\begin{equation*}
		u\Delta u = \nabla (u \nabla u) - |\nabla u|^2.
	\end{equation*}
	Воспользуемся теоремой Гаусса-Остроградского для выражения
	\begin{equation*}
		\iint\limits_{D_{\varepsilon R}} u\Delta u dx dy = \iint\limits_{D_{\varepsilon R}} \nabla (u \nabla u) dx dy  - \iint\limits_{D_{\varepsilon R}}|\nabla u|^2 dx dy.
	\end{equation*}
	Получим
	\begin{equation*}
		\iint\limits_{D_{\varepsilon R}} u\Delta u dx dy =  \iint\limits_{\partial D_{\varepsilon R}}u \dfrac{\partial u}{\partial \eta} ds -  \iint\limits_{D_{\varepsilon R}}|\nabla u|^2 dx dy.
	\end{equation*}
	В силу условия (1) получаем
	\begin{equation*}
		\iint\limits_{\partial D_{\varepsilon R}}u \dfrac{\partial u}{\partial \eta} ds = \iint\limits_{D_{\varepsilon R}}|\nabla u|^2 dx dy.
	\end{equation*}
	Интеграл по границе вычисляется следующим образом
	\begin{equation*}
		\iint\limits_{\partial D_{\varepsilon R}}u \dfrac{\partial u}{\partial \eta} ds = - \int\limits_{\varepsilon}^{R} u(0,y) u'_x(0,y) dy + \int\limits_{\varepsilon}^R u(\pi, y) u'_x(\pi, y) dy + \int\limits_0^\pi u(x,R) u_y(x,R) dx - \int\limits_0^\pi u(x,\varepsilon) u'_y(x, \varepsilon) dx.
	\end{equation*}
	В силу граничных условий (2) первый и второй интеграл обращается в 0. В силу выше доказанной леммы и условия (4) третий интеграл также обращается в 0 при стремлении R к $+\infty$. Рассмотрим последний интеграл
	\begin{equation*}
		\int\limits_0^\pi u(x,\varepsilon) u'_y(x, \varepsilon) dx = \int\limits_0^\pi u(x,\varepsilon) \left[ u'_y(x,\varepsilon) - k u'_x(x, \varepsilon) \right] dx + \int\limits_0^\pi ku(x,\varepsilon) u'_x(x, \varepsilon) dx
	\end{equation*}
	\begin{equation*}
		=\int\limits_0^\pi u(x,\varepsilon) \left[ u'_y(x,\varepsilon) - k u'_x(x, \varepsilon) \right] dx+ \dfrac{k}{2} \int\limits_0^\pi \left(u^2(x,\varepsilon) \right)'_x dx =
	\end{equation*}
	\begin{equation*}
		=  \int\limits_0^\pi u(x,\varepsilon) \left[ u'_y(x,\varepsilon) - k u'_x(x, \varepsilon) \right] dx + \dfrac{k}{2} \left[u^2(\pi, \varepsilon) - u^2(0, \varepsilon) \right].
	\end{equation*}
	Устремим $\varepsilon \to 0+0$, тогда интграл обращается в 0 в силу условия (3) и получаем итоговое выражение
	\begin{equation*}
		\iint\limits_{D^+}|\nabla u|^2 dx dy = \dfrac{k}{2} \left[u^2(\pi, \varepsilon) - u^2(0, \varepsilon) \right],
	\end{equation*}
	в силу граничных условий (2) получаем
	\begin{equation*}
		\iint\limits_{D^+}|\nabla u|^2 dx dy = 0.
	\end{equation*}
	Значит 
	\begin{equation*}
		\dfrac{\partial u}{\partial x} = \dfrac{\partial u}{\partial y} = 0, (x,y) \in D^+,
	\end{equation*}
	т.к. $u(0, y) = 0, \forall y > 0$ получаем, что $u(x,y) \equiv 0$ в $D^+$. Теорема доказана.
	
	
	\section{Задача II-II}
		Рассмотрим в области $D^+ = \{(x,y): x \in (0,\pi), y > 0\}$ следующую задачу.
	\begin{equation}
		\dfrac{\partial^2 u}{\partial x^2}(x,y) + \dfrac{\partial^2 u}{\partial y^2}(x,y) = 0
	\end{equation}
	с граничными условиями 
	\begin{equation}
		\dfrac{\partial u}{\partial x}(0,y) = 0, \ \dfrac{\partial u}{\partial x}(\pi, y) = 0, \ 0 < y < +\infty, 
	\end{equation}
	\begin{equation}
		\dfrac{1}{k} \dfrac{\partial u}{\partial y}(x,0+0) - \dfrac{\partial u}{\partial x}(x,0+0) = \varphi(x), \,\varphi(x) \in L_2(0,\pi),
	\end{equation}
	\begin{equation}
		u(x,y) \rightrightarrows 0, \ y \to +\infty 
	\end{equation}
	
		\textbf{Лемма 2.} \textit{Пусть $u(x,y)$ - решение однородной задачи (1)-(4), тогда $u(0,0) = u(\pi,0)$}.
	
	\textbf{Доказательство.}
	Воспользуемся теоремой Гаусса-Остроградского для векторного поля $\vec{F} = \nabla u$. Тогда $\mathrm{div} \vec{F} = \Delta u$. Рассмотрим прямоугольник $D_R = (0,\pi) \times (0, R) \subset D^+$. 
	\begin{equation*}
		\iint_{D_R} \Delta u dx dy = \iint\limits_{\partial D_R} \nabla u \cdot \vec{n} ds  = \oint\limits_{\partial D_R} \dfrac{\partial u}{\partial n} ds.
	\end{equation*}
	Следовательно в силу (5)
	\begin{equation*}
		\oint\limits_{\partial D_R} \dfrac{\partial u}{\partial n} ds = 0
	\end{equation*}
	Рассмотрим этот интеграл подробнее
	\begin{equation*}
			\oint\limits_{\partial D_R} \dfrac{\partial u}{\partial n} ds  =  -\int\limits_0^\pi \dfrac{\partial u}{\partial y}(x,0) dx  + \int\limits_0^\pi  \dfrac{\partial u}{\partial y}(x,R) dx  + \int\limits_0^R \dfrac{\partial u}{\partial x}(\pi, y) dy - \int\limits_0^R \dfrac{\partial u}{\partial x}(0, y) dy
	\end{equation*}
	В силу условий (6) третий и четвертый интегралы обнуляются, при стремлении $R \to + \infty$  в силу леммы 1 второй интеграл стремится к 0. Получаем (аналогично теореме 2)
	\begin{equation*}
		0 = \oint\limits_{\partial D_R} \dfrac{\partial u}{\partial n} ds = -\int\limits_0^\pi \dfrac{\partial u}{\partial y}(x,0) dx = - k \int\limits_0^\pi \dfrac{\partial u}{\partial x}(x,0)dx = u(0,0) - u(\pi, 0).
	\end{equation*}
	Лемма доказана.
	
	
		\textbf{Теорема 3.} \textit{Решение задачи (5)-(8) единственно для $k \in \mathbb{R}  \backslash \{0\}$}.
	
	\textbf{Доказательство.} Аналогично доказательству теоремы 2. Используем энергетический метод.
	Пусть $u(x,y)$ - решение однородной задачи (5)-(8) ($\varphi(x) \equiv 0$). Рассмотрим прямоугольник $D_{\varepsilon R} = (0, \pi) \times (\varepsilon, R) \subset D^+$. Справедливо следующее равенство.
	\begin{equation*}
		u\Delta u = \nabla (u \nabla u) - |\nabla u|^2.
	\end{equation*}
	Воспользуемся теоремой Гаусса-Остроградского для выражения
	\begin{equation*}
		\iint\limits_{D_{\varepsilon R}} u\Delta u dx dy = \iint\limits_{D_{\varepsilon R}} \nabla (u \nabla u) dx dy  - \iint\limits_{D_{\varepsilon R}}|\nabla u|^2 dx dy.
	\end{equation*}
	Получим
	\begin{equation*}
		\iint\limits_{D_{\varepsilon R}} u\Delta u dx dy =  \iint\limits_{\partial D_{\varepsilon R}}u \dfrac{\partial u}{\partial \eta} ds -  \iint\limits_{D_{\varepsilon R}}|\nabla u|^2 dx dy.
	\end{equation*}
	В силу условия (1) получаем
	\begin{equation*}
		\iint\limits_{\partial D_{\varepsilon R}}u \dfrac{\partial u}{\partial \eta} ds = \iint\limits_{D_{\varepsilon R}}|\nabla u|^2 dx dy.
	\end{equation*}
	Интеграл по границе вычисляется следующим образом
	\begin{equation*}
		\iint\limits_{\partial D_{\varepsilon R}}u \dfrac{\partial u}{\partial \eta} ds = - \int\limits_{\varepsilon}^{R} u(0,y) u'_x(0,y) dy + \int\limits_{\varepsilon}^R u(\pi, y) u'_x(\pi, y) dy + \int\limits_0^\pi u(x,R) u_y(x,R) dx - \int\limits_0^\pi u(x,\varepsilon) u'_y(x, \varepsilon) dx.
	\end{equation*}
	В силу граничных условий (6) первый и второй интеграл обращается в 0. В силу леммы 1 и условия (8) третий интеграл также обращается в 0 при стремлении R к $+\infty$. Рассмотрим последний интеграл
	\begin{equation*}
		\int\limits_0^\pi u(x,\varepsilon) u'_y(x, \varepsilon) dx = \int\limits_0^\pi u(x,\varepsilon) \left[ u'_y(x,\varepsilon) - k u'_x(x, \varepsilon) \right] dx + \int\limits_0^\pi ku(x,\varepsilon) u'_x(x, \varepsilon) dx
	\end{equation*}
	\begin{equation*}
		=\int\limits_0^\pi u(x,\varepsilon) \left[ u'_y(x,\varepsilon) - k u'_x(x, \varepsilon) \right] dx+ \dfrac{k}{2} \int\limits_0^\pi \left(u^2(x,\varepsilon) \right)'_x dx =
	\end{equation*}
	\begin{equation*}
		=  \int\limits_0^\pi u(x,\varepsilon) \left[ u'_y(x,\varepsilon) - k u'_x(x, \varepsilon) \right] dx + \dfrac{k}{2} \left[u^2(\pi, \varepsilon) - u^2(0, \varepsilon) \right].
	\end{equation*}
	Устремим $\varepsilon \to 0+0$, тогда интграл обращается в 0 в силу условия (3) и получаем итоговое выражение
	\begin{equation*}
		\iint\limits_{D^+}|\nabla u|^2 dx dy = \dfrac{k}{2} \left[u^2(\pi, 0+0) - u^2(0, 0+0) \right],
	\end{equation*}
	в силу леммы 2 получаем
	\begin{equation*}
		\iint\limits_{D^+}|\nabla u|^2 dx dy = 0.
	\end{equation*}
	Значит 
	\begin{equation*}
		\dfrac{\partial u}{\partial x} = \dfrac{\partial u}{\partial y} = 0, (x,y) \in D^+,
	\end{equation*}
	т.к. $u(x,y) \rightrightarrows 0$ при $y \to +\infty$  получаем, что $u(x,y) \equiv 0$ в $D^+$. Теорема доказана.
	
	
	\section{Задача I-II}
	
	
	\textbf{Теорема 4.} \textit{Решение задачи (9)-(13) единственно для $k \in \mathbb{R}  \backslash \{0\}$}.
	
	\textbf{Доказательство.}
	Случай $k>0$ рассмотрен в предыдущей работе. Рассмотрим только случай $k < 0$.
	Как и ранее $D_{\varepsilon R} = (0, \pi) \times (\varepsilon, R)$, заметим что
	\begin{equation*}
		\iint\limits_{D_{\varepsilon R}} e^{-k x} u \Delta u dx dy = 0.
	\end{equation*}
	Рассмотрим это выражение подробнее
	\begin{equation*}
		\iint\limits_{D_{\varepsilon R}} e^{-kx} u u_{xx} dx dy = -\int\limits_{\varepsilon}^{R} u(0, y) u_x(0, y) dy + \int\limits_{\varepsilon}^{R} e^{-k\pi} u(\pi, y) u_x(\pi, y) dy - \iint\limits_{D_{\varepsilon R}} e^{-kx} \left(u_x^2 -k u u_x\right) dx dy  = 
	\end{equation*}
	\begin{equation*}
		= - \iint\limits_{D_{\varepsilon R}} e^{-kx} \left(u_x^2 -k u u_x\right) dx dy,
	\end{equation*}
	\begin{equation*}
		\iint\limits_{D_{\varepsilon R}} e^{-kx} u u_{yy} dx dy = -\int\limits_{0}^{\pi} e^{-kx} u(x,\varepsilon) u_y(x, \varepsilon) dx + \int\limits_{0}^{\pi} e^{-kx} u(x, R) u_y(x, R) dx - \iint\limits_{D_{\varepsilon R}} e^{-kx} u_y^2 dx dy.
	\end{equation*}
	Устремим $R \to + \infty$ и $\varepsilon \to 0+0$, тогда
		\begin{equation*}
	\int\limits_{0}^{\pi} e^{-kx} u(x, R) u_y(x, R) dx  \to 0,
	\end{equation*}
	\begin{equation*}
		\int\limits_{0}^{\pi} e^{-kx} u(x,\varepsilon) u_y(x, \varepsilon) dx = k \int\limits_0^\pi e^{-kx} u(x,0+0) u_x(x, 0+0) dx.
	\end{equation*}
	Собираем исходное выражение
	\begin{equation*}
		0 = - \iint\limits_{D_{\varepsilon R}} e^{-kx} \left(u_x^2 -k u u_x\right) dx dy + 
	\end{equation*}
	
		\textbf{Лемма 1.} \textit{Пусть $u(x,y)$ - решение задачи (1)-(4), тогда $u_y(x,y) \rightrightarrows 0$ при $ y \to \infty$}.
		
\textbf{Доказательство.}
Рассмотрим комплексное отображение $w(z) = e^{iz}$, где $z = x+iy$. Это отображение является конформным в области $D^+ = \{(x,y): x \in (0,\pi), y > 0\}$. Анализируем, как данное отображение преобразует нашу полуполосу:
\begin{itemize}
	\item \textbf{Внутренность полуполосы $x \in (0,\pi), y > 0$:} Отображается на верхнюю половину открытого единичного круга $\Omega_{int} = \{w = \sigma + i\tau : \sigma^2 + \tau^2 < 1, \tau > 0\}$.
	\item \textbf{Нижняя граница $y=0, x \in [0,\pi]$:} При $y=0$, $w = e^{ix} = \cos x + i\sin x$. Поскольку $x$ изменяется от $0$ до $\pi$, эта часть границы переходит в верхнюю полуокружность единичного круга, соединяющую точки $w=1$ (при $x=0$) и $w=-1$ (при $x=\pi$). Обозначим её $\partial \Omega_1 = \{w : |w|=1, \mathrm{Im}(w) \ge 0\}$.
	\item \textbf{Левая боковая граница $x=0, y \ge 0$:} При $x=0$, $w = e^{-y}$. Поскольку $y \ge 0$, $e^{-y}$ изменяется от $1$ (при $y=0$) до $0$ (при $y \to +\infty$). Эта часть границы переходит в отрезок $[0,1]$ вещественной оси.
	\item \textbf{Правая боковая граница $x=\pi, y \ge 0$:} При $x=\pi$, $w = e^{-y}(\cos\pi + i\sin\pi) = -e^{-y}$. Поскольку $y \ge 0$, $-e^{-y}$ изменяется от $-1$ (при $y=0$) до $0$ (при $y \to +\infty$). Эта часть границы переходит в отрезок $[-1,0]$ вещественной оси.
	\item \textbf{"Бесконечность" полуполосы $y \to +\infty$:} При $y \to +\infty$, $e^{-y} \to 0$, следовательно $w \to 0$. Таким образом, вся "бесконечность" полуполосы отображается в единственную точку $w=0$ (начало координат) в плоскости $w$.
\end{itemize}
Область $\overline{D^+}$ (включая границы и точку на бесконечности) переходит в замкнутый верхний полудиск $\overline{\Omega} = \{w = \sigma + i\tau : \sigma^2 + \tau^2 \le 1, \tau \ge 0\}$.

Функция $u(x,y)$ является гармонической в $D^+$. При конформном отображении гармонические функции остаются гармоническими. Следовательно, функция $U(w)$, определенная как $U(w(z)) = u(z)$, является гармонической в открытой области $\Omega_{int}$.

Рассмотрим граничные условия для $U(w)$:
\begin{itemize}
	\item Из условия (4) $u(x,y) \rightrightarrows 0$ при $y \to +\infty$ следует, что $U(w) \to 0$ при $w \to 0$. Это соответствует значению функции в центре полудиска.
	\item Из условий (2) $u(0,y) = 0$ и $u(\pi,y) = 0$ для $y>0$ следует, что $U(w)=0$ на отрезках $[0,1)$ и $(-1,0]$ вещественной оси соответственно. Включая точки $w=\pm 1$ (которые соответствуют $(0,0)$ и $(\pi,0)$ в $z$-плоскости, где $u$ также равна нулю), получаем, что $U(w) = 0$ для всех $w \in [-1,1]$ на вещественной оси.
\end{itemize}
Таким образом, для функции $U(w)$ имеем:
\begin{equation*}
	\begin{cases}
		\Delta U(w) = 0, & w \in \Omega_{int} \\
		U(w) = 0, & w \in (-1, 1) \\
		U(w) \to 0, & w \to 0
	\end{cases}
\end{equation*}

Теперь выразим $u_y(x,y)$ через производные функции $U(\sigma, \tau)$. Из соотношений $\sigma = e^{-y}\cos x$ и $\tau = e^{-y}\sin x$:
$$\frac{\partial \sigma}{\partial y} = -e^{-y}\cos x = -\sigma$$
$$\frac{\partial \tau}{\partial y} = -e^{-y}\sin x = -\tau$$
Применяя правило дифференцирования сложной функции:
$$u_y(x,y) = \frac{\partial}{\partial y} U(\sigma(x,y), \tau(x,y)) = \frac{\partial U}{\partial \sigma} \frac{\partial \sigma}{\partial y} + \frac{\partial U}{\partial \tau} \frac{\partial \tau}{\partial y}$$
$$u_y(x,y) = \frac{\partial U}{\partial \sigma} (-\sigma) + \frac{\partial U}{\partial \tau} (-\tau) = - \left( \sigma \frac{\partial U}{\partial \sigma} + \tau \frac{\partial U}{\partial \tau} \right)$$

Для исследования предела $u_y(x,y)$ при $y \to +\infty$, рассмотрим поведение производных $\frac{\partial U}{\partial \sigma}$ и $\frac{\partial U}{\partial \tau}$ при $w \to 0$.
Функция $U(w)$ гармонична в $\Omega_{int}$ и обращается в ноль на отрезке $(-1,1)$ вещественной оси. Мы можем использовать \textbf{принцип Шварца-отражения}. Определим функцию $U^*(w)$ в диске $B(0,1) = \{w: |w|<1\}$ следующим образом:
$$ U^*(w) = \begin{cases}
	U(w), & \mathrm{Im}(w) \geq 0 \\
	-U(\overline{w}), & \mathrm{Im}(w) < 0
\end{cases} $$
Поскольку $U(w)$ непрерывна вплоть до границы $\mathrm{Im}(w)=0$ и $U(w)=0$ на этой части границы, функция $U^*(w)$ является гармонической во всём открытом диске $B(0,1)$.
Гармонические функции бесконечно дифференцируемы в своей области определения. Поскольку $w=0$ является внутренней точкой диска $B(0,1)$, частные производные $\frac{\partial U^*}{\partial \sigma}$ и $\frac{\partial U^*}{\partial \tau}$ существуют, непрерывны и конечны в точке $w=0$. Следовательно, пределы:
$$\lim_{w \to 0} \frac{\partial U}{\partial \sigma} = \frac{\partial U^*}{\partial \sigma}(0) \quad \text{и} \quad \lim_{w \to 0} \frac{\partial U}{\partial \tau} = \frac{\partial U^*}{\partial \tau}(0)$$
существуют и конечны. Обозначим эти предельные значения $C_1 = \frac{\partial U^*}{\partial \sigma}(0)$ и $C_2 = \frac{\partial U^*}{\partial \tau}(0)$.

Теперь вернемся к выражению для $u_y(x,y)$:
$$u_y(x,y) = - \left( \sigma \frac{\partial U}{\partial \sigma} + \tau \frac{\partial U}{\partial \tau} \right) = - \left( e^{-y}\cos x \frac{\partial U}{\partial \sigma} + e^{-y}\sin x \frac{\partial U}{\partial \tau} \right)$$
$$u_y(x,y) = - e^{-y} \left( \cos x \frac{\partial U}{\partial \sigma} + \sin x \frac{\partial U}{\partial \tau} \right)$$
Рассмотрим предел при $y \to +\infty$:
$$\lim_{y \to +\infty} u_y(x,y) = \lim_{y \to +\infty} \left[ - e^{-y} \left( \cos x \frac{\partial U}{\partial \sigma}(w(x,y)) + \sin x \frac{\partial U}{\partial \tau}(w(x,y)) \right) \right]$$
При $y \to +\infty$, $w(x,y) \to 0$. Значения $\frac{\partial U}{\partial \sigma}(w)$ и $\frac{\partial U}{\partial \tau}(w)$ стремятся к конечным значениям $C_1$ и $C_2$ соответственно. Выражение в скобках $(\cos x \cdot C_1 + \sin x \cdot C_2)$ является конечным и ограниченным (поскольку $\cos x$ и $\sin x$ ограничены).
Множитель $e^{-y}$ экспоненциально стремится к нулю.
Следовательно, произведение также стремится к нулю:
$$\lim_{y \to +\infty} u_y(x,y) = 0$$. Лемма доказана.
\end{document}

