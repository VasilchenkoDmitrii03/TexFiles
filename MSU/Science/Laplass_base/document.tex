\documentclass[a4paper, 9pt]{article}

% paper settings (spacing, font etc.)
\usepackage[T2A]{fontenc}
\usepackage[utf8]{inputenc}
\usepackage[english, russian]{babel}
\usepackage{indentfirst}
\usepackage{setspace}\onehalfspacing
\usepackage[left=30mm, top=20mm, right=30mm, bottom=20mm, nohead, footskip=10mm]{geometry}

\usepackage{graphicx}   % include images
\usepackage{hyperref}   % hyperreferences in a rendered document

% math fonts
\usepackage{amsfonts, amssymb, amsmath, mathabx, dsfont}

% theorems, lemmas, etc.
\usepackage{amsthm}
\newtheorem*{theorem*}{Теорема}
\newtheorem{theorem}{Теорема}
\newtheorem{lemma}{Лемма}
\newtheorem{corollary}{Следствие}
\newtheorem{notabene}{Замечание}
\newtheorem{definition}{Определение}

% enumerating settings

\title{О краевой задаче для уравнения Лапласа со смешанными граничными условиями в полуполосе}
\author{Капустин Н.Ю., Васильченко Д.Д.}
\date{}
\begin{document}
	УДК 517.956
	\begin{center}		
		\LARGE{
		 Об одной задаче для уравнения Лапласа со смешанными граничными условиями
		}
		\newline
		\large{Капустин Н.Ю., Васильченко Д. Д.}
		
	\end{center}
	\par
	Рассмотрим краевую задачу для уравнения Лапласа
	\begin{equation}
		\dfrac{\partial^2 u}{\partial x^2} +\dfrac{\partial^2 u}{\partial y^2} = 0
	\end{equation}
	в полуполосе $D = \{(x,y) :\  0 < x < \pi, y > 0\}$	в классе функций $u(x,y) \in C(\overline{D}) \cap C^1(\overline{D} \cap \{y > 0\}) \cap C^2 (D)$\newline
	с граничными условиями
	\begin{equation}
		u(0, y) = 0, \ \dfrac{\partial u}{\partial x} (\pi, y) = 0, \ y > 0, 
	\end{equation}
	\begin{equation}
		\lim\limits_{y \to 0 + 0} \int\limits_0^\pi \left[\dfrac{\partial u}{\partial y}(x,y) - \dfrac{\partial u}{\partial x}(x,y) + \varphi(x) \right]^2 dx = 0, \ \varphi(x) \in L_2(0,\pi) , 
	\end{equation}
	\begin{equation}
		u(x,y) \rightrightarrows 0, y \to +\infty. 
	\end{equation}
	
	В работе будут доказаны теоремы существования и единственности решения
	этой задачи, а также получены интегральные представления для частных
	производных решения.
	
	Аналогичная задача рассматривалась как вспомогательная при изучении задачи Трикоми-Неймана
	для уравнения Лаврентьева-Бицадзе с граничными условиями второго рода на боковых сторонах
	полуполосы и коэффициентом $\dfrac{1}{k}$ при
	$\dfrac{\partial u}{\partial y}(x,y), \vert k\vert>1, $ в статье [1].
	На линии изменения типа ставилось условие склеивания нормальных производных
	по Франклю. Случай $k=1$ (классическая задача с непрерывным градиентом)
	не рассматривался и теорема единственности для вспомогательной задачи
	не доказывалась.
	
	На задачу Трикоми с эллиптической частью в виде полуполосы обратил внимание
	А.В. Бицадзе в связи с математическим моделированием плоскопараллельных
	движений газа. В данном случае построение решения конформным отображением
	приводится к краевой задаче относительно аналитической функции в верхней
	полуплоскости [2, стр. 327]. На основании известной формулы Шварца [2, стр. 315]
	А.В. Бицадзе было выписано в квадратурах решение этой краевой задачи. 
	
	В работе [4] получено интегральное представление регулярного решения задачи для уравнения Лапласа
	в полукруге с краевым условием первого рода на полуокружности и двумя различными
	краевыми условиями типа наклонной производной на двух прямолинейных участках границы.
	\par
	\textbf{Теорема 1.} \textit{Решение задачи (1 - 4) существует, причём его можно представить в виде ряда}
		\begin{equation}
			u(x,y) = \sum\limits_{n=0}^{\infty} A_n e^{-\left(n + \dfrac12\right)y} \sin{\left[\left(n + \dfrac12\right)x\right]},
		\end{equation}
		\textit{где коэффициенты $A_n, \ n =0,1,2, \dots$ определяются из разложения}
		\begin{equation}
			\sum\limits_{n=0}^{\infty} A_n \left(n + \dfrac12 \right) \sin{\left[\left(n +\dfrac12\right)x + \dfrac\pi4\right]} = \dfrac{\varphi(x)}{\sqrt2}.
		\end{equation}
	
	\par
	Доказательство. Докажем существование решения задачи $(1 - 4)$. В силу основного результата работы $[2]$ система  $\left\{\sin{\left[\left(n + \dfrac12\right)x + \dfrac\pi4\right]}\right\}_{n=0}^{\infty}$ образует базис Рисса в пространстве $L_2(0, \pi)$. Разложим $\dfrac{\varphi(x)}{\sqrt2}$ по этой системе. Коэффициенты разложения в формуле (6) удовлетворяют неравенствам Бесселя
	\begin{equation*}
		C_1 \|\varphi \|_{L_2(0,\pi)} \leq \sum\limits_{n=0}^{\infty} A_n^2 \left(n + \dfrac12\right)^2 \leq C_2 \|\varphi \|_{L_2(0,\pi)} ,\ 0 < C_1 < C_2, 
	\end{equation*}
	где $C_1, C_2$ не зависят от $\varphi$. Следовательно сходится ряд $\sum\limits_{n=0}^{\infty} |A_n|$ и сходится равномерно ряд (5). Дифференцировать ряд (5) по x в $D$ можно так как каждая из функций $A_n e^{-\left(n + \dfrac12\right)y} \sin{\left[\left(n + \dfrac12\right)x\right]}$ имеет в области $D$ производную, сам ряд (5) сходится равномерно и ряд производных 
	\begin{equation*}
		\sum\limits_{n=0}^{\infty} A_n \left(n+\dfrac12\right)e^{-\left(n + \dfrac12\right)y} \cos{\left[\left(n + \dfrac12\right)x\right]}
	\end{equation*}
	 сходится равномерно в $D$. Аналогично можно показать, что (5) можно дважды дифференцировать по x и y. Функция (5) является решением уравнения (1) и удовлетворяет граничному условию (2). Условие (4) выполняется так как  $\sum\limits_{n=0}^{\infty} e^{-\left(n + \frac12\right)y} = \dfrac{e^{-y/2}}{1 - e^{-y}}$. Проверим выполнение условия (3).\newline
	 
	Согласно разложению (6), условие (3) принимает вид
	\begin{equation*}
		I(y) =  2 \int\limits_0^\pi \left[	\sum\limits_{n=0}^{\infty} A_n\left(n+\dfrac12\right) \left( e^{-\left(n+\dfrac12\right)y} - 1\right) \sin{\left[\left(n+\dfrac12\right) x  + \dfrac\pi4\right]} \right]^2 dx
	\end{equation*}
	Докажем, что $I(y) \to 0$ при $y \to 0+0$. Запишем
	\begin{equation*}
		I(y) \leq I_1(y) + I_2(y), \text{ где}
	\end{equation*}
	\begin{equation*}
		I_1(y) = 4\int\limits_0^\pi \left[	\sum\limits_{n=0}^{m} A_n\left(n+\dfrac12\right) \left( e^{-\left(n+\dfrac12\right)y} - 1\right) \sin{\left[\left(n+\dfrac12\right) x  + \dfrac\pi4\right]} \right]^2 dx, 
	\end{equation*}
	\begin{equation*}
		I_2(y) = 4\int\limits_0^\pi \left[	\sum\limits_{n=m+1}^{\infty} A_n\left(n+\dfrac12\right) \left( e^{-\left(n+\dfrac12\right)y} - 1\right) \sin{\left[\left(n+\dfrac12\right) x  + \dfrac\pi4\right]} \right]^2 dx.
	\end{equation*}
	Зафиксируем  $\forall \varepsilon > 0$. В силу левой части неравенства Бесселя имеем оценку
	\begin{equation*}
		I_2(y) =  4\int\limits_0^\pi \left[	\sum\limits_{n=m+1}^{\infty} A_n\left(n+\dfrac12\right) \left( e^{-\left(n+\dfrac12\right)y} - 1\right) \sin{\left[\left(n+\dfrac12\right) x  + \dfrac\pi4\right]} \right]^2 dx \leq 
	\end{equation*}
	\begin{equation*}
		\leq  C_3 \sum\limits_{n=m+1}^{\infty} A_n^2 \left(n+\dfrac12\right)^2 \left(e^{-\left(n+\dfrac12\right)y} - 1\right)^2 \leq C_3 \sum\limits_{n=m+1}^{\infty} A_n^2 \left(n+\dfrac12\right)^2 < \dfrac{\varepsilon}{2},
	\end{equation*}
	если  $m$ достаточно велико.\newline
	
	Во втором слагаемом мы имеем дело с конечным числом элементов, поэтому:
	\begin{equation*}
		I_1(y) = 4\int\limits_0^\pi \left[	\sum\limits_{n=0}^{m} A_n\left(n+\dfrac12\right) \left( e^{-\left(n+\dfrac12\right)y} - 1\right) \sin{\left[\left(n+\dfrac12\right) x  + \dfrac\pi4\right]} \right]^2 dx \leq
	\end{equation*}
	\begin{equation*}
		\leq C_4 \sum\limits_{n=0}^{m} A_n^2 \left(n +\dfrac12\right)^2 \left(e^{-\left(n+\dfrac12\right)y} - 1\right)^2 < \dfrac{\varepsilon}{2}
	\end{equation*}
	при $0 < y < \delta$, если $\delta$ достаточно мало. Условие (3) выполнено. Теорема доказана.
	\par
		\textbf{Теорема 2.} \textit{Решение задачи (1- 4) единственно}
	\par
		Доказательство. Докажем единственность решения этой задачи. Пусть $u(x,y)$ - решение однородной задачи.
		Введём обозначения $C_\varepsilon = (0, \varepsilon), C_R = (0, R), D_R = (\pi, R), D_\varepsilon = (\pi, \varepsilon)$. $\prod_{R\varepsilon}$ - прямоугольник $C_\varepsilon C_R D_R D_\varepsilon$. Справедливы следующие соотношения:
		\begin{equation*}
			0 = \iint\limits_{\prod_{R\varepsilon}} (R-y) (u_{xx} + u_{yy}) dx dy = 
		\end{equation*}
		\begin{equation*}
			=	\iint\limits_{\prod_{R\varepsilon}} \left( \left(R - y\right) u_x u\right)_x dx dy  + \iint\limits_{\prod_{R\varepsilon}} \left( \left(R - y\right) u_y u\right)_y dx dy  
			- \iint\limits_{\prod_{R\varepsilon}} \left(R- y\right) \left(u_x^2 + u_y^2\right)dxdy + \iint\limits_{\prod_{R\varepsilon}} u_y u dx dy = 
		\end{equation*}
		\begin{equation*}
			= - \iint\limits_{\prod_{R\varepsilon}} \left(R - y\right) \left(u_x^2 + u_y^2\right) dx dy - 
			\int\limits_{C_\varepsilon D_\varepsilon} \left(R - \varepsilon \right) \left(u_y - u_x\right)u dx - \int\limits_{C_\varepsilon D_\varepsilon} \left(R - \varepsilon\right) u_x u dx - \int\limits_{C_\varepsilon D_\varepsilon}\dfrac{u^2}{2} dx +
		\end{equation*}
		\begin{equation*}
			+ \int\limits_{C_R D_R} \dfrac{u^2}{2}dx
		\end{equation*}
		Отсюда следует
		\begin{equation*}
			\int\limits_{C_\varepsilon D_\varepsilon} \left(R - \varepsilon \right) \left(u_x - u_y\right)u dx + \dfrac12  \int\limits_{C_R D_R} u^2 dx \leq
		\end{equation*}
		\begin{equation*}
			\leq \left(R - \varepsilon\right) \left[\int\limits_{C_\varepsilon D_\varepsilon} \left( u_y - u_x\right)^2 dx \right]^{\frac12} \left[\int\limits_{C_\varepsilon D_\varepsilon} u^2 dx \right]^{\frac12} + \dfrac12 \int\limits_{C_RD_R} u^2 dx \leq
		\end{equation*}
		\begin{equation*}
			\leq \left(R - \varepsilon\right)^2 \int\limits_{C_\varepsilon D_\varepsilon} \left( u_y - u_x\right)^2 dx + \dfrac14 \int\limits_{C_\varepsilon D_\varepsilon} u^2 dx +\dfrac12 \int\limits_{C_RD_R} u^2 dx, 
		\end{equation*}
		\begin{equation*}
			\iint\limits_{\prod_{R\varepsilon}} \left(R - y\right) \left(u_x^2 + u_y^2\right) dx dy + \dfrac{1}{4}\int\limits_{C_\varepsilon D_\varepsilon} u^2 dx +\dfrac{R - \varepsilon}{2}u^2(\pi, \varepsilon) \leq 
		\end{equation*}
		\begin{equation*}
			\leq \left(R - \varepsilon\right)^2 \int\limits_{C_\varepsilon D_\varepsilon} \left( u_y - u_x\right)^2 dx  +\dfrac12 \int\limits_{C_RD_R} u^2 dx
		\end{equation*}
		В силу (3)
		\begin{equation*}
			\lim\limits_{\varepsilon \to 0 + 0} \int\limits_{C_\varepsilon D_\varepsilon} \left(u_y - u_x\right)^2 dx = 0
		\end{equation*}
		отсюда вытекает соотношение
		\begin{equation*}
			\lim\limits_{\varepsilon \to 0 + 0} \iint\limits_{\prod_{R\varepsilon}} \left(R - y\right) \left(u_x^2 + u_y^2 \right) dx dy + \dfrac14 \int\limits_0^\pi u^2(x,0) dx + \dfrac{R}{2}u^2(\pi,0) \leq \dfrac12 \int\limits_{C_RD_R} u^2 dx
		\end{equation*}
		Устремим теперь $R \to \infty$, тогда $\int\limits_{C_RD_R} u^2 dx \to 0$, отсюда $u(x,y) \equiv 0$ в $\overline{D}$. Теорема доказана.
		\par
		\textbf{Теорема 3.} \textit{Пусть $u(x,y)$ - решение задачи $(1)-(4)$, тогда $u_x, u_y$ представимы в виде}
		\begin{equation}
			u_y(x,y) = - Im\  \dfrac{ \sqrt{1 - e^{i2z}} }{\pi} e^{\dfrac{iz}{2}} \int\limits_0^\pi  \dfrac{\sqrt{\sin{t}}}{\left(1 - e^{i(z+t)} \right) \left(1 - e^{i(z-t)}\right)}  \varphi(t) dt, 
		\end{equation}
		\begin{equation}
			u_x(x,y) = Re\   \dfrac{ \sqrt{1 - e^{i2z}} }{\pi} e^{\dfrac{iz}{2}} \int\limits_0^\pi  \dfrac{\sqrt{\sin{t}}}{\left(1 - e^{i(z+t)} \right) \left(1 - e^{i(z-t)}\right)}  \varphi(t) dt,
		\end{equation}
		где $z = x + iy$. 
		\par
		Доказательство. Рассмотрим равенство (6). Система синусов $\left\{\sin{\left[\left(n +\dfrac12\right)x + \dfrac\pi4\right]}\right\}_{n=0}^{\infty}$ образует базис в $L_2(0,\pi)$. Поэтому для коэффициентов $A_n\left(n+\dfrac12\right)$ справедливо следующее представление [2]:
		\begin{equation*}
			A_n\left(n+\dfrac12\right) = \int\limits_0^\pi h_{n+1}(t) \dfrac{\varphi(t)}{\sqrt2} dt, 
		\end{equation*}
		где
		\begin{equation*}
			h_n(t) = \dfrac{2}{\pi}\dfrac{(2\cos{t/2})^\beta}{(\tg{t/2})^{\gamma/\pi}} \sum\limits_{k=1}^n \sin{kt} B_{n-k}, \ B_{l} = \sum\limits_{m=0}^{l} C_{\gamma/\pi}^{l-m} C_{-\gamma / \pi -l}^{m} (-1)^{l-m}, \ C_l^n = \dfrac{l(l-1)\dots (l-n+1)}{n!}. 
		\end{equation*}
		\par
		Пусть $u(x,y)$ - решение задачи $(1 -4)$, тогда
		\begin{equation*}
			u(x,y) = \sum\limits_{n=0}^{\infty} A_n e^{-\left(n + \dfrac12\right)y} \sin{\left[\left(n + \dfrac12\right)x\right]}
		\end{equation*}
		и соотвественно
		\begin{equation*}
			u_y(x,y) = -\sum\limits_{n=0}^{\infty} A_n \left(n +\dfrac12\right) e^{-\left(n + \dfrac12\right)y} \sin{\left[\left(n + \dfrac12\right)x\right]}=
		\end{equation*}
		\begin{equation*}
		  = - \sum\limits_{n=0}^{\infty}  \int\limits_0^\pi \dfrac{\varphi(t)}{\sqrt2}  h_{n+1}(t)  e^{-\left(n + \dfrac12\right)y} \sin{\left[\left(n + \dfrac12\right)x\right]} dt, 
		\end{equation*}
		или
		\begin{equation*}
			  u_y(x,y) = -  Im \ \sum\limits_{n=0}^{\infty}  \int\limits_0^\pi \dfrac{\varphi(t)}{\sqrt2}  h_{n+1}(t)  e^{-\left(n + \dfrac12\right)y} e^{i\left(n + \dfrac12\right)x} dt = 
		\end{equation*}
		\begin{equation*}
			 = -  Im \ \sum\limits_{n=0}^{\infty}  \int\limits_0^\pi \dfrac{\varphi(t)}{\sqrt2}  h_{n+1}(t)  e^{i\left(n+\dfrac12\right) z}  dt = \vert m = n + 1 \vert = 
		\end{equation*}
		\begin{equation*}
			 = -  Im \ \sum\limits_{m=1}^{\infty}  \int\limits_0^\pi \dfrac{\varphi(t)}{\sqrt2}  h_{m}(t)  e^{i\left(m-\dfrac12\right) z}  dt = 
		\end{equation*}
		\begin{equation*}
			 = -  Im \ e^{-\dfrac{iz}{2}}\ \sum\limits_{m=1}^{\infty}  \int\limits_0^\pi \dfrac{\varphi(t)}{\sqrt2}  h_{m}(t)  e^{im z}  dt .
		\end{equation*}
		Поменяем местами знаки интергирования и суммирования
		\begin{equation*}
			u_y(x,y)  = -  Im \ e^{-\dfrac{iz}{2}}\  \int\limits_0^\pi \dfrac{\varphi(t)}{\sqrt2}  \sum\limits_{m=1}^{\infty}   h_{m}(t)  e^{im z}  dt
		\end{equation*}
		Введём новое обозначение:
		\begin{equation*}
			I(t,z) = \sum\limits_{m=1}^{\infty}  h_{m}(t)  e^{im z}
		\end{equation*}
		\begin{equation*}
			I(t,z) =\dfrac{2}{\pi}\dfrac{(2\cos{t/2})^\beta}{(\tg{t/2})^{\gamma/\pi}} \sum\limits_{n=1}^{\infty}   \sum\limits_{k=1}^n \sin{kt} B_{n-k} e^{inz} = 
			\dfrac{2}{\pi}\dfrac{(2\cos{t/2})^\beta}{(\tg{t/2})^{\gamma/\pi}} \sum\limits_{k=1}^{\infty} \sin{kt} \sum\limits_{n=k}^{\infty} e^{inz} B_{n-k}
		\end{equation*}
		и новый индекс $m = n - k$
		\begin{equation*}
			I(t,z) = \dfrac{2}{\pi}\dfrac{(2\cos{t/2})^\beta}{(\tg{t/2})^{\gamma/\pi}} \sum\limits_{k=1}^{\infty} \sin{kt} \sum\limits_{m =0 }^{\infty} e^{i(m+k)z} B_{m} = 
			\dfrac{2}{\pi}\dfrac{(2\cos{t/2})^\beta}{(\tg{t/2})^{\gamma/\pi}} \sum\limits_{k=1}^{\infty} e^{ikz}\sin{kt} \sum\limits_{m =0 }^{\infty} e^{imz} B_{m}
		\end{equation*}
		\begin{equation*}
			\sum\limits_{k=1}^{\infty} e^{ikz}\sin{kt} =  \dfrac{e^{iz} \sin{t}}{\left(1 - e^{i(z+t)} \right) \left(1 - e^{i(z-t)}\right)}
		\end{equation*}
		Рассмотрим второй ряд:
		\begin{equation*}
			\sum\limits_{l =0 }^{\infty} e^{ilz} B_{l} = \sum\limits_{l =0 }^{\infty} e^{ilz} \sum\limits_{m=0}^{l} C^{l - m}_{\gamma/\pi} C^{m}_{-\gamma/\pi - \beta} (-1)^{l-m} = \sum\limits_{m=0}^{\infty} \sum\limits_{l=m}^{\infty} e^{ilz} C^{l - m}_{\gamma/\pi} C^{m}_{-\gamma/\pi - \beta} (-1)^{l-m} = 
		\end{equation*}
		\par
		Введём новый индекс суммирования $k = l -m$
		\begin{equation*}
			\sum\limits_{m=0}^{\infty} \sum\limits_{k=0}^{\infty} e^{i(m+k)z} C^{k}_{\gamma/\pi} C^{m}_{-\gamma/\pi - \beta} (-1)^{k} = \sum\limits_{m=0}^{\infty} e^{imz} C^{m}_{-\gamma/\pi - \beta} \sum\limits_{k=0}^{\infty}  C^{k}_{\gamma/\pi} (-1)^k e^{ikz} = (1 + e^{iz})^{-\gamma/\pi - \beta} (1- e^{iz})^{\gamma/\pi} =
		\end{equation*}
		\begin{equation*}
			= (1 + e^{iz})^{1/2} (1- e^{iz})^{1/2} =\sqrt{1 - e^{i2z}},
		\end{equation*}
		так как в нашем случае $\beta = -1, \ \gamma = \pi/2$.
		Окончательно получаем формулу
		\begin{equation*}
			u_y(x,y) = - Im\ e^{\dfrac{-iz}{2}} \int\limits_0^\pi \dfrac{\varphi(t)}{\sqrt2} I(t,z) dt =
		\end{equation*}
		\begin{equation*}
			 = - Im\ e^{\dfrac{-iz}{2}} \int\limits_0^\pi \dfrac{2}{\pi}\dfrac{(2\cos{t/2})^\beta}{(\tg{t/2})^{\gamma/\pi}}  \dfrac{e^{iz} \sin{t}}{\left(1 - e^{i(z+t)} \right) \left(1 - e^{i(z-t)}\right)} \sqrt{1 - e^{i2z}} \dfrac{\varphi(t)}{\sqrt2} dt =
		\end{equation*}
		\begin{equation*}
			 = - Im\  \dfrac{2}{\pi} e^{\dfrac{-iz}{2}} \int\limits_0^\pi \dfrac{1}{2\cos{t/2} \sqrt{\tan{t/2}}}  \dfrac{e^{iz} \sin{t}}{\left(1 - e^{i(z+t)} \right) \left(1 - e^{i(z-t)}\right)} \sqrt{1 - e^{i2z}} \dfrac{\varphi(t)}{\sqrt2} dt, 
		\end{equation*}
		т.е. представление:
		\begin{equation*}
			u_y(x,y) = - Im\  \dfrac{ \sqrt{1 - e^{i2z}} }{\pi} e^{\dfrac{iz}{2}}  \int\limits_0^\pi  \dfrac{\sqrt{\sin{t}} }{\left(1 - e^{i(z+t)} \right) \left(1 - e^{i(z-t)}\right)}  \varphi(t) dt.
		\end{equation*}
	 	Рассуждая аналогично, получим представление
	 		\begin{equation*}
	 			u_x(x,y) = Re\   \dfrac{ \sqrt{1 - e^{i2z}} }{\pi} e^{\dfrac{iz}{2}} \int\limits_0^\pi  \dfrac{\sqrt{\sin{t}}}{\left(1 - e^{i(z+t)} \right) \left(1 - e^{i(z-t)}\right)}  \varphi(t) dt.
	 		\end{equation*}
		Теорема доказана.
	\par
	Работа выполнена при финансовой поддержке Министерства науки и высшего образования Российской Федерации в рамках реализации программы Московского центра фундаментальной и прикладной математики по соглашению № 075-15-2022-284.
	\newline
	\newline
	\addcontentsline{toc}{section}{Литература}
	\vspace{-2.3cm}
	\renewcommand{\refname}{\begin{center}
			{\normalsize \rm СПИСОК ЛИТЕРАТУРЫ} \end{center}}
	
	\begin{thebibliography}{99} \itemsep=-2pt \vspace{-0.8cm}
		
		\selectlanguage{russian}
		
		\bibitem{Моисеев Е.И.} \textit{Моисеев Е.И.} \textit{Моисеев Т.Е.} \textit{Вафадорова Г.О.} Об интегральном представлении задачи Неймана-Трикоми для уравнения Лаврентьева-Бицадзе ~// Дифференциальные уравнения, \textbf{2015} Т. 51. №8. С.1070-1075
		
		\bibitem{Моисеев Е.И.} \textit{Моисеев Е.И.} О базисности одной системы синусов ~// Дифференциальные уравнения, \textbf{1987} Т. 23. №1. С.177-189
		
		\bibitem{Бицадзе А.В.} \textit{Бицадзе А.В.} Некоторые классы уравнений в частных производных.~// М., Наука,\textbf{1981}, 448 стр. 
		\bibitem{Моисеев Т. Е.} \textit{Моисеев Т. Е.} Об интегральном представлении решения уравнения Лапласа со
		смешанными краевыми условиями ~// Дифференциальные уравнения, \textbf{2011}, т. 47, №10, с.1446-1451. 
	\end{thebibliography}
\end{document}