\documentclass{beamer}
\usepackage[russian]{babel}
\usepackage{amsmath, amssymb}

\title{Задача Трикоми для уравнения Лаврентьева-Бицадзе}
\author{Капустин Н.Ю., Васильченко Д.Д.}
\date{}

\begin{document}
	
	\frame{\titlepage}
	
	\begin{frame}
		\frametitle{Постановка задачи}
		Рассматривается задача Трикоми для уравнения Лаврентьева-Бицадзе:
		\[
		\left(\text{sgn}(y)\right) \frac{\partial^2 u}{\partial x^2}(x,y) + \frac{\partial^2 u}{\partial y^2}(x,y) = 0 \eqno{(1)}
		\]
		в области $D = D^+ \cup D^-$, где:
		\[
		D^+ = \left\{(x,y): 0 < x < \pi, 0 < y < +\infty\right\}
		\]
		\[
		D^- = \left\{(x,y): -y < x < y + \pi, -\pi/2 < y < 0\right\}
		\]
		В классе функций 
		\[
		u(x,y) \in C^2(D^+)\cap C^2(D^-)\cap C(\overline{D^+\cup D^-})
		\]
	\end{frame}
	
	\begin{frame}
		\frametitle{Граничные условия}
		\[
		u(0, y) = 0, \quad u(\pi, y) = 0, \quad 0 < y < +\infty \eqno{(2)}
		\]
		\[
		u(x, -x) = f(x), \quad 0 \leq x \leq \pi/2, \quad f(0) = 0 \eqno{(3)}
		\]
		\[
		u(x, y) \rightrightarrows 0, \text{при} y \to +\infty, \eqno{(4)}
		\]
		\[
		\frac{1}{k} \frac{\partial u}{\partial y}(x, +0) = \frac{\partial u}{\partial y}(x, -0), \quad 0 < x < \pi, \eqno{(5)}
		\]
		\[
		  k\in (-\infty, +\infty), \ k \neq 0
		\]
	\end{frame}
	
	\begin{frame}
		\frametitle{Основные результаты}
		\textbf{Теорема 1:} Решение задачи (1) - (5) единственно.
		Преобразуем условие склеивания (5) в 
		\[
			\dfrac{1}{k}\dfrac{\partial u}{\partial y} - \dfrac{\partial u}{\partial x} \vert_{y = 0+0} = 0. \eqno{(6)}
		\]
		Используя общую формулу решения уравнения (1) в области $D^-$, получим
		\[
			\dfrac{1}{k}\dfrac{\partial u}{\partial y}(x, 0+0) - \dfrac{\partial u}{\partial x}(x, 0+0) = - f'(\dfrac{x}{2}), \ 0 < x < \pi
		\]
		
	\end{frame}
	
	\begin{frame}
		\frametitle{Основные результаты}
		Получаем в $D^+$ вспомогательную задачу
		\[
			\dfrac{\partial^2 u}{\partial x^2}(x,y) + \dfrac{\partial^2 u}{\partial y^2}(x,y) = 0 \eqno{(8)}
		\]
		с граничными условиями
		\[
			u(0,y) = 0, \ u(\pi, y) = 0, \ 0 < y < +\infty, \eqno{(9)}
		\]
		\[
			\dfrac{1}{k} \dfrac{\partial u}{\partial y}(x,0+0) - \dfrac{\partial u}{\partial x}(x,0+0) = -f'\left(\dfrac{x}{2}\right), \eqno{(10)}
		\]
		\[
			u(x,y) \rightrightarrows 0, \ y \to +\infty. \eqno{(11)}
		\]
	\end{frame}
	
	\begin{frame}
		\frametitle{Основные результаты}
			\textbf{Теорема 2.} \textit{Пусть $|k| < 1, \ k \neq 0$, $f(x) \in C[0, \pi/2] \cap C^2(0, \pi/2)$, $f'(x) \in L_2(0, \pi/2)$. Тогда решение задачи (8)-(11) существует и представимо в виде ряда
				\begin{equation*}
					u(x,y) = \sum\limits_{n=1}^{\infty} A_n e^{-ny} \sin{nx},
			\end{equation*}}
			причем условие (10) понимается в интегральном смысле
			\[
				\int\limits_0^\pi \left[	\dfrac{1}{k} \dfrac{\partial u}{\partial y}(x,y) - \dfrac{\partial u}{\partial x}(x,y) + f'\left(\dfrac{x}{2}\right)\right]^2 dx \to 0, \ y \to 0 + 0,
			\]
			а коэффициенты $A_n$ определяется из равенства
			\[
				\sum\limits_{n=1}^{\infty} n A_n \sin{\left[nx + \arctg{k}\right]} = \dfrac{k}{\sqrt{1 + k^2}} f'\left(\dfrac{x}{2}\right) 
			\]
	\end{frame}
	
	\begin{frame}
		\frametitle{Основные результаты}
		\textbf{Теорема 3.} \textit{Пусть $k > 0$, тогда решение задачи (8) - (11) единственно}
		
\textbf{Теорема 4.} \textit{Пусть $|k| < 1\  , k \neq 0$ и $u(x,y)$  - решение задачи (8) - (11), тогда $u_x, u_y$ представимы в виде}
\begin{equation*}
	u_y(x,y) = -\dfrac{2k}{\pi\sqrt{1+k^2}} \mathrm{Im}\ \left(\dfrac{1- e^{iz}}{1 + e^{iz}} \right)^{\gamma/\pi} e^{iz}\int\limits_0^\pi M(t,z) f'(\dfrac{t}{2}) dt, 
\end{equation*}
\begin{equation*}
	u_x(x,y) = \dfrac{2k}{\pi\sqrt{1+k^2}} \mathrm{Re}\ \left(\dfrac{1- e^{iz}}{1 + e^{iz}} \right)^{\gamma/\pi} e^{iz}\int\limits_0^\pi M(t,z)  f'(\dfrac{t}{2}) dt,
\end{equation*}
где $M(t,z) = \dfrac{1}{(\tg{t/2})^{\gamma/\pi}}  \dfrac{ \sin{t}}{\left(1 - e^{i(z+t)} \right) \left(1 - e^{i(z-t)}\right)}$,$\gamma = 2\arctg{k}$, $z = x + iy$.
	\end{frame}
	
	
	\begin{frame}
		\frametitle{Заключение}
		В работе решена задача Трикоми для уравнения Лаврентьева-Бицадзе в области с полуполосой в эллиптической части. Были доказаны теоремы об существовании и единственности решений при различных значениях параметра $k$ и найдены интегральные представления для производных решения вспомогательной задачи.
	\end{frame}
	
\end{document}