\documentclass[14pt]{article}
\usepackage[russian]{babel}
\usepackage{amsmath}
\usepackage{amssymb}
\usepackage[%
left=1.00in,%
right=1.00in,%
top=1.0in,%
bottom=1.0in,%
paperheight=11in,%
paperwidth=8.5in%
]{geometry}%
\title{Задача Трикоми для уравнения Лаврентьева-Бицадзе с полуполосой в эллиптической части}
\author{Капустин Н.Ю., Васильченко Д.Д.}
\date{}
\begin{document}
	\maketitle
		\par
	Рассматривается задача Трикоми для уравнения Лавертьева-Бицадзе
	\begin{equation}
		\left(sgn (y)\right) \dfrac{\partial^2 u}{\partial x^2}(x,y) + \dfrac{\partial^2 u}{\partial y^2}(x,y) = 0
	\end{equation}
	в области $D = D^{+} \cup D^{-}$, где $D^{+} = \left\{(x,y): \ 0 < x < \pi, \ 0 < y < + \infty \right\}$, 
	 \newline$D^{-} = \left\{(x,y): \ -y < x < y + \pi, \ -\pi/2 < y < 0\right\}$ в классе функций $u(x,y) \in C^2(D^{+}) \cap C^2(D^{-}) \cap C(\overline{D^{+} \cup D^{-}})$ с граничными условиями
	\begin{equation}
		u(0,y) = 0, \ u(\pi, y) = 0, \ 0 < y < + \infty,
	\end{equation}
	\begin{equation}
		u(x,-x) = f(x) , \ 0 \leq x \leq \pi/2, \ f(0) = 0, 
	\end{equation}
	\begin{equation}
		u(x,y) \rightrightarrows 0, \ y \to +\infty
	\end{equation}
	и условием склеивания Франкля 
	\begin{equation}
		\dfrac1{k} \dfrac{\partial u}{\partial y}(x, +0) = \dfrac{\partial u}{\partial y} (x, -0), \ 0 < x < \pi, 
	\end{equation}
	где $k \in (-\infty, +\infty), k \neq 0$.
	\par
		\textbf{Теорема 1.} \textit{Решение задачи (1) - (5) единственно.}
		
	\textbf{Доказательство.} 
	Пусть существуют два решения $u_1(x,y), u_2(x,y)$ задачи (1)-(5). Тогда $u(x,y) = u_1(x,y) - u_2(x,y)$ есть решение задачи (1)-(5) с функцией $f(x) \equiv 0$. В этом случае $u(x,y) = F(x+y) - F(0)$.
	
	Отсюда следует, что равенство $\dfrac{\partial u}{\partial y} - \dfrac{\partial u}{\partial x} = 0$ выполняется для всех точек x и y из области гиперболичности. Используя условие склеивания  (5) будем иметь
	\begin{equation}
		\dfrac1{k} \dfrac{\partial u}{\partial y} - \dfrac{\partial u}{\partial x}\vert_{y=0+0} = 0.
	\end{equation}
	В результате получаем задачу для нахождения гармонической функции $u(x,y)$ в области $D^{+}$ с граничными условиями (2),(4),(6).
	
	В силу принципа Зарембы-Жиро и равенства (6) экстремум не может достигаться на интервале $\{(x,y):\ 0 < x < \pi, \ y = 0\}$. На замкнутых боковых сторонах и на бесконечности экстремум не может достигаться в силу условий (2) и (4). Теорема доказана.
	
	Известно, что общее решение в $D^{-}$  уравнения (1) имеет вид 
	\begin{equation}
		u(x,y) = F(x+y) + f(\dfrac{x-y}{2}) - F(0).
	\end{equation}
	Продифференцируем равенство (7):
	\begin{equation*}
		\dfrac{\partial u}{\partial y}(x,y) - \dfrac{\partial u}{\partial x}(x,y) \vert_{y=0+0} = -f'\left(\dfrac{x}2\right), \ 0 < x < \pi.
	\end{equation*}
	Используя условие склеивания (5), приходим к равенству
	\begin{equation*}
		\dfrac{1}{k} \dfrac{\partial u}{\partial y}(x, 0+0) - \dfrac{\partial u}{\partial x}(x, 0 + 0) = - f'\left(\dfrac{x}2\right), \ 0 < x < \pi. 
	\end{equation*}
	
	Тогда получим в области $D^{+}$ вспомогательную задачу для оператора Лапласа 
	\begin{equation}
		\dfrac{\partial^2 u}{\partial x^2}(x,y) + \dfrac{\partial^2 u}{\partial y^2}(x,y) = 0
	\end{equation}
	с граничными условиями 
	\begin{equation}
		u(0,y) = 0, \ u(\pi, y) = 0, \ 0 < y < +\infty, 
	\end{equation}
	\begin{equation}
		\dfrac{1}{k} \dfrac{\partial u}{\partial y}(x,0+0) - \dfrac{\partial u}{\partial x}(x,0+0) = -f'\left(\dfrac{x}{2}\right),
	\end{equation}
	\begin{equation}
		u(x,y) \rightrightarrows 0, \ y \to +\infty 
	\end{equation}
	\par
	\textbf{Теорема 2.} \textit{Пусть $|k| < 1, \ k \neq 0$, $f(x) \in C[0, \pi/2] \cap C^2(0, \pi/2)$, $f'(x) \in L_2(0, \pi/2)$. Тогда решение задачи (8)-(11) существует и представимо в виде ряда
	\begin{equation}
		u(x,y) = \sum\limits_{n=1}^{\infty} A_n e^{-ny} \sin{nx},
		\end{equation}}
	причем условие (10) понимается в интегральном смысле
	\begin{equation*}
		\int\limits_0^\pi \left[	\dfrac{1}{k} \dfrac{\partial u}{\partial y}(x,y) - \dfrac{\partial u}{\partial x}(x,y) + f'\left(\dfrac{x}{2}\right)\right]^2 dx \to 0, \ y \to 0 + 0,
	\end{equation*}
	а коэффициенты $A_n$ определяется из равенства
	\begin{equation}
		\sum\limits_{n=1}^{\infty} n A_n \sin{\left[nx + \arctg{k}\right]} = \dfrac{k}{\sqrt{1 + k^2}} f'\left(\dfrac{x}{2}\right) 
	\end{equation}
	
	\par
\textbf{Теорема 3.} \textit{Пусть $k > 0$, тогда решение задачи (8) - (11) единственно}

\par
\textbf{Теорема 4.} \textit{Пусть $|k| < 1\  , k \neq 0$ и $u(x,y)$  - решение задачи (8) - (11), тогда $u_x, u_y$ представимы в виде}
\begin{equation*}
		u_y(x,y) = -\dfrac{2k}{\pi\sqrt{1+k^2}} \mathrm{Im}\ \left(\dfrac{1- e^{iz}}{1 + e^{iz}} \right)^{\gamma/\pi} e^{iz}\int\limits_0^\pi \dfrac{1}{(\tg{t/2})^{\gamma/\pi}}  \dfrac{ \sin{t}}{\left(1 - e^{i(z+t)} \right) \left(1 - e^{i(z-t)}\right)}  f'(\dfrac{t}{2}) dt, 
\end{equation*}
\begin{equation*}
	u_x(x,y) = \dfrac{2k}{\pi\sqrt{1+k^2}} \mathrm{Re}\ \left(\dfrac{1- e^{iz}}{1 + e^{iz}} \right)^{\gamma/\pi} e^{iz}\int\limits_0^\pi \dfrac{1}{(\tg{t/2})^{\gamma/\pi}}  \dfrac{ \sin{t}}{\left(1 - e^{i(z+t)} \right) \left(1 - e^{i(z-t)}\right)}  f'(\dfrac{t}{2}) dt,
\end{equation*}
где $\gamma = 2\arctg{k}$, $z = x + iy$.



	\begin{center}{ФИНАНСИРОВАНИЕ РАБОТЫ}\end{center}
	\par
	Работа выполнена при финансовой поддержке Министерства науки и высшего образования Российской Федерации в рамках реализации программы Московского центра фундаментальной и прикладной математики по соглашению № 075-15-2022-284.
	\bigskip
	
		\smallskip
	\begin{center}{СПИСОК ЛИТЕРАТУРЫ}\end{center}{\small
		
		1. Моисеев,~Е.И. Об интегральном представлении задачи Неймана-Трикоми для уравнения Лаврентьева-Бицадзе/~Моисеев,~Е.И., Моисеев~Т.Е., Вафадорова~Г.О. //~Дифференц. уравнения. ~--- 2015.~--- Т.~51, №~8~---. С.~1070--1075.
		
		2. Моисеев,~Е.И.  О базисности одной системы синусов /~Е.И. Моисеев //~Дифференц. уравнения. ~--- 1987.~---Т.~23,  №~1~--- С.~177--189.
	\end{document}