\documentclass[
11pt,%
tightenlines,%
twoside,%
onecolumn,%
nofloats,%
nobibnotes,%
nofootinbib,%
superscriptaddress,%
noshowpacs,%
centertags]%
{revtex4}
\usepackage{ljm}
\begin{document}

\titlerunning{LIMIT THEOREMS FOR ONE CLASS OF LARGE-DIMENSIONAL SYSTEMS} % for running heads
\authorrunning{Demidenko} % for running heads
%\authorrunning{First-Author, Second-Author} % for running heads

\title{Limit Theorems for One Class of Large-Dimensional Systems of~Ordinary Differential Equations
}
% Splitting into lines is performed by the command \\
% The title is written in accordance with the rules of capitalization.

\author{\firstname{G.~V.}~\surname{Demidenko}}
\email[E-mail: ]{demidenk@math.nsc.ru} \affiliation{Sobolev
Institute of Mathematics, Novosibirsk, 630090 Russia}
\affiliation{Novosibirsk State University, Novosibirsk, 630090
Russia}

\firstcollaboration{(Submitted by A. B. Muravnik)} % Add if you know submitter.
%\lastcollaboration{ }


\received{June 28, 2024; revised July 15, 2024; accepted July 26, 2024} % The date of receipt to the editor, i.e. December 06, 2017


\begin{abstract} % You shouldn't use formulas and citations in the abstract.
Large-dimensional systems of ordinary differential equations simulating
multistage synthesis are considered. New limit theorems are proved,
connections between solutions to these systems and delay equations are established.
\end{abstract}

\subclass{34A45, 34K07} % Enter 2010 Mathematics Subject Classification.

\keywords{large-dimensional systems of ordinary differential equations,
limit theorems, delay equations, Sobolev spaces
} % Include keywords separeted by comma.

\maketitle

% Text of article starts here.

\section{Introduction}

In this paper, we continue to study connections between solutions to
the Cauchy problem for systems of ordinary differential equations of
the following form
\begin{equation}\label{eq1}
\displaystyle{\frac{d{x}}{dt}} = {A}_{n} {x} + F(t,{x}), \qquad t > 0,
\end{equation}
\begin{equation}\label{eq2}
{A}_{n}=\!\left(\!
  \begin{array}{ccccc}
      \displaystyle-\frac{n-1}{\tau} & 0 & \cdots
      & \cdots & 0
  \\ [10pt]
      \ \ \displaystyle\frac{n-1}{\tau}
      & \displaystyle-\frac{n-1}{\tau} & \ddots
      & \ddots & \vdots
  \\ [10pt]
      0 & \ddots & \ddots & \ddots & \vdots
  \\ [10pt]
      \vdots & \ddots & \ddots
      & \displaystyle-\frac{n-1}{\tau} & 0
  \\ [10pt]
      0  & \cdots & 0
      & \ \ \displaystyle\frac{n-1}{\tau} & - \theta
  \end{array}
\right),
\qquad n \gg 1,
\end{equation}
$$
{x}(t) = ({x}_1(t), {x}_2(t), \dots, {x}_n(t))^{\rm T},
\quad
F(t,{x}) = (g(t,{x}_{n}), 0, \dots, 0)^{\rm T}, \quad \tau, \ \theta > 0,
$$
and solutions to initial value problems for the delay equation
\begin{equation}\label{eq3}
\frac{dy(t)}{dt} = -\theta y(t) + g(t-\tau,y(t-\tau)), \qquad t > \tau.
\end{equation}

Remind that for the first time such connections were established in \cite{1}
when solving the problem of large dimension for multistage substance synthesis
modeled by \eqref{eq1}. The component
${x}_i(t)$
of the vector-function
${x}(t)$
defines the concent\-ration of the substance at the $i$th stage,
$\tau >0$
is the total duration time of the stages from the first stage to the $n$th stage,
$\theta > 0$
is the dissipation parameter. The first equation defines the initiation law,
while the last one determines the utilization law of the substance, intermediate equations
describe dynamics of the substance synthesis (see, for example, \cite{2, 3, 4}).
Note that processes of substance synthesis can have hundreds of thousands
of intermediate stages. In the substance synthesis problem, first of all biologists
are interested in finding the concentration of the final product. Therefore, solving
the Cauchy problem for \eqref{eq1}, \eqref{eq2}, researchers may encounter serious
difficulties, since obtaining analytical formulas for solutions, with rare exceptions,
is impossible due to nonlinearity of the function
$g(t,z)$.
Direct solving \eqref{eq1}, \eqref{eq2} by a computer may be unrealistic
due to enormous dimension
$n \gg 1$.
Thus, a large-dimensional problem arises when modeling biological processes.

A solution to this problem was obtained in \cite{1} by finding approximate
values of the last component
$x_n(t)$
of solutions to the Cauchy problem
\begin{equation}\label{eq4}
\displaystyle{\frac{d{x}}{dt}} = {A}_{n} {x} + F(t,{x}), \quad x|_{t=0} = x^0,
\end{equation}
for $n \gg 1$. The basis for this method was V.A.~Likhoshvai's
hypothesis about possible connections between two different
approaches of biologists when modeling processes of multistage
substance synthesis with very large number of intermediate stages.
The first approach is based on the use of systems of the form
\eqref{eq1} and \eqref{eq2}, the second one is based on the use of
delay equations of the form \eqref{eq3}. Note that these approaches
give, generally speaking, different results for small $n$. Numerical
experiments conducted by S.I.~Fadeev for several systems showed that
differences decrease with increasing $n$. The analytical proof of
this hypothesis for $n \gg 1$ was given by the author in \cite{1}.

Below we briefly describe the idea of the proof of the hypothesis.
We indefinitely increase the number $n$ of the equations in
\eqref{eq1} and consider the series of the Cauchy problems of the
form \eqref{eq4}. If the function $g(t,u)$ satisfies the Lipschitz
condition of $u,$ then, for every fixed $n$, the problem \eqref{eq4}
is uniquely solvable on a certain segment. Then, we get the sequence
of the functions $\{x_n^n(t)\}$ consisting of the last components of
the solutions to the sequence of the Cauchy problems of the form
\eqref{eq4}. In \cite{1}, the following statement on the convergence
of this sequence was proved.

\begin{theorem}
\label{th1}
Let
$x^0 = (0,\ldots,0)^{\rm T}$, $F(t,x) = (g(x_n),0,\ldots,0)^{\rm T}$,
$$
|g(u)| \le G < \infty, \quad |g(u_1)-g(u_2)| \le L|u_1-u_2|,
$$
and let the sequence $\{x_n^n(t)\}$ consist of the last components
of the solutions to the series of the Cauchy problems \eqref{eq4}
for \eqref{eq1}. Then, there exists $t_0 > \tau$ such that the
sequence $\{x_n^n(t)\}$ converges uniformly on the segment $[0,
t_0]$
\begin{equation}\label{eq5}
x^n_n(t) \to y(t), \quad n \to \infty,
\end{equation}
and the limit function
$y(t)$
is the solution to the initial value problem for the delay equation
\begin{equation}\label{eq6}
\frac{dy(t)}{dt} = -\theta y(t) + g(y(t-\tau)), \quad t > \tau,
\end{equation}
\begin{equation}\label{eq7}
y(t) = 0, \quad 0 \le t \le \tau, \quad y(\tau+0) = 0.
\end{equation}
\end{theorem}



limit theorems were proved for some classes of high-dimensional
systems of ordinary differential equations.

In this paper, we present a number of more general limit theorems for non-autonomous

$F(t,x) = (g(t,x_n),0,\ldots,0)^{\rm T}$
which is not bounded.



\section{Limit theorems}

 $F(t,x) =
(g(t,x_n),0,\ldots,0)^{\rm T}$ which is not bounded. We assume that
$g(t,u)$ is a continuous function in $R^2_+$ such that
\begin{equation}\label{eq8}
|g(t,u)| \le G(1 + |u|) < \infty, \ \ t \ge 0, \ \ u \in R, \ \ |g(t,u_1)-g(t,u_2)| \le L|u_1-u_2|,
\end{equation}
where $G$ and  $L$ are constants.

\begin{theorem}
\label{th2} Let the sequence $\{x_n^n(t)\}$ consist of the last
components of the solutions to the series of the Cauchy problems

continuous vector-function $F(t,x) = (g(t,x_n),0,\ldots,0)^{\rm T}$

uniformly on any segment $[0, T]$, $T > \tau$
$$
x^n_n(t) \to y(t), \quad n \to \infty,
$$
and the limit function
$y(t)$
is the solution to the initial value problem for the delay equation
$$
\left\{%
\begin{array}{l}
\displaystyle
\frac{dy(t)}{dt} = -\theta y(t) + g(t-\tau,y(t-\tau)), \quad t > \tau,
\\
y(t) = 0, \quad 0 \le t \le \tau, \quad y(\tau+0) = 0.
\end{array}%
\right.
$$
\end{theorem}

By analogy with Theorem~\ref{th1}, Theorem~\ref{th2} provides a mathematical justification
for approximate calculating the concentration
$x_n(t)$
of the resulting substance on any segment
$[0, T]$
for
$n \gg 1$
under less restrictive conditions on the nonlinear function
$g(t,u)$.

We now give a number of limit theorems for the Cauchy


the last components of the initial vectors are nonzero and all other
components are zero; i.e., the initial vectors have the form
\begin{equation}\label{eq9}
x^n|_{t=0} = x^{n,0} = (0,\dots,0,a)^{\rm T}.
\end{equation}
Indefinitely increasing the number of the equations and considering
only the last components of the solutions to each of the Cauchy


\begin{theorem}

$\{x^n_n(t)\}$ converges uniformly on any segment $[0,T]$, $T >
\tau$
\begin{equation}\label{eq10}
x^n_n(t) \to y(t), \quad n \to \infty.
\end{equation}
The limit function
$y(t)$
is the solution to the initial value problem for the delay equation
\begin{equation}\label{eq11}
\left\{%
  \begin{array}{l}
     \displaystyle{\frac{dy(t)}{dt}}=-\theta y(t)
       + g(t-\tau,y(t-\tau)),   \quad t > \tau,   \\
     y(t)=a e^{-\theta t}, \quad  t \in [0, \tau], \\
     y(\tau+0)=a e^{-\theta \tau},
  \end{array}%
\right.
\end{equation}
and the inequality holds
\begin{equation}\label{eq12}
\max\limits_{t\in[0,T]}|x^n_n(t)-y(t)| \le \frac{c}{\sqrt{n}},
\quad n \ge n_0(\theta, \tau),
\end{equation}
where the constant
$c > 0$
does not depend on
$n$.
\end{theorem}


initial conditions.

Let
$n = 2l + 1$
and let the initial vector have the form
\begin{equation}\label{eq13}
x^{n,0} = (x_1^{n,0},\dots,x_n^{n,0})^{\rm T}, \quad x_{l+1}^{n,0} = a, \quad x_j^{n,0} = 0
\ \hbox{ for } \ j \neq l+1.
\end{equation}
Under such initial conditions, the sequence
$\{x^n_n(t)\}$
also converges, but in this case the convergence is not uniform.
The following theorems are valid.

\begin{theorem}

space $L_1(0,T)$, $T > \tau$:
\begin{equation}\label{eq14}
\|x^n_n(t)-y(t), L_1(0,T)\| \to 0, \quad n \to \infty.
\end{equation}
The limit function
$y(t)$
belongs to the Sobolev space
$W^1_1(\tau,T)$
and it is a generalized solution to the initial value problem for the delay equation
\begin{equation}\label{eq15}
\left\{%
\begin{array}{l}
    \displaystyle{\frac{dy(t)}{dt}}=-\theta y(t)
       + g(t-\tau,y(t-\tau)), \quad t > \tau,    \\
    y(t)=0, \qquad  \qquad \quad t \in [0, \tau/2),    \\
    y(t)=a e^{-\theta(t-\tau/2)}, \quad t \in (\tau/2,\tau], \\
    y(\tau+0)=a e^{-\theta\tau/2}.
\end{array}%
\right.
\end{equation}
\end{theorem}

\begin{theorem}
\label{th5}
Let
$n = ml + 1$,
let $s$ be integer,
$0 \le s < m$,

$$
x^{n,0} = (x_1^{n,0},\dots,x_n^{n,0})^{\rm T},
\quad x_{sl+1}^{n,0} = a,
\quad x_j^{n,0} = 0 \mbox{ \ \ for \ \ } j \neq sl+1.
$$
Then,  holds for the sequence $\{x^n_n(t)\}$ for any $T
> \tau$. The limit function $y(t)$ belongs to the space
$W^1_1(\tau,T)$ and it is a generalized solution to the initial
value problem
$$
\left\{%
\begin{array}{l}
    \displaystyle{\frac{dy(t)}{dt}} = -\theta y(t) + g(t-\tau,y(t-\tau)),
        \quad  t > \tau,
     \\
    y(t) = 0, \quad  t \in \left[0, \frac{m-s}{m}\tau\right),
     \\
    y(t) = a e^{-\theta\left(t-\frac{m-s}{m}\tau\right)} ,
       \quad  t \in \left(\frac{m-s}{m}\tau,\tau\right],\\
    y(\tau+0) = a e^{-\theta\frac{s}{m}\tau}.
\end{array}%
\right.
$$
\end{theorem}

\begin{theorem}
\label{th6}
Let
$i \in \mathbb{N}$
be fixed and let the initial vector in have the form
$$
x^{n,0} = (a_1, \dots,a_i, 0, \dots, 0)^{\rm T}.
$$
the sequence
$\{x^n_n(t)\}$
for any
$T > \tau$.
The limit function
$y(t)$
belongs to the space
$W^1_1(\tau,T)$
and it is a generalized solution to the initial value problem
$$
\left\{%
  \begin{array}{l}
    \displaystyle{\frac{dy(t)}{dt}}=-\theta y(t)
       + g(t-\tau, y(t-\tau)), \quad t > \tau, \\
    y(t)=0, \quad  t \in [0, \tau), \\
    y(\tau+0)=a_1+ \dots +a_i.
  \end{array}%
\right.
$$
\end{theorem}


{\bf Remark.} In comparison with Theorems~\ref{th1} and \ref{th2}, Theorems~\ref{th3}--\ref{th6}
establish convergence in Lebesgue spaces, since, as will be seen,
there is no uniform convergence there.
We emphasize that the limit function
$y(t)$
in these theorems is a generalized solution to the delay equati



\section{Proof of limit theorems}

In this section, we prove the limit theorems formulated in the previous section.
To study the convergence of
$\{x^n_n(t)\}$,
we need to know properties of functions used in the following lemma (see \cite{5}).

\begin{lemma}
\label{lm1}
The integral relation is valid for the last component of the solution to the Cauchy

\begin{equation}\label{eq16}
x^n_n(t) = \sum\limits_{k=1}^n x^{n,0}_k \hat{\psi}^n_{n-k+1} (t) +
\int\limits_0^t \hat{\psi}^n_n(t-s) g(s,x^n_n(s)) ds,
\end{equation}
where
$x^{n,0}_k$
is the $k$th component of the initial vector 
\begin{equation}\label{eq17}
\hat{\psi}^n_1(t) = e^{-\theta t}, \quad
\hat{\psi}^n_k(t) = \frac{e^{-\theta t}}{\left(1-\frac{\theta
\tau}{n-1}\right)^{k-1}} \left(1-e^{-\omega_n
t}\sum\limits_{j=0}^{k-2}\frac{(\omega_n t)^j}{j!}\right),
\quad
k =2, \dots,n,
\end{equation}
$$
\omega_n=\frac{n-1}{\tau}-\theta.
$$
\end{lemma}


Proving the limit theorems, we will use asymptotic properties of the sequence
of the functions
$\{\hat{\psi}^n_k(t)\}$
for different
$k$.
The following lemmas are valid.


\begin{lemma}
\label{lm2}
For any
$T > \tau$
and any small
$\varepsilon > 0$,
the uniform convergence holds
$$
\hat{\psi}^n_n(t) \to \left\{%
\displaystyle
\begin{array}{rl}
       0,   &   t \in [0, \tau(1-\varepsilon)], \\
    e^{-\theta(t-\tau)}, & t \in [\tau(1+\varepsilon), T],
\end{array}%
\right.
\quad
n \to \infty.
$$
\end{lemma}


\begin{lemma}
\label{lm3}
Let
$i \in \mathbb{N}$
be fixed. Then, for any
$T > \tau$
and any small
$\varepsilon > 0$,
the uniform convergence holds
$$
\hat{\psi}^n_{n-i}(t) \to \left\{%
\begin{array}{rlr}
    0, & t \in [0, \tau(1-\varepsilon)], &  \\
      & & \quad n\to \infty.\\
    e^{-\theta(t-\tau)}, & t \in [\tau(1+\varepsilon), T],&   \\
\end{array}%
\right.
$$
\end{lemma}


\begin{lemma}
\label{lm4}
Let
$k \in \mathbb{N}$
be fixed. Then, for any
$T > \tau$
and any small
$\varepsilon > 0$,
the uniform convergence holds
$$
\hat{\psi}^{n}_{k}(t) \to e^{-\theta t},
\quad t \in [\varepsilon, T],
\quad n \to \infty.
$$
\end{lemma}


\begin{lemma}
\label{lm5}
Let
$n = ml+1$, $s > 0$,
let $k$
be integer,
$1 \le sl+k \le ml+1$,
and let
$m$, $s$, $k$
be fixed.
Then, for any
$T > \tau$
and any small
$\varepsilon > 0$,
the uniform convergence holds
$$
\hat{\psi}^{ml+1}_{sl+k}(t) \to \left\{%
  \begin{array}{rlr}
    0, &  t \in [0, \frac{s\tau}{m}(1-\varepsilon)],&   \\
      && l \to \infty. \\
    e^{-\theta\left(t-\frac{s\tau}{m}\right)},
             & t \in [\frac{s\tau}{m}(1+\varepsilon), T],&\\
  \end{array}%
\right.
$$
\end{lemma}

Lemma~\ref{lm2} was proved by the author in \cite{1}, proofs of Lemmas~\ref{lm3}--\ref{lm5}
are carried out similarly.

Hereafter we assume that
$T > \tau$
and
$n \ge n_1 > \theta \tau + 1$,
where
$n_1$
is a number such that
$$
\left(1 - \frac{\theta \tau}{k-1}\right)^k \ge \frac{1}{2} e^{-\theta \tau}, \quad
{\mbox{for}} \quad k \ge n_1.
$$
Then, from the definition of the function $\hat{\psi}^n_n(t)$ it
follows that
\begin{equation}\label{eq18}
0 < \hat{\psi}^n_n(t) < 2e^{-\theta(t-\tau)}, \quad t > 0,
\end{equation}
for
$n \ge n_1$.

\begin{proof}[Proof of Theorem~\ref{th3}.]

\begin{equation}\label{eq19}
x^n_n(t) = a e^{-\theta t} + \int\limits_0^t \hat{\psi}^n_n(t-s) g(s, x^n_n(s))ds,
\quad t > 0.
\end{equation}

\begin{equation}\label{eq20}
|x^n_n(t)| \le c(a,T,G,L) < \infty, \quad t \in [0, T], \quad n \ge n_1.
\end{equation}

For solutions to the initial problem e also have the formula
$$
y(t) = a e^{-\theta t} + \int\limits_0^{t-\tau} e^{-\theta (t - s - \tau)} g(s, y(s)) ds,
\quad t > \tau.
$$
Consequently, we get
$$
x^n_n(t) - y(t) = \int\limits_0^{t-\tau} \biggl(\hat{\psi}^n_n(t-s) g(s, x^n_n(s))
- e^{-\theta (t - s - \tau)} g(s, y(s))\biggr)ds
$$
\begin{equation}\label{eq21}
+ \int\limits_{t-\tau}^t \hat{\psi}^n_n(t-s) g(s, x^n_n(s))ds = I^1_n(t) + I^2_n(t),
\quad t > \tau.
\end{equation}

Consider the second summand in Obviously, we have
$$
I^2_n(t) = \int\limits_0^{\tau} \hat{\psi}^n_n(\xi) g(t-\xi, x^n_n(t-\xi))d\xi.
$$
By, this integral can be estimated as follows
$$
|I^2_n(t)| \le G(1 + c(a,T,G,L)) \int\limits_0^{\tau} \hat{\psi}^n_n(\xi) d\xi.
$$
By, from \cite{7} we have
\begin{equation}\label{eq22}
\int\limits_0^{\tau} \hat{\psi}^n_n(\xi) d\xi
\le \frac{2\tau}{\sqrt{n}} \left(1 + e^{\frac{\theta\tau}{\sqrt{n}}}\right), \quad n \ge n_1.
\end{equation}
Hence,
\begin{equation}\label{eq23}
|I^2_n(t)| \le G(1 + c(a,T,G,L))\frac{2\tau}{\sqrt{n}}
\left(1 + e^{\frac{\theta\tau}{\sqrt{n}}}\right), \quad t \in [\tau, T], \quad n \ge n_1.
\end{equation}

Consider the first summand in. Clearly, we have
$$
I^1_n(t) = \int\limits_0^{t-\tau} \left(\hat{\psi}^n_n(t-s) - e^{-\theta(t-s-\tau)}\right)
g(s,x^n_n(s)) ds
%$$
%$$
+ \int\limits_0^{t-\tau} e^{-\theta(t-s-\tau)}
\left(g(s,x^n_n(s)) - g(s,y(s))\right) ds.
$$

$$
|I^1_n(t)|
\le G(1 + c(a,T,G,L)) \int\limits_0^{t-\tau} \left|\hat{\psi}^n_n(t-s) - e^{-\theta(t-s-\tau)}\right| ds
+ L\int\limits_0^{t-\tau} \left|x^n_n(s) - y(s)\right| ds
$$
$$
= G(1 + c(a,T,G,L)) \int\limits_{\tau}^t \left|\hat{\psi}^n_n(\xi) - e^{-\theta(\xi-\tau)}\right| d\xi
+ L\int\limits_0^{t-\tau} \left|x^n_n(s) - y(s)\right| ds.
$$
$$





\section{Funding}
The study was carried out within the framework of the state contract
of the Sobolev Institute of Mathematics (project
no.~FWNF-2022-0008).
\section{Conflict of interest}
The author declares no conflict of interest.
%
% The Bibliography
%

\begin{thebibliography}{99}

\bibitem{1}
\refitem{article} V.~A.~Likhoshvai, S.~I.~Fadeev, G.~V.~Demidenko,
and Yu.~G. Matushkin, \textquotedblleft Modeling multistage
synthesis without branching by a delay equation\textquotedblright \
Sibir. Zh. Industr. Mat. \textbf{7} (1), 73--94 (2004).

\bibitem{2}
\refitem{article}
B.~C.~Goodwin,
\textquotedblleft Oscillatory behavior of enzymatic control processes\textquotedblright \
Adv. Enzyme Reg. \textbf{3},  425--439 (1965).

\bibitem{3}
\refitem{book}
J.~D.~Murray,
\emph{Lectures on Nonlinear-Differential-Equation Models in Biology}
(Clarendon Press, Oxford, 1977).

\bibitem{4}
\refitem{article}
G.~V.~Demidenko, N.~A.~Kolchanov, V.~A.~Likhoshvai, Yu.~G.~Matushkin, and S.~I.~Fadeev,
\textquotedblleft Mathematical modeling of regular contours of gene networks\textquotedblright \
Comput. Math. Math. Phys. \textbf{44} (12), 2166--2183 (2004).

\bibitem{5}
\refitem{article}
G.~V.~Demidenko and I.~A.~Mel'nik,
\textquotedblleft On a method of approximation of solutions to delay differential equations\textquotedblright \
Sib. Math. J. \textbf{51} (3), 419--434 (2010).

\bibitem{6}
\refitem{article}
G.~V.~Demidenko and I.~A.~Uvarova,
\textquotedblleft A class of systems of ordinary differential equations of large dimension\textquotedblright \
J. Appl. Ind. Math. \textbf{10} (2), 179--191 (2016).

\bibitem{7}
\refitem{article}
G.~V.~Demidenko,
\textquotedblleft A method for solving a biological problem of large dimension\textquotedblright \
J. Appl. Ind. Mat. \textbf{16} (4), 621--631 (2022).

\bibitem{8}
\refitem{article}
G.~V.~Demidenko, V.~A.~Likhoshvai, T.~V.~Kotova, and Yu.~E.~Khropova,
\textquotedblleft
On one class of systems of differential equations and on retarded equations\textquotedblright \
Sib. Math. J. \textbf{47} (1), 45--54 (2006).

\bibitem{9}
\refitem{article}
G.~V.~Demidenko,
\textquotedblleft Systems of differential equations of higher dimension and delay equations\textquotedblright \
Sib. Math. J. \textbf{53} (6), 1021--1028 (2012).

\bibitem{10}
\refitem{article}
I.~I.~Matveeva and I.~A.~Mel�nik,
\textquotedblleft On the properties of solutions to a class of nonlinear systems of differential
equations of large dimension\textquotedblright \
Sib. Math. J. \textbf{53} (2), 248--258 (2012).

\bibitem{11}
\refitem{article} V. V. Ivanov, \textquotedblleft Euler--Dirac
integrals and monotone functions in models of cyclic
synthesis\textquotedblright \ Sib. Math. J. \textbf{57} (6),
1011--1028 (2016).

\bibitem{12}
\refitem{book}
Ch.~Blatter,
\emph{Wavelets -- Eine Einf\"uhrung}
(Friedr. Vieweg and Sohn, 2003).

\end{thebibliography}

\end{document}
