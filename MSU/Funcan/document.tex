 \documentclass[14pt]{article}
 
 \usepackage[utf8x]{inputenc}
 \usepackage[english, russian]{babel}
 \usepackage{amssymb, amsfonts}
 \usepackage{amsmath}
 \usepackage{amsthm}
 \usepackage{mathbbol}
 \theoremstyle{plain}
 \newtheorem{Thm}{Тh}
 \newtheorem{Lem}{Лемма}
 \newtheorem{St}{St}
 \newtheorem{Sled}{Следствие}

 \theoremstyle{definition}
 \newtheorem{Def}{Def}
 \newtheorem{Sample}{Пример}

 \author{Васильченко Д.Д., 306}
 \title{Функциональный анализ. Определения и формулировки}
	
 \begin{document}
	\maketitle
	\section{Теория меры}
		\subsection{Кольцо. Минимальное кольцо. Полукольцо. Структура минимального кольца}
			\begin{Def}
				Непустое семейство множеств K из X называется кольцом, если $\forall A, B \in K$ 
				\begin{enumerate}
					\item $A \cap B \in K$
					\item $A \triangle B \in K$
				\end{enumerate}
			\end{Def}
			\begin{St}
				$A, B \in K$ $\Rightarrow$ $A \cup B \in K$ , $A \backslash  B \in K$
			\end{St}
			\begin{Def}
	 			Кольцо называется $\sigma$-кольцом, если оно допускает счетное объединение
			\end{Def}
			\begin{Def}
				Кольцо называется $\delta$-кольцом, если оно допускает счетное пересечние
			\end{Def}
			\begin{Def}
				Если $X \in K$, то кольцо называется алгеброй. X - единица
			\end{Def}
			\begin{Def}
				Кольцо, которое содержится в $\forall$ кольце, содержащем S, называется минимальным кольцом $K(S)$
			\end{Def}
			\begin{St}
				Минимальное кольцо существует.
			\end{St}
			\begin{Def}
				Непустое семейств S множеств из X называется полукольцом, если
				\begin{enumerate}
					\item $\forall A, B \in S \ \ A \cap B \in S$
					\item $\forall A, B \in S \ B \subset A \Rightarrow \ \exists A_1, \dots, A_n \in S : \ A_i \cap A_j = \varnothing, i \neq j \Rightarrow A \backslash B = \coprod\limits_{k=1}^{n} A_k $
				\end{enumerate}
			\end{Def}
			\begin{Lem}
				Пусть S - полукольцо. $A, B_1, \dots, B_n \in S $, $B_i \cap B_j = \varnothing$, $i \neq j$ $\Rightarrow$ $\exists A_1, \dots, A_m \in S: \ A_i \cap A_j = \varnothing, \ i \neq j$ : $A \backslash \left( \coprod\limits_{k=1}^{n} B_k \right) = \coprod\limits_{l=1}^{m} A_l$
			\end{Lem}
			\begin{Thm}
				(О структуре минимального кольца)\newline 
				Пусть S - полукольцо. K(S) - минимальное кольцо, порожденное S $\Rightarrow$ K(S) состоит из всевозможных множеств вида $\coprod\limits_{k=1}^{m} A_K$, $A_i \cap A_j = \varnothing$, $i \neq j$ и $A_j \in S, \ j = \overline{1, m}$
			\end{Thm}
		\subsection{Общее определение меры}
		 S - полукольцо
		 \begin{Def}
			Мерой множества $A \in S$ называется число $\mu(A)$, удовлетворяющее условиям:
			\begin{enumerate}
				\item $\mu(A) \geq 0$
				\item $\mu\left( A_1 \coprod A_2 \right) = \mu(A_1) + \mu(A_2)$
			\end{enumerate}
			Если 2 верно для счётного объединения, то мера называется $\sigma$-аддитивной. 
		 \end{Def}
		 \begin{St}
			S - полукольцо с мерой $\mu$. $A, B \in S$, $A \subset B$ $\Rightarrow$ $\mu(A) \leq \mu(B)$
		 \end{St}
		 \begin{Def}
			 Мера $\mu$, заданная на кольце K, называется непрерывной, если $\forall$ монотонной последовательности множеств $\cup_{n=1}^{\infty} A_n = A \subseteq K$ справедливо $\lim\limits_{n\to \infty} \mu(A_n) = \mu(A)$
	     \end{Def}
		 \begin{Thm}
			Мера непрерывна $\Leftrightarrow$ она $\sigma$-аддитивна.
		\end{Thm}
		\begin{Sled}
			В силу принципа двойственности, если $\mu$, заданная на K, непрерывна, то $\forall \ \{A_n  \}_{n =1}^{\infty}$, $A_{n+1} \subset A_n$, $ A = \cap_{n=1}^{\infty} A_n$ $\Rightarrow \ \lim\limits_{n\to \infty} \mu(A_n) = \mu(A)$
		\end{Sled}
		\subsubsection*{Продолжение меры на минимальное кольцо}
		Пусть S - полукольцо, K - кольцо, $S \subset K$. На S задана мера m, На K задана мера $\mu$.
		\begin{Def}
			Говорят, что мера $\mu$ есть продолжение меры m, если $\forall A \in S \ \mu(A) = m(A)$
		\end{Def}
		\begin{Thm}
			Если K(S) - минимальное кольцо, порожденное S, то $\exists!$ продолжение меры с полукольца на кольцо. Если m - $\sigma$-аддитивна, то и $\mu$ $\sigma$-аддитивна.
		\end{Thm}
		\subsubsection*{Свойтва $\sigma$-аддитивной меры}
		\begin{Thm}
			Пусть K - кольцо с мерой $\mu$, $A, A_1, \dots \in K$ $\Rightarrow$ 
			\begin{enumerate}
				\item если $A_i \cap A_j = \varnothing, \ i \neq j$, тогда $\coprod\limits_{k=1}^{\infty} A_k \subset A \Rightarrow \sum\limits_{k=1}^{\infty} \mu(A_k) \leq \mu(A)$
				\item если мера $\mu$ $\sigma$- аддитивная и $A \subset \cup_{k=1}^{\infty} A_k \Rightarrow \mu(A) \leq \sum\limits_{k=1}^{\infty} \mu(A_k)$
			\end{enumerate}
		\end{Thm}
		\begin{Thm}
			Длина счетно аддитивна на $S  = \{[a, b)  \}$
		\end{Thm}
	\subsection{Мера Лебега}
		 K-алгебра элементарных множеств со счётно аддитивной мерой m. X - единица, $m(X) < \infty$
		 \begin{Def}
			Верхней мерой множества $A \subset X$ назыается
			$$
				\mu^{*}(A) = \inf\limits_{B_1, \dots, B_n, \dots \in K \ A \subset \cup_{k=1}^{\infty} B_k} \sum\limits_{k=1}^{\infty} m(B_k)
			$$
		 \end{Def}
	     \begin{Thm}
			Если $A \in K$ $\Rightarrow$ $\mu^{*}(A) = m(A)$
		\end{Thm}
		 \begin{Def}
			Множество $A \subset X$ называется измеримым по Лебегу, если $\mu^*(A) + \mu^*( x \backslash A) = m(X)$
		\end{Def}
		\begin{Def}
			Нижней мерой множества $A \subset X$ называется $\mu_*(A) = m(X) - \mu^*(X \backslash A)$
		\end{Def}
		\begin{Thm}
			Пусть $A, A_1, A_2, \dots \subset X$, $A \subset \cup_{k=1}^{\infty} A_k$ $\Rightarrow$ $\mu^*(A) \leq \sum\limits_{k=1}^{\infty} \mu^*(A_k)$
		\end{Thm}
		\begin{Sled}
			$\mu_*(A) \leq \mu^*(A)$
		\end{Sled}
		\begin{Sled}
			$\mu_*(A) \geq 0$
		\end{Sled}
		\begin{Sled}
			$\forall A,B \subset X$ $\Rightarrow$ $|\mu^*(A) - \mu^*(B)| \leq \mu^*(A \triangle B)$
		\end{Sled}
		\begin{Lem}
			Если $\mu^{*}(A) = 0 \ \Rightarrow$ A измеримо и $\mu(A) = 0$ 
		\end{Lem}
		\begin{Def}
			Мера называется полной, если $\forall$ подмножество множества меры ноль измеримо и имеет меру нуль.
		\end{Def}
		\begin{Thm}
			Множество $A \subset X$ измеримо $\Leftrightarrow$ $\forall \varepsilon > 0 \exists B \in K \mu^*(A \triangle B) < \varepsilon$
		\end{Thm}
		\begin{Sled}
			A - измеримо, если $\forall \varepsilon > 0$ $\exists С$ измеримое: $\mu^*(A \triangle C) < \varepsilon$
		\end{Sled}
		\begin{Thm}
			$\mu^*$ - мера на измеримых множествах
		\end{Thm}
		\begin{Sled}
			$\mu^{*}$ - счётно аддитивная мера на измеримых множествах
		\end{Sled}
		\begin{Thm}
			Счётное объединение измеримых множеств - измеримо
		\end{Thm}
		\begin{Sled}
			Счётное пересечение измеримых множеств - измеримо
		\end{Sled}
		Случай $m(x) = \infty$ \newline
		\begin{Def}
			Мера $\mu$ называется $\sigma$-конечной, если $\exists X_1, X_2, \dots \in K: \ \mu(X_i) < \infty, \ i = 1, 2, \dots $ $\Rightarrow$ $X = \coprod\limits_{i=1}^{\infty} X_i$
		\end{Def}
		\begin{Def}
			Множество A называется измеримым, если измеримы все $A \cap X_i$, $i = 1, 2, \dots$ при этом мерой A называется $\mu(A) = \sum\limits_{i=1}^{\infty} \mu(A \cap X_i)$
		\end{Def}
	\subsection{Измеримые множества на R}
		$X = R$, обычная мера Лебега на прямой.\newline
		\begin{St}
			$\forall открытое множество на прямой измеримо$
		\end{St}
		\begin{Thm}
			Всякое открытое множество на прямой представимо ввиде не более, чем счётного объединения попарно непересекающихся интервалов.
		\end{Thm}
	\subsection{Канторово множество}
		Канторово множество имеет меру нуль.
	\subsection{Борелевские множества}
		\begin{Def}
			Борелевскими называются множества, получающиеся в результате счётного объединения или пересечения открытых множеств
		\end{Def}
		\begin{St}
			Борелевская мера не полна. 
		\end{St}
		\begin{St}
			$\forall$ измеримое множество можно заключить в борелевском множестве той же меры. 
		\end{St}
	\subsection{Многомерный случай $R^n$}
		\begin{Thm}
			Всякое открытое множество в $R^m$ измеримо по Лебегу.
		\end{Thm}
	\subsection{Мера Жордана}
		$m(X) < \infty$
		\begin{Def}
			Верхняя мера Жордана: $\mu_J^*(A) = \inf\limits_{A \subset \cap_{i=1}^n B_i} \sum\limits_{i=1}^n m(B_i)$
		\end{Def}
		\begin{Def}
			Нижняя мера Жордана: $\mu_{*J} = m(X) - \mu_J^*( x \ A)$
		\end{Def}
	\subsection{Мера Лебега-Стилтьеса}
		$F(t)$ - неубывающая функция, $t \in R$. $X = R$, $S = \{[a,b) \}$ $\Rightarrow$ $m_F([a,b)) = F(b) - F(a)$.
		\begin{Thm}
			$m_F$ - $\sigma$-аддитивная $\Leftrightarrow$ $F(t)$ непрервыная слева. 
		\end{Thm}
	\subsection{Сравнение мер. Абсолютная непрерывность}
		\begin{Def}
			Пусть $\sigma$ - аддитивные меры $\mu$ и $v$ заданы на $\sigma$ - алгебры $\sum$. Мера $v$ называется абсолютно непрерыной относительно меры $\mu$, если из $\mu(A) = 0$ $\Rightarrow$ $v(A) = 0, \ \forall A \in \sum$.
		\end{Def}
		\begin{Thm}
			Пусть $\sigma$-аддитивные меры $\mu, v$ заданы на $\sigma$-алгебре $\sum$  $\Rightarrow$ мера v абсолютно непреревына относительна $\mu$  $\Leftrightarrow$ $\forall \varepsilon > 0 \ \exists \delta > 0 \ \mu(A) < \delta \  \Rightarrow v(A) < \varepsilon \ \forall A \in \sum$

		\end{Thm}
	\subsection{Канторова лестница}
		\begin{Def}
			Функция $f(t)$ называется абсолютно непрерывной, если $\forall \varepsilon > 0 \ \exists \delta > 0 \ \forall (a_k, b_k), k = \overline{1, N} , \ (a_k, b_k) \cap (a_j, b_j) = \empty, k \neq j, \ \sum\limits_{k=1}^{N} (b_k- a_k) < \delta \Rightarrow \sum\limits_{k=1}^{N} | F(b_k) - F(a_k)| < \varepsilon$
		\end{Def}

		\begin{Thm}
			Пусть мера $\mu$ порожденная длиной, и мера $v$ - функцией $f$ $\Rightarrow$ $v$ абсолютно непрерывна относительно $\mu$ $\Leftrightarrow$ $f(t)$ - абослютно непрерывна.
		\end{Thm}
		Канторова лестинца: \newline
		Хз как здесь написать, картинки нужны)
	\subsection{Взаимно сингулярные меры}
		\begin{Def}
			Две $\sigma$-аддитивные меры $\mu$ и $v$ заданные на общей $\sigma$-алгебре $\sum$, называются взаимно сингулярными, если $\exists A \in \sum: \ \mu(A) = v(X \backslash A) = 0$
		\end{Def}
\section{Измеримые функции}
	Триплет: $\{X, \Sigma, \mu\}$. 
	\begin{Def}
		Функция $f(x)$, $x \in X$ называется измеримой, если $\forall c \in R$ множество $\{x \in X: \ f(x) < c \}$ измеримо. (далее сокращенно $\{f < c \}$)
	\end{Def}	
	\begin{Thm}
		Следующие 4 утверждения эквиваленты:
		\begin{enumerate}
			\item $\forall c \in R$ измеримо $\{ f < c\}$ 
			\item $\forall c \in R$ измеримо $\{ f \leq c\}$ 
			\item $\forall c \in R$ измеримо $\{ f > c\}$
			\item $\forall c \in R$ измеримо $\{ f \geq c\}$  
		\end{enumerate}
	\end{Thm}
	\begin{Lem}
		$f$ - измерима $\Rightarrow$ $f + c$ - измерима, $c = const$
	\end{Lem}
	\begin{Lem}
		$f$ - измерима $\Rightarrow$ $c*f$ - измерима, $c = const$
	\end{Lem}
	\begin{Lem}
		$f, \ g$ - измеримы $\Rightarrow$ $\{ f < g\}$ - измерима 
	\end{Lem}
	\begin{Thm}
		$f, \ g$ - измеримы $\Rightarrow$ $f \pm g$ - измерима		
	\end{Thm}
	\begin{Thm}
		Линейная комибнация измеримых функций измерима		
	\end{Thm}
	\begin{Lem}
		$f$ - измерима $\Rightarrow$ $|f|$ - измерима
	\end{Lem}
	\begin{Lem}
		$f$ - измерима $\Rightarrow$ $f^2$ - измерима
	\end{Lem}
	\begin{Thm}
		$f, \ g$ - измеримы $\Rightarrow$ $f*g$ - измерима
	\end{Thm}
	\begin{Thm}
		$f, \ g$ - измеримы и $g \neq 0$ $\Rightarrow$ $\dfrac{f}{g}$ - измерима
	\end{Thm}
	\begin{Def}
		Если какое-либо свойство выполнено во всех точках, за исключением точек множества меры ноль, то свойство выполнено почти всюду
	\end{Def}
	\begin{Def}
		Функции $f$ и $g$ называются эквивалентными, если $f = g$ почти всюду
	\end{Def}
	\begin{Thm}
		Если $g$ -измерима и $f \sim g$, то $f$ - измерима
	\end{Thm}
	\begin{Thm}
		Если $f_k(x)$ - измеримы при $k \in N$ и $f_k(x) \to f(x)$, то $f(x)$ - измерима
	\end{Thm}
	\begin{Thm}
		Пусть $f_k(x)$ - измеримы при $k \in N$, тогда $\max{ \left( f_1(x), \dots , f_n(x) \right)}, \min{\left( f_1(x), \dots, f_n(x) \right)}, \overline{f}(x)  = \sup\limits_{k \geq 1} f_k(x), \underline{f}(x) = \inf\limits_{k \geq 1} f_k(x)$ - измеримы
	\end{Thm}
	\begin{Thm}
		Пусть $f_k(x), \ k \in N$  измеримы $\Rightarrow$ $\overline{\lim} f_k$ и $\underline{\lim} f_k$ при условии их конечности почти всюду измеримы
	\end{Thm}
	\begin{Thm}
		Если $f(x)$ - измерима и дифференцируема, то $f'(x)$ - измерима	
	\end{Thm}
	\subsection{Сходимость по мере}
		Пусть $\mu(X) < \infty$.
		\begin{Def}
			Говорят, что последовательность измеримых функций $f_k(x), \ x\in X, \ k \in N$ сходится по мере к измеримой функции $f(x)$, если $\forall \varepsilon > 0 \ \lim\limits_{n \to \infty} \mu \left( \{|f_n - f| \geq \varepsilon\} \right)	=0$. ($f_n \xrightarrow{\mu} f$)
		\end{Def}
		\begin{Thm}
			Если $f$ и $g$ - пределы $f_k$ по мере, то $f \sim g$
		\end{Thm}
		\begin{Thm}
			(Рисса)\newline
				Из последовательности измеримых функций, сходящейся по мере к измеримой функции можно выбрать подпоследовательность, которая сходится к этой функции почти всюду. 
		\end{Thm}
		\begin{Thm}
			(Егорова) \newline
			Пусть $f_n \to f$ почти всюду на $X$, все $f_n$ измеримы и $\mu(X) < \infty$ $\Rightarrow$ $\forall \delta > 0 \ \exists X_\delta  \subset X \ \mu(X \backslash X_\delta) < \delta$ и $f_n \rightrightarrows f$ на $X_\delta$
		\end{Thm}
	\section{Интеграл Лебега}
		\subsection{Ограниченные функции}
		\begin{Def}
			Функция называется простой, если она измерима и принимает конечное число значений. $f(x) = \sum\limits_{k=1}^{n} f_k \mathbb{1}_{A_k}(x)$, где $X = \coprod\limits_{k=1}^{n} A_k, \ A_k \in \Sigma$.
		\end{Def}
		\begin{Def}
			Интеграл Лебега от простой функции: $(L)\int\limits_{X} f d\mu = \sum\limits_{k=1}^n f_k \mu(A_k)$
		\end{Def}
		Свойства такого интеграла:
		\begin{enumerate}
			\item $\int\limits_{X}C* f d\mu = C* \int\limits_{X} f d\mu$
			\item $\int\limits_{X} (f \pm g) d\mu = \int\limits_{X} f d\mu \pm \int\limits_{X} g d\mu$
			\item Линейность
			\item $| \int\limits_{X} fd\mu| \leq \max{|f_1|, \dots, |f_n|} \mu (X)$
		\end{enumerate}
		\begin{Lem}
			Пусть $f_n(x)$ - последовательность простых функций и $f_n \rightrightarrows f$ на $X$, тогда $\int\limits_{X} f_n d\mu $ - сходится как числовая последовательность.
		\end{Lem}
		\begin{Def}
			Пусть $f(x)$ - равномерный предел простых функий $f_n(x)$, тогда интегралом Лебега от функции $f(x)$ называется $(L) \int\limits_X f d\mu = \lim\limits_{n \to \infty} \int\limits_{X} f_n d\mu$
		\end{Def}
		\begin{Lem}
			Для $\forall$ ограниченой измеримой функции f $\exists$ существует последовательность $f_n$ простых функций, такая что $f_n \rightrightarrows f$. 
		\end{Lem}
		\begin{Thm}
			$\forall$ измеримая ограниченая функция $f(x)$ интегрируема по Лебегу, причём $\int\limits_{X} f d\mu$ может быть найден, как предел $\int\limits_{X} f d\mu = \lim\limits_{n \to \infty} \sum\limits_{k=1}^{n} \dfrac{k}{n} \mu(A_{k,n}) = \lim\limits_{n \to \infty} \sum\limits_{k=1}^{N} \dfrac{k}{n} \mu(\{\dfrac{k}{n} \leq f < \dfrac{k+1}{n}\})$
		\end{Thm}
		\subsection{Неограниченные функции}
			\begin{Def}
				Функция $f(x)$ называется простой со счётным числом значений, если $f(x) = \sum\limits_{k=1}^{\infty} f_k \mathbb{1}_{A_k}(x)$, где $\coprod\limits_{k=1}^{\infty} A_k = X, \ A_k \in \Sigma$.
			\end{Def}
			\begin{Def}
				Интегралом от простой функции со счётным числом значение при условии его существования ($\sum\limits_{k=1}^{\infty} |f_k| \mu(A_k)$ - сходящийся) называется $(L)\int\limits_x f d\mu = \sum\limits_{k=1}^{\infty} f_k \mu(A_k) $
			\end{Def}
			Свойства такого интеграла:
			\begin{enumerate}
				\item $\int\limits_{X}C* f d\mu = C* \int\limits_{X} f d\mu$
				\item $\int\limits_{X} (f \pm g) d\mu = \int\limits_{X} f d\mu \pm \int\limits_{X} g d\mu$
				\item Линейность
				\item $| \int\limits_{X} fd\mu| \leq \sup\limits_{X}{|f_k|} \mu (X)$
				\item Если $|f| \leq g$ и g - интегрируема, то $f$ - интегрируема и $|\int\limits_X f d\mu | \leq \int\limits_X g d\mu$
			\end{enumerate}
			\begin{Lem}
				Пусть $f_n$ - последовательность интегрируемых простых функций со счётным числом значений, равномерно сходящаяся к $f$ ($f_n \rightrightarrows f$), тогда $\int\limits_X f_n d\mu$ сходится, как числовая последовательность. 
			\end{Lem}
			\begin{Def}
				Функция $f$ называется интегрируемой по Лебегу, если существует последовательность $f_n$ - простых функций со счётным числом значений, равномерно сходящаяся к f ($f_n \rightrightarrows f$ на X). В этом случае интеграл Лебега от f: $(L) \int\limits_X f d\mu = \lim\limits_{n \to \infty} \int\limits_X f_n d\mu $. 
			\end{Def}
			\begin{Lem}
				Пусть $f_n$, $\overline{f}_n$ - последовательности простых функций со счётным числом значений и $f_n \rightrightarrows f$, $\overline{f}_n \rightrightarrows f$, тогда $\lim\limits_{n \to \infty} \int\limits_X f_n d\mu = \lim\limits_{n \to \infty} \int\limits_X \overline{f}_n d\mu$
			\end{Lem}
			\begin{Lem}
				Пусть $f(x)$ - инетгрируема на X и $f_n$ - последовательность простых функций со счётным числом значений, такая что $f_n \rightrightarrows f$  на X, тогда начиная с некоторого номера $N$ функции $f_n$ интегрируемы. 
			\end{Lem}
			\underline{\bf{Свойства интеграла Лебега}}
			\begin{enumerate}
				\item $\int\limits_{X}C* f d\mu = C* \int\limits_{X} f d\mu$
				\item $\int\limits_{X} (f \pm g) d\mu = \int\limits_{X} f d\mu \pm \int\limits_{X} g d\mu$
				\item Если $f(x) \geq 0$ п.в. $\Rightarrow$ $\int\limits_X f d\mu \geq 0$
				\item Если $f(x) \leq g(x)$ п.в. $\Rightarrow$ $\int\limits_X f d\mu \leq \int\limits_X g d\mu$
				\item Если $f(x)$ интегрируема, то и $|f(x)|$ - интегрируема, обратное неверно.
				\item Если $f(x)$ - измерима, а $g(x)$ - интегрируема и $|f(x)| \leq g(x)$, то $f(x)$ - интегрируема и $|\int\limits_X f d\mu| \leq \int\limits_X g d\mu$
				\item Если $f(x)$ - интегрируема, а $g(x)$ - измерима и ораничена, то $f*g$ - интегрируема
				\item (Аддитивность по множеству интегрирования) Пусть $X = A \coprod B$, $f$ - интегрируема $\Rightarrow$ $\int\limits_X f d\mu = \int\limits_A f d\mu + \int\limits_B f d\mu$
				\item Если $f$ - измерима и $\mu(A) = 0$ $\Rightarrow$ $\int\limits_A f d\mu = 0$.
				\item Если $f = 0$ п.в, то $\int\limits_X f d\mu = 0$
				\item Если $f$- интегрируема и $f \geq 0$ , но  $\int\limits_X f d\mu = 0$ $\Rightarrow$ $ f = 0$ п.в. 
			\end{enumerate}
			\begin{Thm}
				(Абсолютная непрерывность интеграла Лебега) \newline
				Пусть  $f$ - интегрируема на X $\Rightarrow$ $\forall \varepsilon > 0 \  \exists \delta > 0:  \forall A \in \Sigma: \mu(A) < \delta \Rightarrow |\int\limits_A f d\mu| \leq \varepsilon$	
			\end{Thm}
			\begin{Thm}
				(Счётная аддитивность интеграла Лебега) \newline
				Пусть $f(x)$ - интегрируема на X и $X = \coprod\limits_{k=1}^{\infty} A_k$, $A_k \in \Sigma$ $\Rightarrow$ $\int\limits_X f d\mu = \sum\limits_{k=1}^{\infty} \int\limits_{A_k} f d\mu$. 
			\end{Thm}
			\begin{Thm}
				Пусть дана $f(x)$, $x \in X$ и $X = \coprod\limits_{k=1}^{\infty} A_k$, все $A_k$ - измеримы, $f(x)$ интегрируема на $A_k$ и сходится ряд $\sum\limits_{k=1}^{\infty} \int\limits_{A_k} |f| d\mu$. Тогда $f(x)$ интегрируема на X.
			\end{Thm}
		\subsection{Предельные переходы}
			\begin{Thm}
				Пусть $f_n$ - интегрируемые на X функции и $f_n \rightrightarrows f$, тогда $f$ - интегрируема на X и $\int\limits_X f d\mu = \lim\limits_{n \to \infty} \int\limits_X f_n d\mu$. 
			\end{Thm}
			\begin{Thm}
				(Лебега)\newline
				Пусть $f_n$ последовательность измеримых функций и $f_n \xrightarrow{\mu} f$ и $\exists F$ - интегрируемая, такая что $|f_n| \leq F$ п.в. $\Rightarrow$ $f, f_n$ - интегрируемы на X и $\int\limits_X f d\mu = \lim\limits_{n \to \infty} \int\limits_X f_n d\mu$
			\end{Thm}
			\begin{Thm}
				(Леви)\newline
				Пусть $f_n$ - последовательность интегруемых на X функций и $f_n \leq f_{n+1}$ п.в. и $\exists c: \ \int\limits_X f_n d\mu \leq C$ $\Rightarrow$ $f(x) = \lim\limits_{n\to\infty} f_n(x)$  (конечный или бесконечный) и f - интегрируема на X, $\int\limits_X f d\mu \leq C$ и возможен переход к пределу ($\int\limits_X f d\mu = \lim\limits_{n \to \infty} \int\limits_X f_n d\mu$).
			\end{Thm}
			\begin{Sled}
				Пусть $f_n$ - последовательность неотрицательных интегрируемых на X функций и $\sum\limits_{n=1}^{\infty} \int\limits_X f_n d\mu < + \infty$ $\Rightarrow$ $F(x) = \sum\limits_{k=1}^{+\infty} f_n(x)$ интегрируема на X и $\int\limits_X F d\mu = \sum\limits_{n=1}^{\infty} \int\limits_X f_n d\mu$.
			\end{Sled}
			\begin{Thm}
				(Фату)
				$f_n(x) \geq 0$ - интегрируемы на X и $\exists с:\ \int\limits_X f_n d\mu \leq c$ $\Rightarrow$ $f(x) = \underline{\lim\limits_{n\to\infty} }f_n(x)$  - интегрируема на X и $\int\limits_X f d\mu \leq c$. 
			\end{Thm}
	\section{Сравнение интеграла Лебега и Римана}
		\begin{Thm}
			Пусть $f(x)$ интегрируема по Риману на X $\Rightarrow$ $f(x)$ интегрируема по Лебегу на X
		\end{Thm}
		\begin{Thm}
			(критерий интегрируемости по Риману)\newline
			$f(x)$ интегрируема по Риману на X $\rightrightarrows$ $f(x)$ почти всюду непрерывна на X.
		\end{Thm}
	\section{Пространство суммируемых функций L1}
		$X, \mu$ (полная)
		\begin{Def}
			$L_1(X, \mu)$ -пространство функций, для которых существует и конечен интеграл $\int\limits_X |f| d\mu$. $\|f\|_{L_1} = \int\limits_x |f|d\mu$. 
		\end{Def}
		\begin{Thm}
			Пространство $L_1$ - полное. 		
		\end{Thm}
		\begin{Lem}
			Пусть $f(x) \in L_1(X, \mu)$, тогда $\exists$ $\{f_n(x)\}$ - простые функции со счётным числом значений: $\| f_n - f \|_{L_1} \to 0$
		\end{Lem}
		\begin{Lem}
			Пусть $f(x) \in L_1(X, \mu)$, тогда $\exists$ $\{f_n(x)\}$ - простые функции с конечным числом значений: $\| f_n - f \|_{L_1} \to 0$
		\end{Lem}
		\begin{Thm}
			$d\mu = dx$ $f\in L_1(0,1)$ $\Rightarrow$ $\exists \{\phi_n\}_{n=1}^{\infty}, \ \phi_n \in C(0,1): \ \|f - \phi_n\|_{L_1} \to 0$
		\end{Thm}
		\begin{Thm}
			(О непрерывности в интегральной метричке)\newline
			$d\mu = dx$, $f \in L_1(0,1)$ $\Rightarrow$ $\forall \varepsilon >  0 \ \exists \delta > 0 \ \forall \Delta: |\Delta| < \delta \Rightarrow \int\limits_{(0,1)} |f(x+\Delta) - f(x)| dx < \varepsilon$. Вне (0,1) f доопределяется 0. 
		\end{Thm} 
	\section{Пространство Lp}
		\begin{Def}
			Пространство $L_p(X, \mu)$ - пространство функций, для которых существует и конечен интеграл $\int\limits_X |f|^p d\mu$. Норма $\|f\|_{L_p} = \left(\int\limits_X |f|^p d\mu \right)^{1/p}$
		\end{Def}
		\begin{Thm}
			(Неравенство Юнга)\newline
			$\dfrac{1}{p} + \dfrac{1}{q} = 1$, $a, b > 0$ $\Rightarrow$ $ab \leq \dfrac{a^p}{p} + \dfrac{b^q}{q}$. 
		\end{Thm}
		\begin{Thm}
			(Неравенство Гёльдера) \newline
			$\dfrac{1}{p} + \dfrac{1}{q} = 1$, $f\in L_p(X, \mu), g \in L_q(X, \mu)$, $fg \in L_1(X, \mu)$ $\Rightarrow$ $\|fg\|_{L_1} \leq \|f\|_{L_p} \|g\|_{L_q}$
		\end{Thm}
		\begin{Thm}
			(Неравенство Минковского)\newline
			$f, g\in L_p(X, \mu)$ $\Rightarrow$ $\|f+g\|_{L_p} \leq \|f\|_{L_p} + \|g\|_{L_p}$ 
		\end{Thm}
		\begin{Thm}
			Пространство $L_p(X, \mu)$ - полное. 
		\end{Thm}
			\begin{Lem}
			Пусть $f(x) \in L_p(X, \mu)$, тогда $\exists$ $\{f_n(x)\}$ - простые функции со счётным числом значений: $\| f_n - f \|_{L_1} \to 0$
		\end{Lem}
		\begin{Lem}
			Пусть $f(x) \in L_p(X, \mu)$, тогда $\exists$ $\{f_n(x)\}$ - простые функции с конечным числом значений: $\| f_n - f \|_{L_1} \to 0$
		\end{Lem}
		\begin{Thm}
			Если $f \in L_p(D, dx)$, $D$ - ограниченое замкнутое множество $\Rightarrow$ $\exists$ $\{\phi_n\}_{n=1}^{\infty}: \phi_n \in C(D): \ \|\phi_n - f\|_{L_p} \to 0$
		\end{Thm}
		\begin{Thm}
			$d\mu = dx$, $f\in L_p(X, \mu)$ $\Rightarrow$ $\forall \varepsilon > 0 \exists \delta > 0 \forall \Delta: \ |\Delta| < \delta \Rightarrow \|f(x+\Delta) - f(x)\|_{L_p} < \varepsilon$. Вне $D$ доопределяем f нулем. 
		\end{Thm}
	\section{Заряды}
		X, $\Sigma$ - сигма-алгебра
		\begin{Def}
			Отображение $\Phi(A): \ \Sigma \to \mathbb{R}$ называется зарядом, если $A = \coprod\limits_{k=1}^{\infty} A_k, \ A_k \in \Sigma$, $\Phi(A) = \sum\limits_{k=1}^{\infty} \Phi(A_k)$, ряд должен абсолютно сходиться. 
		\end{Def}
		\begin{Def}
			Множество $A \in \Sigma$ называется положительным относительно заряда $\Phi$, если $\forall B \subset A, \ B \in \Sigma \Rightarrow \Phi(B) \geq 0$. 
		\end{Def}
		\begin{Lem}
			Пусть $A \in \Sigma$, тогда $\sup\limits_{B \subset A, B\in \Sigma}|\Phi(B)| < \infty$
		\end{Lem}
		\begin{Thm}
			(Жордана)\newline
			Пусть заряд $\Phi(A)$ задан на множестве $X$, тогда $\exists X^{-}, X^{+}: \ X^{-} \cap X^{+} = \emptyset$, $X = X^{+} \coprod X^{-}$, $X^{-}$ - отрицательно относительно $\Phi$, $X^{+}$ - положительно относительно $\Phi$.  
		\end{Thm}
		\begin{Sled}
			Заряд $\Phi$ представляется в виде разности двух мер $\nu^{\pm}$: $\Phi = \nu^{+} - \nu^{-}$, $\forall A \in \Sigma$.  
		\end{Sled}
		\begin{Thm}
			(Радона-Никодима)\newline
			Пусть $\nu, \mu$ - $\sigma$ -аддитивные меры, заданные на $\Sigma$,  $\nu$ абсолютно непрерывна относительно $\mu$, тогда существует $f$ такая, что  $\nu(A) = \int\limits_A f d\mu$. 
		\end{Thm}
		\begin{Lem}
			Условия предыдущей теоремы + $\nu(X) > 0$ $\Rightarrow$ $\exists A \in \Sigma, \delta > 0: \ \nu(A) > 0, \ \forall B \subset A, \ B \in \Sigma \Rightarrow \nu(B) \geq \delta \mu(B)$. 
		\end{Lem}
		\begin{Def}
			Функция $f$ из теоремы Радона-Никодима называется производной Радона-Никодима ($f = \dfrac{d\nu}{d\mu})$.
		\end{Def}
		\begin{Thm}
			(о замене переменной) \newline
			Пусть $\nu$ абсолютно непрерывна относительно $\mu$, $\rho = \dfrac{d\nu}{d\mu}$ - производная Радона-никодима и $f\rho$ - интегрируема $\Rightarrow$ $f$ - интегрируема и $\int\limits_X f d\nu = \int\limits_X f\rho d\mu$
		\end{Thm}
	\section{Теорема Фубини}
		Пусть $X \subset \mathbb{R}^n, Y \subset \mathbb{R}^m$, $Z = X \times Y$, Пусть $D = A \times B, \mu_z(D) = \mu_x(A) * \mu_y(B)$ и справедливо $\mu_z(A) = \int\limits_A \mu_y(D_x) d\mu_x$, где $D_x = \{y: \ (x,y) \in D \}$
		\begin{Lem}
			Пусть $P_x, P_y$ - полукольца, тогда $P_z = P_x \times P_y$ - полукольцо.
		\end{Lem}
		\begin{Thm}
			Пусть $\mu_x, \mu_y$ - $\sigma$ -аддитивные меры на $P_x, P_y$, тогда $\mu_z$ $\sigma$-аддитивная на $P_z$.
		\end{Thm}
		\begin{Lem}
			Пусть $\mu$ - мера Лебега, C измеримо относительно $\mu$, тогда $\exists D: \ C \subset D, \ \mu(C) = \mu(D), \ D = \cap_{k=1}^{\infty} C_k, \ C_{k+1} \subset C_k, \ C_k = \cup_{n=1}^{\infty} B_{n,k}, \ B_{n, k} \subset B_{n+1,k}, B_{n,k}$ - элементарные. 
		\end{Lem}
		\begin{Thm}
			Пусть C измеримо относительно $\mu_z = \mu_x \otimes \mu_y$, тогда $C_x$ измеримо относительно $\mu_y$ и $\mu_y(C_x)$ интегируема по X и $\mu_z(C) = \int\limits_X \mu_y(C_x) d\mu_z$
		\end{Thm}
		\begin{Thm}
			(Фубини)\newline
			\begin{enumerate}
				\item Пусть $f(x,y)$ интегрируема по $\mu_z$ на Z, тогда  $f(x,y)$ п.в. интегрируема по $Z_x$, а функция $I(x) = \int\limits_Y f(x,Y) d\mu_y$ интегрируема по X и $\int\limits_Z f(x,y) d\mu_z = \int\limits_X I(x) d\mu_x$
				\item Пусть $f(x) \geq 0$ и существует повторный интеграл, тогда существует и двойной и они равны. 
			\end{enumerate}			
		\end{Thm}
	\section{Метрические пространства}
		\begin{Def}
			Пространство M называется метрическим, если для $\forall x,y \in M$ задано отображение $\rho: M\times M \to \mathbb{R}$, обладающее следующими свойствами:
			\begin{enumerate}
				\item Неотрицательность $\rho(x,y) \geq 0, \rho(x,y) = 0 \rightrightarrows x = y$
				\item Симметричность $\rho(x,y) = \rho(y,x)$
				\item Неравенство треугольника $\rho(x,y) \leq \rho(x,z) + \rho(z,y)$
			\end{enumerate}
		\end{Def}
			\begin{Def}
				Открытый шар с центром в точке $x_0$, радиусом $r$ из $M$: $B(x_0, r) = \{x\in M: \ \rho(x,x_0) < r\}$
			\end{Def}
			\begin{Def}
				Замкнутый шар с центром в точке $x_0$, радиусом $r$ из $M$: $\overline{B}(x_0, r) = \{x\in M: \ \rho(x,x_0) \leq r\}$
			\end{Def}
			\begin{Def}
				Множество называется открытым, если $\forall$ точка этого множество входит в это множество с некоторой окрестностью (открытый шар)
			\end{Def}
			\begin{Def}
				Точка $x_0$ называется предельной для множества $M$, если $\forall r > 0 \Rightarrow (B(x_0, r) \cap M) \backslash \{x_0\} \neq \emptyset$
			\end{Def}
			\begin{Def}
				Замыкание множества - присоединение к множеству его предельных точке			
			\end{Def}
			\begin{Def}
				Множество называется замкнутым, если совпадает со своим замыканием
			\end{Def}
			\begin{St}
				Пусть $\rho$ - метрика $\Rightarrow$ $\dfrac{\rho}{1 + \rho}$ - тоже метрика.
			\end{St}
			\begin{St}
				Пусть $G \subset M$ - открыто, $F \subset M$ - замкнуто, тогда $M\backslash G$ -замкнуто, $M \backslash F$ - открыто. 
			\end{St}
			\begin{Thm}
				\begin{enumerate}
					\item Объединение произвольного числа открытых множество открыто
					\item Пересечение конечного числа открытых множеств открыто
					\item Пересечение произвольного числа замкнутых множеств замкнуто
					\item Объединение конечного числа замкнутых множество замкнуто
				\end{enumerate}	
			\end{Thm}
			\subsection{Последовательности}
				\begin{Def}
					Последовательность $\{x_n\}_{n=1}^{\infty}$ из $M$ сходится к $x_0 \in M$, если $\lim\limits_{n\to \infty} \rho(x_n, x_0) = 0$
				\end{Def}
				\begin{Def}
					Метрическое пространство M называется полным, если в нем $\forall$ фундаментальная последовательность сходится.			
				\end{Def}
				\begin{Def}
					Отображение $f: X \to Y$ называется непрерывным в точке $x_0 \in M$, если $\forall \{x_n\}_{n=1}^{\infty}$ из X, такой что $x_n \to x_0$ следует, что $f(x_n) \to f(x)$. 
				\end{Def}
				\subsection{Сжимающие отображения}
				\begin{Def}
					Отображение $f: X \to X$ называется сжимающим, если $\exists \alpha \in [0,1)$: $\rho(f(x), f(y)) \leq \alpha \rho(x,y), \forall x,y \in M$. 
				\end{Def}
				\begin{St}
					Сжимающее отображение непрерывно	
				\end{St}
				\begin{Thm}
					(принцип сжимающих отображений) \newline
					Пусть М - полное пространство и $f: M \to M$  - сжимающее, тогда $\exists! x_0 \in M: \ f(x_0) = x_0$ - неподвижная точка.
				\end{Thm}
				\begin{St}
					Скорость сходимости: $\rho(x_n, x) \leq \dfrac{\alpha^n}{1-\alpha} \rho(x_0, x_1)$	
				\end{St}
				\begin{Thm}
					Пусть M - полное метрическое пространство $f^m, \ m\in \mathbb{N}$ - сжимающее отображение в M $\Rightarrow$ для $f$ $\exists!$ неподвижная точка. 
				\end{Thm}
				\begin{Thm}
					Пусть M - полное метрическое пространство, $\overline{B}(x_0, r) \subset M$ и $f: \overline{B}(x_0, r) \to M$, f - сжимающее на шаре. Тогда если $\rho(f(x_0), x_0) \leq (1- \alpha) r$, то $\exists! x' \in \overline{B}(x_0, r)$ - неподвижная точка. 
				\end{Thm}
				\begin{Def}
					Метрическое пространство M называется компактным, если из $\forall$ последовательность его элементов можно выделить сходяющуюся подпоследовательность.	
				\end{Def}
				\begin{Thm}
					Пусть M - метрическое, полное, компактное и $f: M \to M$, $\rho(f(x), f(y)) < \rho(x,y), \forall x,y \in M, x \neq y$, тогда $\exists! x' \in M: \ f(x') = x'$. 
				\end{Thm}
			\subsection{Теорема Хаусдорфа о пополнении метрического пространства}
				\begin{Def}
					Два метрических пространства называется измотреческими, если между ними существует биекция, сохраняющая расстояние между точками ($M \sim M'$)	
				\end{Def}
				\begin{Thm}
					Пусть M - метрическое пространство, тогда существует и единственно полное метрическое пространство $\widetilde{M}$, такое что $M_0 \sim M$, $M_0 \subset \widetilde{M}$, $\overline{M_0} = \widetilde{M}$
				\end{Thm}
			\subsection{Теорема Бэра о категориях}
				\begin{Def}
					Множество A назывется всюду плотным в M, если $\overline{A} = M$.
				\end{Def}
				\begin{Def}
					Множество A называется нигде не плотным в M, если $\overline{A}$ не содержит ни одного шара из M. 
				\end{Def}
				\begin{Lem}
					Замкнутое множество F - нигде не плотно в M $\Leftrightarrow$ $ \overline{(M \backslash F)} = M$
				\end{Lem}
				\begin{Thm}
					(О вложенных шарах)\newline
					Пусть M - полное метрическое пространство, $\overline{B}(x_{n+1}, r_{n+1}) \subset \overline{B}(x_n, r_n)$ - вложенные шары и $\lim\limits_{n\to\infty} r_n = 0$, тогда $\cap_{n=1}^{\infty} \overline{B}(x_n, r_n) \neq \emptyset$
				\end{Thm}
				\begin{Thm}
					Пусть M - полное метрическое пространство, $G_n \subset M$ - открытые множества, $\overline{G} = M$, тогда $\cap_{n=1}^{\infty} G_n \neq \emptyset$. 
				\end{Thm}
				\begin{Def}
					Множество называется множеством I категории, если оно представимо в виде не более чем счётного объединения нигде не плотных множеств. Иначе - II категории.
				\end{Def}
				\begin{Thm}
					(Бэра)\newline
					Полное пространство - множество II категории.
				\end{Thm}
		\section{Компактность в метрических пространствах}
			\begin{Def}
				Пространство M называется компактным, если из любой последовательность его элементов можно выделить сходящуюся подпоследовательность
			\end{Def}
			\begin{Def}
				Пространство M называется предкомпактным, если из любой последовательностми его элементов можно выделить фундаментальную последовательность
			\end{Def}
		 	\begin{St}
				М - предкомпактно и полно $\Rightarrow$ М - компактно	 		
		 	\end{St}
	 		\begin{Def}
	 			Пространство M называется ограниченным, если $\sup\limits_{x,y \in M} \rho(x,y) < \infty$. 	
	 		\end{Def}
 			\begin{Lem}
 				Предкомпактное пространство ограничено. 
 			\end{Lem}
 			\begin{Def}
 				Пусть М - метрическое. Множество $Q \subset M$ называется $\varepsilon$ - сетью для $E \subset M$, если $E \subset \cup_{x \in Q} B(x, \varepsilon)$			
 			\end{Def}
 			\begin{Def}
 				$M$ - вполне ограничено, если $\forall \varepsilon > 0$ существует конечная $\varepsilon$-сеть, покрывающая M. 
 			\end{Def}
 			\begin{St}
 				Из вполне ограниченности следует обычная ограниченность, но не наоборот. 
 			\end{St}
 			\begin{Thm}
 				M - предкомпактно $\Leftrightarrow$ M- вполне ограничено. 
 			\end{Thm}
 			\begin{Sled}
 				(признак предкомпактности) \newline
 				M - предкомпактно, если $\forall \varepsilon> 0$ существует конечная предкомпактная $\varepsilon$ - сеть, покрывающая M. 		
 			\end{Sled}
 			\begin{Thm}
 				(Гейне-Бореля) \newline
 				M- компактно $\Leftrightarrow$ из $\forall$ открытого покрытия M можно выделить конечное подпокрытие. 
 			\end{Thm}
 			\subsection{Критерии предкомпактности}
 				$K \subset \mathbb{R}^n$ - компакт
 				\begin{Thm}
 					Множество $E \subset C(K)$ предкомпактно $\Leftrightarrow$ E - равномерно ограничено и равностепенно непрерывно. 
 				\end{Thm}
 				\begin{Thm}
 					(Рисса)
 					Множество $E \subset L_p(K)$ предкомпактно $\Leftrightarrow$ E - равномерно ограничено и равностепенно в смысле $L_p$: 
 					\begin{enumerate}
 						\item $\exists M > 0$ $\|f\|_{L_p} \leq M$, $\forall f \in E$
 						\item $\forall \varepsilon > 0 \ \exists \delta > 0: \ \|f(x+\Delta) - f(x)\|_{L_p} < \varepsilon, \forall f \in E, |\Delta| < \delta$
 					\end{enumerate}
 				\end{Thm}
 				\bf{\underline{Случай $l_p$:}}\newline
 				$x \in l_p$, $x = (x_1, x_2, x_3, \dots)$ \newline
 				$P_N(x) = (x_1, \dots, x_N, 0, \dots)$\newline
 				$R_N(x) = x - P_N(x)$ \newline
 				$\|P_N(x)\|_{l_p} \leq \|x\|_{l_p}$
 				\begin{Thm}
 					Множество $E \subset l_p$ предкомпактно $\Leftrightarrow$ 
 					\begin{enumerate}
 						\item $\exists M> 0 \ \|x\|_{l_p} \leq M, \forall x \in E$
 						\item $\forall \varepsilon > 0 \exists N(\varepsilon): \|R_N(x) \| < \varepsilon, \forall x \in E$
 					\end{enumerate}
 				\end{Thm}
 	\section{Банаховы пространства}
 		Пусть X- линейное нормированное пространство. 
 		\begin{Def}
 			Если X - полное, то называется Банаховым. 
 		\end{Def}
 		\begin{Def}
 			Нормы $\|* \|_1$, $\|* \|_2$ называется эквивалентными, если $\exists c_1, c_2: \Rightarrow \ c_1 \|*\|_1 \leq \|*\|_2 \leq c_2 \|*\|_1$
 		\end{Def}
 		\begin{Def}
 			Пространство называется сепарабельным, если в нём существует счётное всюду плотное множество.
 		\end{Def}
 		\subsection{Отображения}
 			$X, Y$ - нормированные пространства.
 			\begin{Def}
 				 Отображение $f: X \to Y$ называется непрерывным в точке $x_0 \in X$, если $\forall \{x_n\}_{n=1}^{\infty}: \ x_n \to x$ $\Rightarrow \ f(x_n)\to f(x)$.			
 			\end{Def}
 			\begin{Lem}
 				Если линейный оператор непрерывен в точке $x_0 \in X$, то он непрерывен на всём X. 
 			\end{Lem}
 			\begin{Def}
 				Отображение $A$ называется ограниченным, если переводит любое ограниченное множество в ограниченное множество
 			\end{Def}
 			\begin{Def}
 				Норма оператора: $\|A\| = \sup\limits_{\|x\| \leq 1} \|Ax\|$
 			\end{Def}
 			\begin{Lem}
 				Пусть A - линейный, тогда $\|A\| = \sup\limits_{\|x\| = 1} \|Ax\| = \sup\limits_{x \geq 0} \dfrac{\|Ax\|}{\|x\|}$
 			\end{Lem}
 			\begin{Thm}
 				Линейный оператор A - непрерываен $\Leftrightarrow$ A - ограничен. 
 			\end{Thm}
 			\begin{Def}
 				$L(X,Y)$ - пространство линейных операторов, действующих из X в Y
 			\end{Def}
 			\begin{Thm}
 				Если Y - банахово, то $L(X, Y)$ - банахово. 			
 			\end{Thm}
 			\begin{Thm}
 				(Банаха-Штейнгауза)\newline
 				Пусть X, Y - нормированные пространства, $A_n \in L(X,Y)$ и $E = \{x: \ \overline{\lim\limits_{n\to\infty}}\|A_nx\|< \infty \}$ - множество II категории, тогда $\exists M > 0: \ \|A_n\| \leq M$.	
 			\end{Thm}
 			\begin{Sled}
 				Пусть X - банохово, Y - нормированное, $A_n \in L(X,Y)$ и $\overline{\lim\limits_{n\to\infty}}\|A_nx\|< \infty, \forall x \in X$, тогда $\exists M > 0: \ \|A_n\| \leq M$. 
 			\end{Sled}
 			\begin{Lem}
 				Пусть $x(t) \in C[a,b]$, $A(x) = \int\limits_a^b \phi(t)x(t) dt$, $\phi(t) \in L(a,b)$, тогда  $\|A\| = \int\limits_a^b | \phi(t) | dt$
 			\end{Lem}
 			\begin{Thm}
 				(о расходимости тригонометрического ряда)\newline
 				Пусть $S_n(f,x)$ -тригонометрический ряд для f, тогда $\exists f \in C[-\pi, \pi]: $ $S_n(f,0)$ - расходится. 
 			\end{Thm}
 	\section{Обратные операторы}
 		$A : X \to Y$, $X,Y$ - линейные нормированные пространства
 		\begin{Def}
 			Оператор $A_{L}^{-1}: Y \to X$ называется обратным левым, если $A_{L}^{-1} A = E$, правый аналогично
 		\end{Def}
 		\begin{Thm}
 			Если $\exists A_{L}^{-1},  A_{R}^{-1}$, то $ A_{L}^{-1} = A_{R}^{-1} = A^{-1}$
 		\end{Thm}
 		\begin{Thm}
 			Следующие три утверждения эквивалентны:
 			\begin{enumerate}
 				\item $Ax = y$ не может иметь двух решений x
 				\item $\ker{A} = \{0\}$
 				\item $\exists A_{L}^{-1}$
 			\end{enumerate}		
 		\end{Thm}
 		\begin{Thm}
 			Следующие три утверждения эквивалентны:
 			\begin{enumerate}
 				\item $Ax = y$ разрешимо
 				\item $R(A) = Y$, R - область значений оператора
 				\item $\exists A_{R}^{-1}$
 			\end{enumerate}		
 		\end{Thm}
 		\begin{Def}
 			Оператор A называется обратимым, если уравнение $Ax = y$ однозначно разрешимо и устойчиво к изменениям правой части $y$ (Обратимый - тот у которого существует ограниченный обратный)
 		\end{Def}
 		\begin{Thm}
 			Пусть X - банахово, Y - нормированное, $A: X \to Y$, $\exists M > 0: \ \|Ax\| \geq M\|x\|,\ \forall x \in X$, $\overline{R}(A) = Y$, тогда A - обратимый.
 		\end{Thm}
 		\begin{Thm}
 			Пусть X - банахово, $A:X\to X$ и $\|A\| < 1$, тогда $(E-A)$  - обратим. 
 		\end{Thm}
 		\begin{Thm}
 			Пусть X - банахово, $A:X \to X$	и A - ограничен, тогда
 			\begin{enumerate}
 				\item $\exists R = \lim\limits_{n\to\infty} \sqrt[n]{\|A^n\|}$ - спектральный радиус оператора
 				\item Если $R < 1$, то $E-A$ - обратим. 
 			\end{enumerate}		
 		\end{Thm}
 		\begin{Thm}
 			Пусть X - банахово, $A:X\to Y$ - обратимый, $B:X\to Y$, $\|A - B\| < \dfrac{1}{\|A^{-1}\|} $ $\Rightarrow$ B - обратим. 
 		\end{Thm}
 		\begin{Sled}
 			Множество обратимых операторов открыто. 
 		\end{Sled}
 		\begin{Thm}
 			Пусть X - банахово, $A: X \to Y$ -обратимый, $\{A_n\}: X \to Y$ и $\|A_n - A\| \to 0$ $\Rightarrow$ $A_n$ -обратимы, начиная с некоторого номера и $\|A_n^{-1} - A^{-1}\| \to 0$
 		\end{Thm}
 		\begin{Thm}
 			(Банаха об обратном операторе)\newline
 			Пусть $X, Y$ - банаховы пространства, $A: X\to Y$ - однозначно определённый ограниченный оператор, $D(A) = X, R(A) = Y$, тогда A - обратим. 
 		\end{Thm}
 		\begin{Sled}
 			Пусть X - полно относительно двух норм $\|*\|_1, \|*\|_2$ и $\exists M > 0: \ \|x\|_1 \leq M\|x\|_2, \forall x \in X$, тогда $\exists m> 0: \ \|x\|_2 \leq m\|x\|_1, \forall x \in X$
 		\end{Sled}
 		\begin{Def}
 			Оператор $A:X \to Y$ называется замкнутым, если $\forall x_n \to x: \ Ax_n \to y \Rightarrow x \in D(A), Ax = y$
 		\end{Def}
 		\begin{Def}
 			График оператора $A$- множество	$\Gamma(A) = \{(x,Ax): \forall x \in X\}$.
 		\end{Def}
 		\begin{St}
 			Оператор замкнут по норме $\Leftrightarrow$ его график замкнут по норме $\|x\|^{*} = \|x\| + \|Ax\|$
 		\end{St}
 		\begin{Thm}
 			(о замкнутом графике)\newline
 			Пусть $X, Y$ - банаховы, $A:X\to Y$, $D(A) = X$ линейный оператор A замкнут $\Rightarrow$ А - ограничен
 		\end{Thm}
 	\section{Функционалы}
 		\begin{Thm}
 			(Хана-Банаха)\newline
 			Пусть M - многообразие в X (М замкнуто относительно операции сложения и умножения на элемент из поля), X - линейное нормированное пространство и на $M$ задан линейный ограниченный функционал $f(x): M \to \mathbb{R}$, тогда $\exists F: X \to \mathbb{R}$ - продолжение $f$ с сохранением нормы: $f(x) = F(x), \forall x \in M$, $\|f\| = \|F\|$.\newline
 			(В нашем курсе теорема доказана только если X - сепарабельно, но теорема верна и в общем случае)
 		\end{Thm}
 		\begin{Sled}
 			Пусть $x_0 \in X, x_0 \neq 0$, тогда $\exists f(x):X \to \mathbb{R}$ - линейный ограниченный функционал такой, что $f(x_0) = \|x_0\|$ и $\|f\| = 1$. 			
 		\end{Sled}
 		\begin{Sled}
 			Если $f(x_0) = 0$ для любого линейного ограниченного функционала, то $x_0 = 0$.
 		\end{Sled}
 		\begin{Sled}
 			Пусть M - замкнутое многообразие в X, $M \neq X$, $x_0 \in X \backslash M$, тогда $\exists$ линейный ограниченный функционал $f(x): X \to \mathbb{R}: \ f(x) = 0 \forall x \in M, f(x_0) = 1$. 	
 		\end{Sled}
 	\subsection{Сопряженные пространства}
 		\begin{Def}
			$X^{*}$ - пространство линейных ограниченных функционалов над X. ($X^{*} = L(X, \mathbb{R})$). $X^{*}$ называется сопряженным к X
 		\end{Def}
 		\begin{St}
 			$X^{*}$ - полно, т.к. $\mathbb{R}$ - полно
 		\end{St}
 		\begin{Thm}
 			Из сепарабельности $X^{*}$ следует сепарабельность $X$. 
 		\end{Thm}
 		\begin{Def}
 			Пространство $X^{**}$ второе сопряженное пространство, $X^{**} = (X^{*})^{*}$ 
 		\end{Def}
 		\begin{Thm}
 			$X \subset X^{**}$
 		\end{Thm}
 		\begin{Lem}
 			$\tau_x(x^{*}) = x^{*}(x), x \in X, x^{*} \ in X^{*}$, тогда $\|\tau_x \| = \|x\|$
 		\end{Lem}
 		\begin{Def}
 			Пространство называется рефлексивным, если $X^{**} = X$
 		\end{Def}
 		\begin{Def}
 			Последовательность $\{x_n\}_{n=1}^{\infty} \subset X$ называется слабосходящейся, если $\forall x^{*} \in X^{*}$ сходится последовательность $x^{*}(x_n)$.
 		\end{Def}
 		\begin{Def}
 			Последовательность $\{x_n\}_{n=1}^{\infty} \subset X$ называется слабо фундаментальной, если $\forall x^{*} \in X^{*}$ последовательность $x^{*}(x_n)$ фундаментальна. 
 		\end{Def}
 		\begin{Lem}
 			Из сходимости по мере следует слабая сходимость(обратное неверно)
 		\end{Lem}
 		\begin{Lem}
 			Слабый предел единственен
 		\end{Lem}
 		\begin{Thm}
 			Слабо фундаментальная последовательность ограничена. 
 		\end{Thm}
 		\begin{Thm}
 			Рефлексивное пространство слабо полно
 		\end{Thm}
 		\begin{Thm}
 			(О слабой компактности)\newline
 			В сепарабельном рефлексивном пространстве из всякой ограниченной последовательности можно выделить слабо сходящуюся подпоследовательность. 
 		\end{Thm}
 	
\end{document}
 



































