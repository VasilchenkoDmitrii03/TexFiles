\documentclass[9pt, a4paper]{extarticle}

\usepackage[russian]{babel}
\usepackage{amsfonts, amssymb, amsmath, mathabx, dsfont}

% theorems, lemmas, etc.
\usepackage{amsthm}
\newtheorem*{theorem*}{Теорема}
\newtheorem{theorem}{Теорема}
\newtheorem{lemma}{Лемма}
\newtheorem{corollary}{Следствие}
\newtheorem{notabene}{Замечание}
\newtheorem{definition}{Определение}

% enumerating settings
\DeclareMathOperator*\lowlim{\underline{lim}}
\DeclareMathOperator*\uplim{\overline{lim}}

\numberwithin{equation}{section}
\numberwithin{lemma}{section}
\numberwithin{definition}{section}
\numberwithin{notabene}{section}
\numberwithin{corollary}{section}
\usepackage[left=30mm, top=20mm, right=15mm, bottom=20mm, nohead, footskip=10mm]{geometry}
\begin{document}
	\section{Гильбертовы пространства}
	\subsection{Определение и простейшие свойства гильбертова пространства}
	\begin{definition}
		Полное евклидово (унитарное) бесконечномерное пространство называется Гильбертовым (обычно обозначается $H$)
	\end{definition}
	\begin{theorem}
		Норма согласованная со скалярным произведением существует $\Leftrightarrow$ выполнено равенство $\|x+y\|^2 + \|x-y\|^2 = 2 \left(\|x\| + \|y\|\right)^2$
	\end{theorem}
	\subsection{Теорема об элементе с наименьшей нормой. Разложение гильбертова пространства
		в прямую ортогональную сумму подпространств }
	\begin{definition}
		Множество называется выпуклым, если вместе с любой парой своих точек оно содержит и соединяющий их отрезок		
	\end{definition}
	\begin{theorem}
		(об элементе с наименьшей нормой)\newline
		Пусть $M$ - замкнутое выпуклое подмножество $H$, тогда в $M$ существует элемент с наименьшей нормой и он единственен.
	\end{theorem}
	\begin{definition}
		Множество всех элементов $H$ ортогональных подмножеству $L$ называется ортогональным дополнением к $L$ (обозначается $L^\perp$)
	\end{definition}
	\begin{theorem}
		(о разложении Гильбертова пространства в сумму)\newline
		Пусть $L$ - замкнутое линейное подмножество $H$, тогда справедливо $H = L \oplus L^\perp$, т.е. $\forall x \in H \  \exists! \  x_1 \in L, x_2 \in L^\perp: \ x = x_1 + x_2$
	\end{theorem}
	\subsection{Теорема Рисса о представлении линейного ограниченного функционала}
	\begin{lemma}
		Пусть $f(x)$ - линейный ограниченный функционал над $H$ и $f \notequiv 0$, тогда $\dim{\left(\ker{f}\right)^\perp} = 1$
	\end{lemma}
	\begin{theorem}
		(Рисса о представлении линейного ограниченного функционала)\newline
		$\forall \ f(x) \in H^{*}$ $\exists! \ h \in H: \ f(x) = (x,h), \ \|f\| = \|h\|$
	\end{theorem}
	\subsection{Слабая сходимость}
		Свойства слабо сходящихся последовательностей:
		\begin{enumerate}
			\item $x_n \rightharpoondown x_0 , \ \|x_n\| \to \|x_0\| \Rightarrow x_n \to x_0$
			\item $x_n \rightharpoondown x \Rightarrow  \lowlim\limits_{n\to\infty} \|x_n\| \geq \|x\|$
			\item (Лемма Кадеца) $x_n \rightharpoondown x \Rightarrow \exists \{n_k\}: \ \dfrac{x_{n_1} + \dots + x_{n_k}}{k} \to x$
		\end{enumerate}
	\subsection{Полные, замкнутые, ортонормированные системы}
		\begin{definition}
			Система называется замкнутой в $H$, если любой элемент из $H$ можно приблизить конечной линейной комбинацией из элементов системы с наперёд заданной точностью.
		\end{definition}
		\begin{definition}
			Система $\left\{x_n\right\}_{n=1}^{\infty}$ называется полной, если из $(x,x_k) = 0, \forall k \in \mathbb{N}$ следует $x = 0$. 
		\end{definition}
		\begin{theorem}
			В $H$ понятие замкнутости и полноты эквивалентны.		
		\end{theorem}
		\begin{theorem}
				(Рисса-Фишера)\newline
				Пусть $\{e_n\}$ - полная система и пусть задана $\{c_k\} \subset \mathbb{C}: \ \sum\limits_{k=1}^{\infty} |c_k|^2 < \infty$ $\Rightarrow$  $\exists! \ x\in H : \ (x,e_k) = c_k$ и $\sum\limits_{k=1}^{\infty} |c_k|^2 = \|x\|^2$
		\end{theorem}
		\subsection{Процесс ортогонализации}
		\begin{theorem}
			В сепарабельном $H$ существует полная ортонормированная система.			
		\end{theorem}
		\begin{theorem}
			Все сепарабельные гильбертовы пространства изоморфны с изоментией между собой. 
		\end{theorem}
		\section{Пространства Соболева. Обобщённые решения краевых задач}
		\begin{definition}
		Пространство Соболева: Рассмотрим пространсво $C^1[0,1]$ со скалярным произведением $(u,v)_w = \int\limits_0^1 uvdt + \int\limits_0^1 u'v'dt$ пополним это пространство по норме, тогда получим пространство Соболева $W_2^1(0,1)$.
		\end{definition}
		\begin{definition}
			Рассмотрим  $\| u_n - u_m\|_{W_2^1(0,1)}$ - фундаментальная, тогда Обобщённой производной функции $u$ называется $\lim\limits_{n\to\infty} u'_n = v$
		\end{definition}
	
		\begin{lemma}
			Пусть $u(x) \in C^1[0,1]$, тогда $\|u\|_C \leq \sqrt2 \|u\|_{W_2^1(0,1)}$
		\end{lemma}
		\begin{theorem}
			(Вложения)\newline
			Пространство $W_2^1(0,1) \subset C(0,1)$, причем ограничено ($\exists M > 0: \ \|u\|_C \leq M \|u\|_{W_2^1(0,1)}$)
		\end{theorem}
		\begin{theorem}
			Вложение $W_2^1(0,1) \subset C(0,1)$ компактно. 
		\end{theorem}
		\begin{corollary}
			Из последовательности, ограниченной в $W_2^1(0,1) \subset C(0,1)$ можно выбрать подпоследовательность, сходяющуюся в $L_2[0,1]$.
		\end{corollary}
	\subsection{Обобщённые решения краевых задач}
	Пространство $\dot{W}_2^1(0,1) $ - пространство Соболева, но функции дополнительно обращаются в 0 на концах отрезка.\newline 
	\textbf{1-ая краевая задача}
	\begin{equation}
		\begin{cases}
			\left(a(t) u'(t)\right)' - c(t) u(t) = - f(t), \\
			u(0) = u(1) = 0, \\
			0 < a_0 \leq a(t) \leq a1 < \infty, \\
			0 \leq c_0 \leq c(t) \leq c1 < \infty, \\
			f(t) \in L_2(0,1), \\
			a(t), b(t) - \text{Ограниченные и измеримые на [0,1]}
		\end{cases}		
	\end{equation}
	\begin{definition}
		Обобщённым решением первой краевой задачи (2.1) называется функция $u \in \dot{W}_2^1(0,1)$, удовлетворяющая тождеству $\forall v \in \dot{W}_2^1(0,1)$
		$$
			\int_0^1 \left(a(t) u'(t) v'(t) + c(t) u(t) v(t) \right) dt = \int\limits_0^1 f(t)v(t)dt
		$$
	\end{definition}
	\begin{theorem}
		Обобщённое решение задачи (2.1) существует и единственно
	\end{theorem}
	\begin{lemma}
		(Неравенство Пуанкаре)\newline
		Пусть $u \in \dot{W}_2^1(0,1)$, тогда $\int\limits_0^1 u^2 dt \leq \int\limits (u')^2 dt$
	\end{lemma}
	\textbf{2-ая краевая задача}
	\begin{equation}
		\begin{cases}
			\left(a(t) u'(t)\right)' - c(t) u(t) = - f(t), \\
			u'(0) = u'(1) = 0, \\
			0 < a_0 \leq a(t) \leq a1 < \infty, \\
			0 < c_0 \leq c(t) \leq c1 < \infty, \\
			f(t) \in L_2(0,1), \\
			a(t), b(t) - \text{Ограниченные и измеримые на [0,1]}
		\end{cases}		
	\end{equation}
	\begin{definition}
		Обобщённым решением второй краевой задачи (2.2) называется функция $u \in \dot{W}_2^1(0,1)$, удовлетворяющая тождеству $\forall v \in \dot{W}_2^1(0,1)$
		$$
		\int_0^1 \left(a(t) u'(t) v'(t) + c(t) u(t) v(t) \right) dt = \int\limits_0^1 f(t)v(t)dt
		$$
	\end{definition}
\begin{theorem}
	Обобщённое решение задачи (2.2) существует и единственно
\end{theorem}
\end{document}