\documentclass[9pt, a4paper]{extarticle}

\usepackage[russian]{babel}
\usepackage{amsfonts, amssymb, amsmath, mathabx, dsfont}

% theorems, lemmas, etc.
\usepackage{amsthm}
\newtheorem*{theorem*}{Теорема}
\newtheorem{theorem}{Теорема}
\newtheorem{lemma}{Лемма}
\newtheorem{corollary}{Следствие}
\newtheorem{notabene}{Замечание}
\newtheorem{definition}{Определение}

% enumerating settings
\DeclareMathOperator*\lowlim{\underline{lim}}
\DeclareMathOperator*\uplim{\overline{lim}}

\numberwithin{equation}{section}
\numberwithin{lemma}{section}
\numberwithin{definition}{section}
\numberwithin{notabene}{section}
\numberwithin{corollary}{section}
\usepackage[left=30mm, top=20mm, right=15mm, bottom=20mm, nohead, footskip=10mm]{geometry}
\begin{document}
	\section{Гильбертовы пространства}
	\subsection{Определение и простейшие свойства гильбертова пространства}
	\begin{definition}
		Полное евклидово (унитарное) бесконечномерное пространство называется Гильбертовым (обычно обозначается $H$)
	\end{definition}
	\begin{theorem}
		Норма согласованная со скалярным произведением существует $\Leftrightarrow$ выполнено равенство $\|x+y\|^2 + \|x-y\|^2 = 2 \left(\|x\|^2 + \|y\|^2\right)$
	\end{theorem}
	\subsection{Теорема об элементе с наименьшей нормой. Разложение гильбертова пространства
		в прямую ортогональную сумму подпространств }
	\begin{definition}
		Множество называется выпуклым, если вместе с любой парой своих точек оно содержит и соединяющий их отрезок		
	\end{definition}
	\begin{theorem}
		(об элементе с наименьшей нормой)\newline
		Пусть $M$ - замкнутое выпуклое подмножество $H$, тогда в $M$ существует элемент с наименьшей нормой и он единственен.
	\end{theorem}
	\begin{definition}
		Множество всех элементов $H$ ортогональных подмножеству $L$ называется ортогональным дополнением к $L$ (обозначается $L^\perp$)
	\end{definition}
	\begin{theorem}
		(о разложении Гильбертова пространства в сумму)\newline
		Пусть $L$ - замкнутое линейное подмножество $H$, тогда справедливо $H = L \oplus L^\perp$, т.е. $\forall x \in H \  \exists! \  x_1 \in L, x_2 \in L^\perp: \ x = x_1 + x_2$
	\end{theorem}
	\subsection{Теорема Рисса о представлении линейного ограниченного функционала}
	\begin{lemma}
		Пусть $f(x)$ - линейный ограниченный функционал над $H$ и $f \notequiv 0$, тогда $\dim{\left(\ker{f}\right)^\perp} = 1$
	\end{lemma}
	\begin{theorem}
		(Рисса о представлении линейного ограниченного функционала)\newline
		$\forall \ f(x) \in H^{*}$ $\exists! \ h \in H: \ f(x) = (x,h), \ \|f\| = \|h\|$
	\end{theorem}
	\subsection{Слабая сходимость}
		Свойства слабо сходящихся последовательностей:
		\begin{enumerate}
			\item $x_n \rightharpoondown x_0 , \ \|x_n\| \to \|x_0\| \Rightarrow x_n \to x_0$
			\item $x_n \rightharpoondown x \Rightarrow  \lowlim\limits_{n\to\infty} \|x_n\| \geq \|x\|$
			\item (Лемма Кадеца) $x_n \rightharpoondown x \Rightarrow \exists \{n_k\}: \ \dfrac{x_{n_1} + \dots + x_{n_k}}{k} \to x$
		\end{enumerate}
	\subsection{Полные, замкнутые, ортонормированные системы}
		\begin{definition}
			Система называется замкнутой в $H$, если любой элемент из $H$ можно приблизить конечной линейной комбинацией из элементов системы с наперёд заданной точностью.
		\end{definition}
		\begin{definition}
			Система $\left\{x_n\right\}_{n=1}^{\infty}$ называется полной, если из $(x,x_k) = 0, \forall k \in \mathbb{N}$ следует $x = 0$. 
		\end{definition}
		\begin{theorem}
			В $H$ понятие замкнутости и полноты эквивалентны.		
		\end{theorem}
		\begin{theorem}
				(Рисса-Фишера)\newline
				Пусть $\{e_n\}$ - полная система и пусть задана $\{c_k\} \subset \mathbb{C}: \ \sum\limits_{k=1}^{\infty} |c_k|^2 < \infty$ $\Rightarrow$  $\exists! \ x\in H : \ (x,e_k) = c_k$ и $\sum\limits_{k=1}^{\infty} |c_k|^2 = \|x\|^2$
		\end{theorem}
		\subsection{Процесс ортогонализации}
		\begin{theorem}
			В сепарабельном $H$ существует полная ортонормированная система.			
		\end{theorem}
		\begin{theorem}
			Все сепарабельные гильбертовы пространства изоморфны с изоментией между собой. 
		\end{theorem}
		\section{Пространства Соболева. Обобщённые решения краевых задач}
		\begin{definition}
		Пространство Соболева: Рассмотрим пространсво $C^1[0,1]$ со скалярным произведением $(u,v)_w = \int\limits_0^1 uvdt + \int\limits_0^1 u'v'dt$ пополним это пространство по норме, тогда получим пространство Соболева $W_2^1(0,1)$.
		\end{definition}
		\begin{definition}
			Рассмотрим  $\| u_n - u_m\|_{W_2^1(0,1)}$ - фундаментальная, тогда Обобщённой производной функции $u$ называется $\lim\limits_{n\to\infty} u'_n = v$
		\end{definition}
	
		\begin{lemma}
			Пусть $u(x) \in C^1[0,1]$, тогда $\|u\|_C \leq \sqrt2 \|u\|_{W_2^1(0,1)}$
		\end{lemma}
		\begin{theorem}
			(Вложения)\newline
			Пространство $W_2^1(0,1) \subset C(0,1)$, причем ограничено ($\exists M > 0: \ \|u\|_C \leq M \|u\|_{W_2^1(0,1)}$)
		\end{theorem}
		\begin{theorem}
			Вложение $W_2^1(0,1) \subset C(0,1)$ компактно. 
		\end{theorem}
		\begin{corollary}
			Из последовательности, ограниченной в $W_2^1(0,1) \subset C(0,1)$ можно выбрать подпоследовательность, сходяющуюся в $L_2[0,1]$.
		\end{corollary}
	\subsection{Обобщённые решения краевых задач}
	Пространство $\dot{W}_2^1(0,1) $ - пространство Соболева, но функции дополнительно обращаются в 0 на концах отрезка.\newline 
	\textbf{1-ая краевая задача}
	\begin{equation}
		\begin{cases}
			\left(a(t) u'(t)\right)' - c(t) u(t) = - f(t), \\
			u(0) = u(1) = 0, \\
			0 < a_0 \leq a(t) \leq a1 < \infty, \\
			0 \leq c_0 \leq c(t) \leq c1 < \infty, \\
			f(t) \in L_2(0,1), \\
			a(t), b(t) - \text{Ограниченные и измеримые на [0,1]}
		\end{cases}		
	\end{equation}
	\begin{definition}
		Обобщённым решением первой краевой задачи (2.1) называется функция $u \in \dot{W}_2^1(0,1)$, удовлетворяющая тождеству $\forall v \in \dot{W}_2^1(0,1)$
		$$
			\int_0^1 \left(a(t) u'(t) v'(t) + c(t) u(t) v(t) \right) dt = \int\limits_0^1 f(t)v(t)dt
		$$
	\end{definition}
	\begin{theorem}
		Обобщённое решение задачи (2.1) существует и единственно
	\end{theorem}
	\begin{lemma}
		(Неравенство Пуанкаре)\newline
		Пусть $u \in \dot{W}_2^1(0,1)$, тогда $\int\limits_0^1 u^2 dt \leq \int\limits (u')^2 dt$
	\end{lemma}
	\textbf{2-ая краевая задача}
	\begin{equation}
		\begin{cases}
			\left(a(t) u'(t)\right)' - c(t) u(t) = - f(t), \\
			u'(0) = u'(1) = 0, \\
			0 < a_0 \leq a(t) \leq a1 < \infty, \\
			0 < c_0 \leq c(t) \leq c1 < \infty, \\
			f(t) \in L_2(0,1), \\
			a(t), b(t) - \text{Ограниченные и измеримые на [0,1]}
		\end{cases}		
	\end{equation}
	\begin{definition}
		Обобщённым решением второй краевой задачи (2.2) называется функция $u \in W_2^1(0,1)$, удовлетворяющая тождеству $\forall v \in W_2^1(0,1)$
		$$
		\int_0^1 \left(a(t) u'(t) v'(t) + c(t) u(t) v(t) \right) dt = \int\limits_0^1 f(t)v(t)dt
		$$
	\end{definition}
\begin{theorem}
	Обобщённое решение задачи (2.2) существует и единственно
\end{theorem}
\subsection{Обобщённое решение краевых задач для уравнений в частных производных}
\textbf{1-ая краевая задача}\newline
$D \subset \mathbb{R}^n$ - ограниченная область. 
\begin{equation}
	\begin{cases}
		\sum\limits_{i,j=1}^n \dfrac{\partial }{\partial x_j} \left(a_{ij}(x) \dfrac{\partial u}{\partial x_i}\right) - c(x)u(x) = - f(x) \\
		u\vert_{\sigma D} = 0, \\
		a_0 \|\xi\| \leq \sum\limits_{i,j=1}^{n} a_{ij}(x) |\xi_i| |\xi_j| \leq a_1 \|\xi\| \\ 	
		0 < a_0 \leq a(x) \leq a1 < \infty, \\
		0 \leq c_0 \leq c(x) \leq c1 < \infty, \\
		f(x) \in L_2(D), \\
		a(x), b(x) - \text{Ограниченные и измеримые на D}
	\end{cases}		
\end{equation}
$\dot{W}_2^1(D)$ - аналогично функции обращаются в 0 на границе.
\begin{definition}
	Обобщённым решением второй краевой задачи (2.3) называется функция $u \in \dot{W}_2^1(D)$, удовлетворяющая тождеству $\forall v \in \dot{W}_2^1(D)$
	$$
	\int_D \left(\sum\limits_{i,j = 1}^n a_{ij}(x) \dfrac{\partial u}{\partial x_i} \dfrac{\partial v}{\partial x_i} + c(x) u(x) v(x) \right) dx = \int\limits_D f(x)v(x)dx
	$$
\end{definition}
\begin{lemma}
	(Неравенство Пуанкаре)\newline
	$u(x) \in \dot{W}_2^1(D)$, тогда справедливо $\int_D u^2(x) dx  \leq C \int_D (\nabla u)^2 dx$, $C$ зависит только от области.
\end{lemma}
\begin{theorem}
	Обобщённое решение задачи (2.3) существует и единственно.
\end{theorem}
\textbf{2-ая краевая задача}\newline
Все аналогично, только на границе $\dfrac{\partial u}{\partial n}|_{\sigma D} = 0$ и $c_0 > 0$. 
\section{Компактные (вполне непрерывные) операторы в гильбертовом пространстве}
\subsection{Сопряженный оператор}
	\begin{definition}
		Оператор B называется сопряжённым к оператору A, если $\forall x, y \in H: \ (Ax, y) = (x, By)$
	\end{definition}
	\begin{theorem}
		Пусть A - линейный ограниченный оператор, тогда $\exists! A^*$ - линейный и ограниченный и $\|A\| = \|A^*\|$
	\end{theorem}
\subsection{Вполне непрерывные операторы}
	\begin{definition}
		Оператор A называется вполне непрерывным, если слабосходяющуюся последовательность переводит в сильно сходяющуюся.
	\end{definition}
	\begin{lemma}
		Пусть A - линейный ограниченный, тогда из $x_n \rightharpoondown x$ следует $Ax_n  \rightharpoondown  Ax$
	\end{lemma}
	\begin{definition}
		Пусть A - вполне непрерывный, тогда $A^*$ - вполне непрерывный
	\end{definition}
\subsection{Компактный оператор}
	\begin{definition}
		Оператор A называется компактным, если ограниченное множество переводит в предкомпактное.
	\end{definition}
	\begin{theorem}
		Оператор A - компактный $\Leftarrow$ он вполне непрерывен. 
	\end{theorem}
\subsection{Приближение компактных операторов}
	\begin{theorem}
		Пусть A - ограниченный, $A_n$ - вполне непрерывны и $\|A_n\| \to \|A\|$, тогда A - компактный. 	
	\end{theorem}
	\begin{lemma}
		Пусть A - компактный, тогда $\exists z \in H: \|z\| = 1$ и $\|Az\| = \|A\|$.
	\end{lemma}
	\begin{theorem}
		Пусть $\{e_n\}_{n=1}^\infty$ - ОНБ в сепарабельном H. $P_nx  = \sum\limits_{k=1}^{n} (x,e_k) e_k$. Пусть A - компактный, тогда $\|A - P_nAP_n\| \to 0$.
	\end{theorem}
\section{Теория Фредгольма для вполне непрерывных операторов}
\subsection{Третья теорема фредгольма}
	Будем рассматривать оператор $T = E - A$, где A - вполне непрерывный.
	\begin{theorem}
		$\exists a > 0: \ \|Tx\| \geq a \|x\|, \forall x \perp \ker{T}$
	\end{theorem}
	\begin{theorem}
		$R(T)$ - замкнуто
	\end{theorem}
	\begin{theorem}
		Пусть B - линейный ограниченный оператор, тогда справедливо разложение $H = \ker B \oplus \overline{R(B^*)} = \ker B^* \oplus \overline{R(B)}$
	\end{theorem}
	\begin{theorem}
		(III - Фредгольма) \newline
		Уравнение $Tx = y$ разрешимо $\Leftrightarrow$ $y \perp \ker T^*$
	\end{theorem}
\subsection{Первая теорема Фредгольма}
	\begin{theorem}
		$def\ T = \dim\ker T < \infty$ 
	\end{theorem}
	\begin{theorem}
		(о стабилизации ядер)\newline
		$\exists N \in \mathbb{N}: $ $\ker T \subset \ker T^2 \subset \dots \subset \ker T^N = \ker T^{N+1} = \dots$
	\end{theorem}
	\begin{theorem}
		(I - Фредгольма) \newline
		Уравнение $Tx = y$ разрешимо при $\forall$ правой части $\Leftrightarrow$ $\ker{T} = \emptyset$
	\end{theorem}
\subsection{Вторая теорема Фредгольма}
	\begin{theorem}
		(II - Фредгольма) \newline
		$def\ T = def\ T^* < \infty$
	\end{theorem}
\subsection{Общее операторное уравнение. Альтернатива Фредгольма}
	\begin{equation*}
		(A - \lambda E) x = y, \ \lambda \neq 0 \in \mathbb{C}, \ A - \text{вп. непр.}
	\end{equation*}
	можем переписать в виде
	\begin{equation*}
		(E - \tilde{A}) x = \tilde{y}, \ \tilde{y} = -y / \lambda, \ \tilde{A}= - A / \lambda
	\end{equation*}
	\begin{theorem}
		Уравнение $	(A - \lambda E) x = y$ разрешимо для любой правой части $\Leftrightarrow$ $\ker{A - \lambda E} = \{0\}$
	\end{theorem}
	\begin{theorem}
		$def (A - \lambda E) = def (A^* - \overline{\lambda} E) < \infty$
	\end{theorem}
	\begin{theorem}
		Уравнение $	(A - \lambda E) x = y$ разрешимо $\Leftrightarrow$ $y \perp \ker{A^* - \overline{\lambda} E}.$
	\end{theorem}
	\begin{theorem}
		(Альтернатива Фредгольма)\newline
		Либо уравнение $	(A - \lambda E) x = y$ разрешимо для любой правой части, либо $\ker{A - \lambda E} \neq \{0\}$
	\end{theorem}
\section{Спектральная теория линейный ограниченных операторов}
\subsection{Спектр оператора}
$X$ - банахово.
	\begin{definition}
		Точка $\lambda \in \mathbb{C}$ называется регулярной точкой оператора A, если 
		\begin{enumerate}
			\item $\ker{(A - \lambda E)} = \{0\}$
			\item $R(A - \lambda E) = X$
			\item $\exists (A - \lambda E)^{-1}$ - ограниченный и определённый на всем X. 
		\end{enumerate}
	\end{definition}
	\begin{definition}
		Множество регулярных точек оператора A обозначаем $\rho(A)$
	\end{definition}
	\begin{definition}
		$\sigma(A) = \mathbb{C} \backslash \rho(A)$  - спектр оператора A. 
	\end{definition}
	\begin{theorem}
		Пусть $A$ - ограниченный оператор и $|\lambda| > \|A\|$ $\Rightarrow$  $\lambda \in \rho(A)$. 
	\end{theorem}
	\begin{definition}
		$R_A(\lambda) = (A - \lambda E) ^{-1}$ - резольвента оператора A. 
	\end{definition}
	\begin{theorem}
		Пусть $A$ - ограниченный оператор, $\lambda \in \rho(A)$, $|\Delta| < \dfrac{1}{\|R_A(\lambda)\|}$ $\Rightarrow$ $\lambda + \Delta \in \rho(A)$.  
	\end{theorem}
	\begin{theorem}
		(Тождество Гильберта)\newline
		Если A - ограниченный и $\lambda, \mu \in \rho(A)$, то 
		$R_A(\lambda) - R_A(\mu) = (\lambda - \mu) R_A(\lambda) R_A(\mu)$
	\end{theorem}
	\begin{theorem}
		H - гильбертово. Пусть $A: H \to H$ - линейный ограниченный оператор, тогда $\sigma(A) \neq \emptyset$.		
	\end{theorem}
\subsection{Спектр вполне непрерывного оператора}
	\begin{definition}
		(Классификация точек спектра) \newline
		Пусть $\lambda \in \sigma(A)$, тогда
		\begin{enumerate}
			\item Если $\ker{(A - \lambda E)} \neq \{0\}$, то $\lambda$ приндалжеит точечному спектру $\sigma_p(A)$.
			\item Если $\ker{(A - \lambda E)} = \{0\}$, $R(A - \lambda E) \neq X$, но $\overline{R(A - \lambda E)} = X$, то $\lambda$ принадлежит непрерывному спектру $\sigma_c(A)$.
			\item Если $\ker{(A - \lambda E)} = \{0\}$, $R(A - \lambda E) \neq X$ и $\overline{R(A - \lambda E)} \neq X$, то $\lambda$ принадлежит остаточному спектру $\sigma_r(A)$.
		\end{enumerate}	
	\end{definition}
	\begin{theorem}
		Пусть A - вполне непрерывный оператор и $\lambda \neq 0  \in \sigma(A)$, тогда $\lambda \in \sigma_p(A)$.
	\end{theorem}
	\begin{theorem}
		Если $\dim{H} = \infty$ и $A$ - вполне непрерывный, то $0 \in \sigma(A)$. 	
	\end{theorem}
	\begin{theorem}
		Пусть A - вполне непрерывный оператор, тогда если в спектре $\sigma(A)$ есть последовательность $\lambda_n$, то $\lambda_n \to 0$. 
	\end{theorem}
	
\subsection{Спектр самосопрожяенного оператора}
	$H$ - гильбретово. $A: H \to H$ - линейный ограниченный самосопряженный. 
	\begin{theorem}
		Пусть A - ограниченный самосопряженный оператор, тогда $\|A\| = \sup\limits_{\|x\| = 1} (Ax, x) = \mu$. 
	\end{theorem}
	\begin{theorem}
		Ограниченный линейный оператор A - самоспряженный $\Leftrightarrow$ $Im \ (Ax, x) = 0, \forall x \in X$
	\end{theorem}
	\begin{theorem}
		Пусть A ограниченный линейный самосопряженный оператор, тогда $\sigma(A) \subset \mathbb{R}$
	\end{theorem}
	\begin{lemma}
		Собственные вектора $A$ отвечающие различным собственным значениям ортгональны. 
	\end{lemma}
\subsection{Теорема Гильберта-Шмидта}
	Пусть H - гильбертово. $A: H \to H$ - линейный вполне непрервный самосопряженный оператор.
	\begin{theorem}
		Пусть $M = \sup\limits_{\|x\| = 1} (Ax, x), -m = \inf\limits_{\|x\| = 1} (Ax, x)$, тогда $\sigma(A) \subset [-m, M]$, если $dim\ H = \infty$, то $0 \in [-m, M]$. 
	\end{theorem}
	\begin{theorem}
		$\exists \lambda$ - собственное значение A: $\|A\| = |\lambda|$
	\end{theorem}
	\begin{theorem}
		(Гильберта-Шмидта)\newline
		В замыкании образа оператора A содержится полная ортноромированная система собственных векторов, отвечающих $\lambda \neq 0$
	\end{theorem}
\subsection{Теорема Гильберта-Шмидта для интегрального оператора}
	Пусть $Ax(t) = \int_D K(t,s) x(s) ds$ - интегральный оператор.
	\begin{enumerate}
		\item $K(t,s) = \overline{K(s,t)}$
		\item D - ограниченная область
		\item $\int_D |K(t,s)|^2 ds \leq C , \forall t \in D$
	\end{enumerate}
	\begin{theorem}
		Если $y = Ax$, то ряд по собственным функциям A сходится абсолютно и равномерно в D к функции $y(t)$. 
	\end{theorem}
\section{Нелинейные операторы. Теорема Шаудера о неподвижной точке}
\subsection{Теорема Брауэра о неподвижной точке}
 \begin{theorem}
 	(Брауэра)\newline
 	Любое непрерывное отображение замкнутого шара в себя в конечномерном нормированном пространстве имеет неподвижную точку. 
 \end{theorem}
\end{document}