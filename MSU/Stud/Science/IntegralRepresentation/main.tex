\documentclass[12pt, a4paper]{extarticle}

\usepackage[russian]{babel}
\usepackage{amsfonts, amssymb, amsmath, mathabx, dsfont}
% theorems, lemmas, etc.
\usepackage{amsthm}
\newtheorem*{theorem*}{Теорема}
\newtheorem{theorem}{Теорема}
\newtheorem{lemma}{Лемма}
\newtheorem{corollary}{Следствие}
\newtheorem{notabene}{Замечание}
\newtheorem{definition}{Определение}

% enumerating settings
\numberwithin{equation}{section}
\numberwithin{lemma}{section}
\numberwithin{definition}{section}
\numberwithin{notabene}{section}
\numberwithin{corollary}{section}

\begin{document}
	\section*{Интегральное представление решения уравнения Лапласа}
	Ранее была рассмотрена задача
	\begin{equation}
		\dfrac{\partial^2 u}{\partial x^2} +\dfrac{\partial^2 u}{\partial y^2} = 0
	\end{equation}
	В полуполосе $D = \{(x,y) \vert 0 < x < \pi, 0 < y\}$\newline
	В классе функций $u(x,y) \in C(\overline{D}) \cap C^1(\overline{D} \cap \{y > 0\}) \cap C^2 (D)$ \newline
	с граничными условиями
	\begin{equation}
		u(0, y) = 0, \ \dfrac{\partial u}{\partial x} (\pi, y) = 0
	\end{equation}
	\begin{equation}
		\lim\limits_{y \to 0 + 0} \int\limits_0^\pi \left[\dfrac{\partial u}{\partial y} - \dfrac{\partial u}{\partial x} + \varphi(x) \right]^2 dx = 0, \ \varphi(x) \in L_2[0,\pi]
	\end{equation}
	\begin{equation}
		u(x,y) \rightrightarrows 0, y \to \infty
	\end{equation}
	И была доказана теорема:
	\begin{theorem}
		Решение задачи (2.1 - 2.4) существует и единственно, причём его можно представить в виде ряда
		\begin{equation}
			u(x,y) = \sum\limits_{n=0}^{\infty} A_n e^{-\left(n + \dfrac12\right)y} \sin{\left[\left(n + \dfrac12\right)x\right]},
		\end{equation}
		где коэффициенты $A_n, \ n =0,1,2, \dots$ находятся из разложения
		\begin{equation}
			\sum\limits_{n=0}^{\infty} A_n \left(n + \dfrac12 \right) \sin{\left[\left(n +\dfrac12\right)x + \dfrac\pi4\right]} = \dfrac{\varphi(x)}{\sqrt2}
		\end{equation}
	\end{theorem}
	\newpage
	Рассмотрим теперь интегральное представление производных решения этой задачи:
	\begin{theorem}
		Пусть $u(x,y)$ - решение задачи $(1)-(4)$, тогда $u_x, u_y$ представимы в виде
		\begin{equation}
				u_y(x,y) = - Im\  \dfrac{ \sqrt{1 - e^{i2z}} }{\pi} e^{\dfrac{+iz}{2}} \int\limits_0^\pi  \dfrac{\sqrt{\sin{t}}}{\left(1 - e^{i(z+t)} \right) \left(1 - e^{i(z-t)}\right)}  \varphi(t) dt
		\end{equation}
		\begin{equation}
			u_x(x,y) = Re\   \dfrac{ \sqrt{1 - e^{i2z}} }{\pi} e^{\dfrac{+iz}{2}} \int\limits_0^\pi  \dfrac{\sqrt{\sin{t}}}{\left(1 - e^{i(z+t)} \right) \left(1 - e^{i(z-t)}\right)}  \varphi(t) dt
		\end{equation}
	\end{theorem}
	\begin{proof}:
		\newline
		Рассмотрим уравнение (6). Система синусов $\sin{\left[\left(n +\dfrac12\right)x + \dfrac\pi4\right]}$ образует базис в $L_2(0,\pi)$. Поэтому для коэффициентов $A_n\left(n+\dfrac12\right)$ справедливо следующее представление:
		\begin{equation*}
			A_n\left(n+\dfrac12\right) = \int\limits_0^\pi h_{n+1}(t) \dfrac{\varphi(t)}{\sqrt2} dt, 
	\end{equation*}
	где
	\begin{equation*}
		h_n(t) = \dfrac{2}{\pi}\dfrac{(2\cos{t/2})^\beta}{(\tan{t/2})^{\gamma/\pi}} \sum\limits_{k=1}^n \sin{kt} B_{n-k}
	\end{equation*}
	Пусть $u(x,y)$ - решение задачи (1)-(4), тогда
	\begin{equation*}
		u(x,y) = \sum\limits_{n=0}^{\infty} A_n e^{-\left(n + \dfrac12\right)y} \sin{\left[\left(n + \dfrac12\right)x\right]}
	\end{equation*}
	и соотвественно
	\begin{equation*}
		u_y(x,y) = -\sum\limits_{n=0}^{\infty} A_n \left(n +\dfrac12\right) e^{-\left(n + \dfrac12\right)y} \sin{\left[\left(n + \dfrac12\right)x\right]}
	\end{equation*}
	Здесь как раз возникает нужный нам коэффициент $A_n \left(n+\dfrac12\right)$, поэтому
	\begin{equation*}
		u_y(x,y)  = - \sum\limits_{n=0}^{\infty}  \int\limits_0^\pi \dfrac{\varphi(t)}{\sqrt2}  h_{n+1}(t)  e^{-\left(n + \dfrac12\right)y} \sin{\left[\left(n + \dfrac12\right)x\right]} dt
	\end{equation*}
	$\sin{\left[\left(n + \dfrac12\right)x\right]} = Im \ e^{i\left(n + \dfrac12\right)x}$, поэтому
	\begin{equation*}
		u_y(x,y)  = -  Im \ \sum\limits_{n=0}^{\infty}  \int\limits_0^\pi \dfrac{\varphi(t)}{\sqrt2}  h_{n+1}(t)  e^{-\left(n + \dfrac12\right)y} e^{i\left(n + \dfrac12\right)x} dt
	\end{equation*}
	Обозначим $z = x + iy$
\begin{equation*}
	u_y(x,y)  = -  Im \ \sum\limits_{n=0}^{\infty}  \int\limits_0^\pi \dfrac{\varphi(t)}{\sqrt2}  h_{n+1}(t)  e^{i\left(n+\dfrac12\right) z}  dt
\end{equation*}
	Для дальнейших операций нам было бы удобно, чтобы суммирование начинолось от 1, а не 0, поэтому сделаем замену $m = n +1$
	\begin{equation*}
		u_y(x,y)  = -  Im \ \sum\limits_{m=1}^{\infty}  \int\limits_0^\pi \dfrac{\varphi(t)}{\sqrt2}  h_{m}(t)  e^{i\left(m-\dfrac12\right) z}  dt
	\end{equation*}
	\begin{equation*}
	u_y(x,y)  = -  Im \ e^{-\dfrac{iz}{2}}\ \sum\limits_{m=1}^{\infty}  \int\limits_0^\pi \dfrac{\varphi(t)}{\sqrt2}  h_{m}(t)  e^{im z}  dt
\end{equation*}
	Поменяем местами знаки интергирования и суммирования
	\begin{equation*}
		u_y(x,y)  = -  Im \ e^{-\dfrac{iz}{2}}\  \int\limits_0^\pi \dfrac{\varphi(t)}{\sqrt2}  \sum\limits_{m=1}^{\infty}   h_{m}(t)  e^{im z}  dt
	\end{equation*}
	Введём новое обозначение:
	\begin{equation*}
		I(t,z) = \sum\limits_{m=1}^{\infty}  h_{m}(t)  e^{im z}
	\end{equation*}
	\begin{equation*}
		I(t,z) =\dfrac{2}{\pi}\dfrac{(2\cos{t/2})^\beta}{(\tan{t/2})^{\gamma/\pi}} \sum\limits_{n=1}^{\infty}   \sum\limits_{k=1}^n \sin{kt} B_{n-k} e^{inz} = 
		\dfrac{2}{\pi}\dfrac{(2\cos{t/2})^\beta}{(\tan{t/2})^{\gamma/\pi}} \sum\limits_{k=1}^{\infty} \sin{kt} \sum\limits_{n=k}^{\infty} e^{inz} B_{n-k}
	\end{equation*}
	Введём новый индекс $m = n - k$
	\begin{equation*}
		I(t,z) = \dfrac{2}{\pi}\dfrac{(2\cos{t/2})^\beta}{(\tan{t/2})^{\gamma/\pi}} \sum\limits_{k=1}^{\infty} \sin{kt} \sum\limits_{m =0 }^{\infty} e^{i(m+k)z} B_{m} = 
		\dfrac{2}{\pi}\dfrac{(2\cos{t/2})^\beta}{(\tan{t/2})^{\gamma/\pi}} \sum\limits_{k=1}^{\infty} e^{ikz}\sin{kt} \sum\limits_{m =0 }^{\infty} e^{imz} B_{m}
	\end{equation*}
	Первый ряд можем вычислить по формуле суммы бесконечно убывающей геометрической прогрессии
	\begin{equation*}
		 \sum\limits_{k=1}^{\infty} e^{ikz}\sin{kt} =  \sum\limits_{k=1}^{\infty} e^{ikz}\dfrac{1}{2i}\left(e^{ikt} - e^{-ikt}\right) = \dfrac1{2i} \left(\dfrac{1}{1 - e^{i(z+t)}} -  \dfrac{1}{1 - e^{i(z-t)}}\right) = 
	\end{equation*}
	\begin{equation*}
		= \dfrac{1}{2i}  \dfrac{e^{i(z+t)} - e^{i(z-t)}}{\left(1 - e^{i(z+t)} \right) \left(1 - e^{i(z-t)}\right)} =  \dfrac{e^{iz} \sin{t}}{\left(1 - e^{i(z+t)} \right) \left(1 - e^{i(z-t)}\right)}
	\end{equation*}
	Рассмотрим второй ряд:
	\begin{equation*}
		\sum\limits_{l =0 }^{\infty} e^{ilz} B_{l} = \sum\limits_{l =0 }^{\infty} e^{ilz} \sum\limits_{m=0}^{l} C^{l - m}_{\gamma/\pi} C^{m}_{-\gamma/\pi - \beta} (-1)^{l-m} = \sum\limits_{m=0}^{\infty} \sum\limits_{l=m}^{\infty} e^{ilz} C^{l - m}_{\gamma/\pi} C^{m}_{-\gamma/\pi - \beta} (-1)^{l-m} = 
	\end{equation*}
	Введём новый индекс суммирования $k = l -m$
	\begin{equation*}
		\sum\limits_{m=0}^{\infty} \sum\limits_{k=0}^{\infty} e^{i(m+k)z} C^{k}_{\gamma/\pi} C^{m}_{-\gamma/\pi - \beta} (-1)^{k} = \sum\limits_{m=0}^{\infty} e^{imz} C^{m}_{-\gamma/\pi - \beta} \sum\limits_{k=0}^{\infty}  C^{k}_{\gamma/\pi} (-1)^k e^{ikz} = (1 + e^{iz})^{-\gamma/\pi - \beta} (1- e^{iz})^{\gamma/\pi} 
	\end{equation*}
	В нашем случае $\beta = -1, \gamma = \pi/2$, поэтому
	\begin{equation*}
		= (1 + e^{iz})^{1/2} (1- e^{iz})^{1/2} =\sqrt{1 - e^{i2z}} 
	\end{equation*}
	Собираем все решение:
	\begin{equation*}
		u_y(x,y) = - Im\ e^{\dfrac{-iz}{2}} \int\limits_0^\pi \dfrac{\varphi(t)}{\sqrt2} I(t,z) dt 
	\end{equation*}
	\begin{equation*}
	u_y(x,y) = - Im\ e^{\dfrac{-iz}{2}} \int\limits_0^\pi \dfrac{2}{\pi}\dfrac{(2\cos{t/2})^\beta}{(\tan{t/2})^{\gamma/\pi}}  \dfrac{e^{iz} \sin{t}}{\left(1 - e^{i(z+t)} \right) \left(1 - e^{i(z-t)}\right)} \sqrt{1 - e^{i2z}} \dfrac{\varphi(t)}{\sqrt2} dt
	\end{equation*}
	Подставляя $\beta$ и $\gamma$ получим
		\begin{equation*}
		u_y(x,y) = - Im\  \dfrac{2}{\pi} e^{\dfrac{-iz}{2}} \int\limits_0^\pi \dfrac{1}{2\cos{t/2} \sqrt{\tan{t/2}}}  \dfrac{e^{iz} \sin{t}}{\left(1 - e^{i(z+t)} \right) \left(1 - e^{i(z-t)}\right)} \sqrt{1 - e^{i2z}} \dfrac{\varphi(t)}{\sqrt2} dt
	\end{equation*}
	\begin{equation*}
	u_y(x,y) = - Im\  \dfrac{e^{\dfrac{+iz}{2}}}{\pi}  \int\limits_0^\pi  \dfrac{\sqrt{\sin{t}} \sqrt{1 - e^{i2z}}}{\left(1 - e^{i(z+t)} \right) \left(1 - e^{i(z-t)}\right)}  \varphi(t) dt
\end{equation*}
	\end{proof}
\end{document}