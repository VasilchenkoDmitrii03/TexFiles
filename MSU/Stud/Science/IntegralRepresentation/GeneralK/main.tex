\documentclass[12pt, a4paper]{extarticle}

\usepackage[russian]{babel}
\usepackage{amsfonts, amssymb, amsmath, mathabx, dsfont}
% theorems, lemmas, etc.
\usepackage{amsthm}
\newtheorem*{theorem*}{Теорема}
\newtheorem{theorem}{Теорема}
\newtheorem{lemma}{Лемма}
\newtheorem{corollary}{Следствие}
\newtheorem{notabene}{Замечание}
\newtheorem{definition}{Определение}

% enumerating settings
\numberwithin{equation}{section}
\numberwithin{lemma}{section}
\numberwithin{definition}{section}
\numberwithin{notabene}{section}
\numberwithin{corollary}{section}

\begin{document}
	\section*{Интегральное представление решения уравнения Лапласа}
	Ранее была рассмотрена задача
	\begin{equation}
		\dfrac{\partial^2 u}{\partial x^2} +\dfrac{\partial^2 u}{\partial y^2} = 0
	\end{equation}
	В полуполосе $D = \{(x,y) \vert 0 < x < \pi, 0 < y\}$\newline
	В классе функций $u(x,y) \in C(\overline{D}) \cap C^1(\overline{D} \cap \{y > 0\}) \cap C^2 (D)$ \newline
	с граничными условиями
	\begin{equation}
		u(0, y) = 0, \ \dfrac{\partial u}{\partial x} (\pi, y) = 0
	\end{equation}
	\begin{equation}
		\lim\limits_{y \to 0 + 0} \int\limits_0^\pi \left[\dfrac{1}{k}\dfrac{\partial u}{\partial y} - \dfrac{\partial u}{\partial x} + \varphi(x) \right]^2 dx = 0, \ \varphi(x) \in L_2[0,\pi]
	\end{equation}
	\begin{equation}
		u(x,y) \rightrightarrows 0, y \to \infty
	\end{equation}
	И была доказана теорема:
	\begin{theorem}
		Решение задачи (2.1 - 2.4) существует и единственно, причём его можно представить в виде ряда
		\begin{equation}
			u(x,y) = \sum\limits_{n=0}^{\infty} A_n e^{-\left(n + \dfrac12\right)y} \sin{\left[\left(n + \dfrac12\right)x\right]},
		\end{equation}
		где коэффициенты $A_n, \ n =0,1,2, \dots$ находятся из разложения
		\begin{equation}
			\sum\limits_{n=0}^{\infty} A_n \left(n + \dfrac12 \right) \sin{\left[\left(n +\dfrac12\right)x + \dfrac\pi2 - \arctan{1/k} \right]} = \dfrac{k}{\sqrt{1+k^2}} \varphi(x)
		\end{equation}
	\end{theorem}
	\newpage
	Рассмотрим теперь интегральное представление решения этой задачи:
	\begin{theorem}
		Решение задачи (0.1)-(0.4) представимо в следующем виде:
		\begin{equation}
			u(x,y) = Im \ \dfrac{2e^{iz/2}}{\pi} \int\limits_0^\pi  \dfrac{\left(\sin{t/2}\right)^{1 - \gamma/\pi} \left(\cos{t/2}\right)^{\gamma/\pi}}{\left(1 - e^{i(z+t)} \right) \left(1 - e^{i(z-t)}\right)} \dfrac{\left(1 - e^{iz}\right)^{\gamma/\pi}}{\left(1 + e^{iz}\right)^{\gamma/\pi + \beta}} \Psi(t) dt
		\end{equation}	
	\end{theorem}	
	\begin{proof}
		Рассмотрим
		\begin{equation*}
			\sum\limits_{n=0}^{\infty} A_n \left(n + \dfrac12 \right) \sin{\left[\left(n +\dfrac12\right)x + \dfrac\pi2 - \arctan{1/k} \right]} = 
		\end{equation*}
		\begin{equation*}
			=\sum\limits_{n=0}^{\infty} A_n \left(n + \dfrac12 \right) \cos{\left[\left(n +\dfrac12\right)x - \arctan{1/k} \right]} = 
			 \dfrac{k}{\sqrt{1+k^2}} \varphi(x)
		\end{equation*}
		Проинтегрируем это равенство от 0 до x по x:
		\begin{equation*}
			\sum\limits_{n=0}^{\infty} A_n \sin{\left[\left(n +\dfrac12\right)x - \arctan{1/k} \right]} = \dfrac{1}{\sqrt{1+k^2}} \sum\limits_{n=0}^{\infty} A_n + \dfrac{k}{\sqrt{1+k^2}} \int\limits_0^x \varphi(\tau) d\tau = \Psi(x)
		\end{equation*}
		Система $\left\{\sin{\left[\left(n +\dfrac12\right)x - \arctan{1/k} \right]}		\right\}_{n=0}^{\infty}$ обзарует базис Рисса в $L_2(0,\pi)$ при $k\in (-\infty, 0) \cup (1, +\infty)$. Поэтому для коэффициентов $A_n$ справедливо разложение
		\begin{equation*}
				A_n = \int\limits_0^\pi h_{n+1}(t) \Psi(t) dt
		\end{equation*}
		\begin{equation*}
			u(x,y) = \sum\limits_{n=0}^{\infty} \int\limits_0^\pi e^{-\left(n + \dfrac12\right)y} \sin{\left[\left(n + \dfrac12\right)x\right]} h_{n+1}(t) \Psi(t) dt
		\end{equation*}
	Обозначим $z = x + iy$
		\begin{equation*}
			u(x,y) = Im \  \sum\limits_{n=0}^{\infty} \int\limits_0^\pi e^{i\left(n+\dfrac12\right)z} h_{n+1}(t) \Psi(t) dt
		\end{equation*}
	Заменим индекс $m = n + 1$
			\begin{equation*}
			u(x,y) = Im \ e^{-iz/2} \sum\limits_{m=1}^{\infty} \int\limits_0^\pi e^{imz} h_{m}(t) \Psi(t) dt
		\end{equation*}
	\begin{equation*}
		u(x,y) = Im \ e^{-iz/2} \int\limits_0^\pi \sum\limits_{m=1}^{\infty}  e^{imz} h_{m}(t) \Psi(t) dt
	\end{equation*}
	Введём новое обозначение:
	\begin{equation*}
		I(t,z) = \sum\limits_{m=1}^{\infty}  h_{m}(t)  e^{im z}
	\end{equation*}
	\begin{equation*}
		I(t,z) =\dfrac{2}{\pi}\dfrac{(2\cos{t/2})^\beta}{(\tan{t/2})^{\gamma/\pi}} \sum\limits_{n=1}^{\infty}   \sum\limits_{k=1}^n \sin{kt} B_{n-k} e^{inz} = 
		\dfrac{2}{\pi}\dfrac{(2\cos{t/2})^\beta}{(\tan{t/2})^{\gamma/\pi}} \sum\limits_{k=1}^{\infty} \sin{kt} \sum\limits_{n=k}^{\infty} e^{inz} B_{n-k}
	\end{equation*}
	Введём новый индекс $m = n - k$
	\begin{equation*}
		I(t,z) = \dfrac{2}{\pi}\dfrac{(2\cos{t/2})^\beta}{(\tan{t/2})^{\gamma/\pi}} \sum\limits_{k=1}^{\infty} \sin{kt} \sum\limits_{m =0 }^{\infty} e^{i(m+k)z} B_{m} = 
		\dfrac{2}{\pi}\dfrac{(2\cos{t/2})^\beta}{(\tan{t/2})^{\gamma/\pi}} \sum\limits_{k=1}^{\infty} e^{ikz}\sin{kt} \sum\limits_{m =0 }^{\infty} e^{imz} B_{m}
	\end{equation*}
	Первый ряд можем вычислить по формуле суммы бесконечно убывающей геометрической прогрессии
	\begin{equation*}
		\sum\limits_{k=1}^{\infty} e^{ikz}\sin{kt} =  \sum\limits_{k=1}^{\infty} e^{ikz}\dfrac{1}{2i}\left(e^{ikt} - e^{-ikt}\right) = \dfrac1{2i} \left(\dfrac{1}{1 - e^{i(z+t)}} -  \dfrac{1}{1 - e^{i(z-t)}}\right) = 
	\end{equation*}
	\begin{equation*}
		= \dfrac{1}{2i}  \dfrac{e^{i(z+t)} - e^{i(z-t)}}{\left(1 - e^{i(z+t)} \right) \left(1 - e^{i(z-t)}\right)} =  \dfrac{e^{iz} \sin{t}}{\left(1 - e^{i(z+t)} \right) \left(1 - e^{i(z-t)}\right)}
	\end{equation*}
	Рассмотрим второй ряд:
	\begin{equation*}
		\sum\limits_{l =0 }^{\infty} e^{ilz} B_{l} = \sum\limits_{l =0 }^{\infty} e^{ilz} \sum\limits_{m=0}^{l} C^{l - m}_{\gamma/\pi} C^{m}_{-\gamma/\pi - \beta} (-1)^{l-m} = \sum\limits_{m=0}^{\infty} \sum\limits_{l=m}^{\infty} e^{ilz} C^{l - m}_{\gamma/\pi} C^{m}_{-\gamma/\pi - \beta} (-1)^{l-m} = 
	\end{equation*}
	Введём новый индекс суммирования $k = l -m$
	\begin{equation*}
		\sum\limits_{m=0}^{\infty} \sum\limits_{k=0}^{\infty} e^{i(m+k)z} C^{k}_{\gamma/\pi} C^{m}_{-\gamma/\pi - \beta} (-1)^{k} = \sum\limits_{m=0}^{\infty} e^{imz} C^{m}_{-\gamma/\pi - \beta} \sum\limits_{k=0}^{\infty}  C^{k}_{\gamma/\pi} (-1)^k e^{ikz} = (1 + e^{iz})^{-\gamma/\pi - \beta} (1- e^{iz})^{\gamma/\pi} 
	\end{equation*}
	Собираем все решение:
	\begin{equation*}
		u(x,y) = Im\ e^{\dfrac{-iz}{2}} \int\limits_0^\pi I(t,z) \Psi(t) dt
	\end{equation*}
	\begin{equation*}
		u(x,y) = Im\ e^{\dfrac{-iz}{2}} \int\limits_0^\pi   \dfrac{2}{\pi}\dfrac{(2\cos{t/2})^\beta}{(\tan{t/2})^{\gamma/\pi}}\dfrac{e^{iz} \sin{t}}{\left(1 - e^{i(z+t)} \right) \left(1 - e^{i(z-t)}\right)} (1 + e^{iz})^{-\gamma/\pi - \beta} (1- e^{iz})^{\gamma/\pi}  \Psi(t) dt
	\end{equation*}
		В нашем случае $\beta = -1, \ \gamma = -2 \arctan{1/k}$. Оценим сходимость данного ряда: \newline Сперва рассмотрим следующий множитель
	\begin{equation*}
		\dfrac{(2\cos{t/2})^\beta  \sin{t}}{(\tan{t/2})^{\gamma/\pi}} = \dfrac{(\cos{t/2})^{\gamma/\pi}\sin{t}}{ 2\cos{t/2} * (\sin{t/2})^{\gamma/\pi}}
		 = \dfrac{(\cos{t/2})^{\gamma/\pi}\sin{t/2} \cos{t/2}}{ \cos{t/2} * (\sin{t/2})^{\gamma/\pi}} = 
	\end{equation*}
\begin{equation*}
	= \left(\sin{t/2}\right) ^{1 - \gamma/\pi}  (\cos{t/2})^{\gamma/\pi}
\end{equation*}
	 Рассмотрим показатели этих выражений: $\gamma = -2\arctan{1/k}$
	\end{proof}
\end{document}