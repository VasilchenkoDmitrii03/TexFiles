\documentclass[12pt, a4paper]{article}
\usepackage{amsthm,amsfonts,amsmath,amssymb,amscd}
\usepackage[T2A]{fontenc}                         
\usepackage[utf8]{inputenc}                      
\usepackage[english, russian]{babel}
\usepackage{graphicx}
\usepackage{indentfirst}
\usepackage{cite} 
\usepackage{psfrag}

\IfFileExists{pscyr.sty}{\usepackage{pscyr}}{}

\usepackage[top=2cm,bottom=2cm,left=3.5cm,right=1.5cm]{geometry}

\linespread{1.5}

\begin{document}
\begin{titlepage}
\begin{center}
\includegraphics[width=8cm, height=4cm]{msu}
\end{center}
\begin{center}
Московский государственный университет имени М.В. Ломоносова\\
\vspace{0.1 cm}
Факультет вычислительной математики и кибернетики\\
\vspace{0.1 cm}
Кафедра общей математики

\vspace{3cm}
{\Large Васильченко Дмитрий Дмитриевич }\\
\vspace{1cm}

{\bf\LARGE Тема выпускной квалификационной работы}\\ \vspace{2cm}
ВЫПУСКНАЯ КВАЛИФИКАЦИОННАЯ РАБОТА

\end{center}
\vspace{2cm}
\begin{flushright}

{\bf Научный руководитель:}\\
д.ф.-м.н., профессор\\ 
Н.\,Ю. Капустин

\end{flushright}

 \vspace{4.0cm}

\centerline {Москва, 2025}

\end{titlepage}
\setcounter{page}{2}
\tableofcontents
\newpage
\section{Введение}
Смешанные уравнения представляют собой класс уравнений, которые включают элементы как обыкновенных дифференциальных уравнений, так и уравнений с частными производными. Эти уравнения возникают в различных приложениях, где необходимо учитывать как временные, так и пространственные изменения системы. Например, они могут описывать процессы в биологии, такие как распространение веществ в тканях, или в физике, где необходимо учитывать как временные, так и пространственные изменения полей.

Исследование смешанных уравнений началось с работ таких ученых, как Трикоми, Франкль и Геллерстедта, которые заложили основы для дальнейшего развития этой области. В частности, Трикоми ввел понятие смешанного типа уравнений и разработал первые методы их решения. Впоследствии другие ученые, такие как Моисеев, продолжили развитие этой темы, предложив новые подходы и методы для решения смешанных уравнений.

Одной из основных трудностей при решении смешанных уравнений является необходимость учитывать специфические особенности этих уравнений, такие как наличие граничных условий и начальных данных. Это требует разработки специальных численных методов, которые могли бы эффективно справляться с этими особенностями. В настоящее время активно исследуются различные подходы к решению смешанных уравнений, включая методы конечных разностей, конечных элементов и спектральные методы.

Таким образом, исследование смешанных уравнений является важной и актуальной задачей, которая требует дальнейшего развития и совершенствования методов их решения. В данной работе будут рассмотрены основные подходы к решению смешанных уравнений и проанализированы их применения в различных областях науки и техники.
\newpage
\section{Акутальность}
В последние годы наблюдается значительный рост интереса к исследованию смешанного типа уравнений, которые сочетают в себе элементы как обыкновенных, так и дифференциальных уравнений с частными производными. Это связано с тем, что такие уравнения находят широкое применение в различных областях науки и техники, включая математическое моделирование, физику, биологию и инженерию. Смешанные уравнения позволяют более точно описывать сложные процессы, которые не могут быть полностью охвачены традиционными методами.

Одной из ключевых причин актуальности данной темы является необходимость разработки новых методов для решения смешанных уравнений, которые могли бы учитывать специфические особенности этих уравнений и обеспечивать высокую точность и эффективность вычислений. В настоящее время существует множество подходов к решению таких уравнений, однако каждый из них имеет свои ограничения и требует дальнейшего совершенствования.

Кроме того, смешанные уравнения часто возникают при моделировании различных процессов естествознания, например, при изучении дифференциальных операторов, которые описывают физические явления. Это делает данную тему особенно важной для дальнейшего развития науки и техники.
\section{Основные результаты}
\subsection{Постановка задачи}
	Рассматривается задача Трикоми для уравнения Лавертьева-Бицадзе
\begin{equation}
	\left(sgn (y)\right) \dfrac{\partial^2 u}{\partial x^2}(x,y) + \dfrac{\partial^2 u}{\partial y^2}(x,y) = 0
\end{equation}
в области $D = D^{+} \cup D^{-}$, где $D^{+} = \left\{(x,y): \ 0 < x < \pi, \ 0 < y < + \infty \right\}$, \newline
$D^{-} = \left\{(x,y): \ -y < x < y + \pi, \ -\pi/2 < y < 0\right\}$ в классе функций $u(x,y) \in C^2(D^{+}) \cap C^2(D^{-}) \cap C(\overline{D^{+} \cup D^{-}})$ с граничными условиями
\begin{equation}
	u(0,y) = 0, \ \dfrac{\partial u}{\partial x}(\pi, y) = 0, \ 0 < y < + \infty,
\end{equation}
\begin{equation}
	u(x,-x) = f(x) , \ 0 \leq x \leq \pi/2, \ f(0) = 0, 
\end{equation}
\begin{equation}
	u(x,y) \rightrightarrows 0, \ y \to +\infty
\end{equation}
и условием склеивания Франкля 
\begin{equation}
 \dfrac{\partial u}{\partial y}(x, +0) = \dfrac{\partial u}{\partial y} (x, -0), \ 0 < x < \pi.
\end{equation}
\subsection{Теорема единственности}
\textbf{Теорема 1.} \textit{Решение задачи (1) - (5) единственно.}

\textbf{Доказательство.} 
Пусть существуют два решения $u_1(x,y), u_2(x,y)$ задачи (1)-(5). Тогда $u(x,y) = u_1(x,y) - u_2(x,y)$ есть решение задачи (1)-(5) с функцией $f(x) \equiv 0$. В этом случае $u(x,y) = F(x+y) - F(0)$.

Отсюда следует, что равенство $\dfrac{\partial u}{\partial y} - \dfrac{\partial u}{\partial x} = 0$ выполняется для всех точек x и y из области гиперболичности. Используя условие склеивания  (5) будем иметь
\begin{equation}
	\dfrac{\partial u}{\partial y} - \dfrac{\partial u}{\partial x}\vert_{y=0+0} = 0.
\end{equation}
В результате получаем задачу для нахождения гармонической функции $u(x,y)$ в области $D^{+}$ с граничными условиями (2),(4),(6).

В силу принципа Зарембы-Жиро и равенства (6) экстремум не может достигаться на интервале $\{(x,y):\ 0 < x < \pi, \ y = 0\}$. На замкнутых боковых сторонах и на бесконечности экстремум не может достигаться в силу условий (2) и (4). Теорема доказана.

Известно, что общее решение в $D^{-}$  уравнения (1) имеет вид 
\begin{equation}
	u(x,y) = F(x+y) + f(\dfrac{x-y}{2}) - F(0).
\end{equation}
Продифференцируем равенство (7):
\begin{equation*}
	\dfrac{\partial u}{\partial y}(x,y) - \dfrac{\partial u}{\partial x}(x,y) \vert_{y=0+0} = -f'\left(\dfrac{x}2\right), \ 0 < x < \pi.
\end{equation*}
Используя условие склеивания (5), приходим к равенству
\begin{equation*}
	\dfrac{1}{k} \dfrac{\partial u}{\partial y}(x, 0+0) - \dfrac{\partial u}{\partial x}(x, 0 + 0) = - f'\left(\dfrac{x}2\right), \ 0 < x < \pi. 
\end{equation*}

\subsection{Постановка вспомогательной задачи}
	Получим в области $D^{+}$ вспомогательную задачу для оператора Лапласа 
\begin{equation}
	\dfrac{\partial^2 u}{\partial x^2}(x,y) + \dfrac{\partial^2 u}{\partial y^2}(x,y) = 0
\end{equation}
с граничными условиями 
\begin{equation}
	u(0,y) = 0, \ \dfrac{\partial u}{\partial x}(\pi, y) = 0, \ 0 < y < +\infty, 
\end{equation}
\begin{equation}
	\dfrac{\partial u}{\partial y}(x,0+0) - \dfrac{\partial u}{\partial x}(x,0+0) = -f'\left(\dfrac{x}{2}\right),
\end{equation}
\begin{equation}
	u(x,y) \rightrightarrows 0, \ y \to +\infty 
\end{equation}


\subsection{Существование и единственность решения вспомогательной задачи}

\textbf{Теорема 2.} \textit{	Решение задачи (8) - (11) существует, причём его можно представить в виде ряда
	\begin{equation}
		u(x,y) = \sum\limits_{n=0}^{\infty} A_n e^{-\left(n + \dfrac12\right)y} \sin{\left[\left(n + \dfrac12\right)x\right]},
	\end{equation}
	где коэффициенты $A_n, \ n =0,1,2, \dots$ находятся из разложения
	\begin{equation}
		\sum\limits_{n=0}^{\infty} A_n \left(n + \dfrac12 \right) \sin{\left[\left(n +\dfrac12\right)x + \dfrac\pi4\right]} = \dfrac{\varphi(x)}{\sqrt2}
\end{equation}}

\textbf{Доказательство.} 

	
	Докажем существование решения задачи $(1) - (4)$. В силу основного результата работы $[2]$ система $\left\{\sin\left[\left(n+\beta/2\right)x + \gamma/2\right]\right\}_{n=1}^{\infty}$ образует базис Рисса в пространствен $L_2(0,\pi)$, если $-\dfrac12 < \gamma/\pi < \dfrac32$ и $-\dfrac32 < \gamma/\pi + \beta < \dfrac12$ . В нашем случае $\{\sin{\left[\left(n + \dfrac12\right)x + \dfrac\pi4\right]}\}_{n=0}^{\infty}$ образует базис Рисса в пространстве $L_2(0, \pi)$ т.к. $\gamma = \pi/2$, $\beta=-1$. Поэтому коэффициенты разложения в формуле (6) удовлетворяют неравенствам Бесселя
	\begin{equation*}
		C_1 \|\varphi \|_{L_2(0,\pi)} \leq \sum\limits_{n=0}^{\infty} A_n^2 \left(n + \dfrac12\right)^2 \leq C_2 \|\varphi \|_{L_2(0,\pi)} , 0 < C_1 < C_2, 
	\end{equation*}
	а значит сходится ряд $\sum\limits_{n=0}^{\infty} |A_n|$ и сходится равномерно ряд (5). То, что функция (5) при $y > 0$ - решение уравнения (1), удовлетворяющее условиям (2) - это очевидно. В силу равенства $\sum\limits_{n=0}^{\infty} e^{-\left(n + \frac12\right)y} = \dfrac{e^{-y/2}}{1 - e^{-y}}$, также очевидно, что выполнено условие (4). Проверим выполнение условия (3).\newline
	Выразим функцию $\varphi(x)$ из представления (6) и подставим в условие (3)
	\begin{equation*}
		I(y) =  2 \int\limits_0^\pi \left[	\sum\limits_{n=0}^{\infty} A_n\left(n+\dfrac12\right) \left( e^{-\left(n+\dfrac12\right)y} - 1\right) \sin{\left[\left(n+\dfrac12\right) x  + \dfrac\pi4\right]} \right]^2 dx
	\end{equation*}
	Докажем, что $I(y) \to 0$ при $y \to 0+0$. 
	\begin{equation*}
		I(y) \leq 4\int\limits_0^\pi \left[	\sum\limits_{n=0}^{m} A_n\left(n+\dfrac12\right) \left( e^{-\left(n+\dfrac12\right)y} - 1\right) \sin{\left[\left(n+\dfrac12\right) x  + \dfrac\pi4\right]} \right]^2 dx + 
	\end{equation*}
	\begin{equation*}
		+ 4\int\limits_0^\pi \left[	\sum\limits_{n=m+1}^{\infty} A_n\left(n+\dfrac12\right) \left( e^{-\left(n+\dfrac12\right)y} - 1\right) \sin{\left[\left(n+\dfrac12\right) x  + \dfrac\pi4\right]} \right]^2 dx
	\end{equation*}
	В силу левой части неравенства Бесселя имеем оценку
	\begin{equation*}
		\int\limits_0^\pi \left[	\sum\limits_{n=m+1}^{\infty} A_n\left(n+\dfrac12\right) \left( e^{-\left(n+\dfrac12\right)y} - 1\right) \sin{\left[\left(n+\dfrac12\right) x  + \dfrac\pi4\right]} \right]^2 dx \leq 
	\end{equation*}
	\begin{equation*}
		\leq  C_3 \sum\limits_{n=m+1}^{\infty} A_n^2 \left(n+\dfrac12\right)^2 \left(e^{-\left(n+\dfrac12\right)y} - 1\right)^2 \leq C_3 \sum\limits_{n=m+1}^{\infty} A_n^2 \left(n+\dfrac12\right)^2 < \dfrac{\varepsilon}{2}
	\end{equation*}
	Это верно $\forall \varepsilon > 0$, если $m \geq N =N(\varepsilon)$\newline
	Во втором слагаемом мы имеем дело с конечным числом элементов, поэтому:
	\begin{equation*}
		\int\limits_0^\pi \left[	\sum\limits_{n=0}^{m} A_n\left(n+\dfrac12\right) \left( e^{-\left(n+\dfrac12\right)y} - 1\right) \sin{\left[\left(n+\dfrac12\right) x  + \dfrac\pi4\right]} \right]^2 dx \leq
	\end{equation*}
	\begin{equation*}
		\leq C_4 \sum\limits_{n=0}^{m} A_n^2 \left(n +\dfrac12\right)^2 \left(e^{-\left(n+\dfrac12\right)y} - 1\right)^2 < \dfrac{\varepsilon}{2}
	\end{equation*}
	Это верно, если $0 < y < \delta$ (m зафиксировано в зависимости от N). Условие (3) выполнено. Теорема доказана.

\textbf{Теорема 3.} \textit{Решение задачи (8) - (11) единственно.}

\textbf{Доказательство.} 
	Докажем единственность решения этой задачи. Пусть $u(x,y)$ - разность двух решений - решение задачи с $\varphi(x) \equiv 0$. Необходимо получить выражение, где слева будет входить модуль или чётная степень функции u, а справа будет 0.\newline
	Введём обозначения $A_\varepsilon = (0, \varepsilon), A_R = (0, R), B_R = (\pi, R), B_\varepsilon = (\pi, \varepsilon)$. $D_{R\varepsilon}$ - прямоугольник $A_\varepsilon A_R B_R B_\varepsilon$. Справедливы следующие соотношения:
	\begin{equation*}
		0 = \iint\limits_{D_{R\varepsilon}} (R-y) (u_{xx} + u_{yy}) dx dy = I
	\end{equation*}
	Заметим, что 
	\begin{equation*}
		(R - y) (u_{xx} + u_{yy}) u = \left( \left(R - y\right) u_x u\right)_x  + \left( \left(R - y\right) u_y u\right)_y - \left(R- y\right) \left(u_x^2 + u_y^2\right) +  u_y u = 
	\end{equation*}
	\begin{equation*}
		= \left(R-y\right) \left(u_{xx} u + u_x^2\right) + \left(-u_y + \left(R-y\right) u_{yy} u + \left(R-y\right)u_y^2\right) - \left(R- y\right) \left(u_x^2 + u_y^2\right)+  u_y u
	\end{equation*}
	\newline Подставим это выражение в интеграл
	\begin{equation*}
		I	=	\iint\limits_{D_{R\varepsilon}} \left( \left(R - y\right) u_x u\right)_x dx dy  + \iint\limits_{D_{R\varepsilon}} \left( \left(R - y\right) u_y u\right)_y dx dy   
		- \iint\limits_{D_{R\varepsilon}} \left(R- y\right) \left(u_x^2 + u_y^2\right) + \iint\limits_{D_{R\varepsilon}} u_y u dx dy
	\end{equation*}
	Упростим теперь эти интегралы:\newline
	\begin{equation*}
		\iint\limits_{D_{R\varepsilon}} \left( \left(R - y\right) u_x u\right)_x dx dy = \int\limits_{[\varepsilon, R]} \left[(R-y)u_xu\right] \vert_0^\pi dy =  
	\end{equation*}
	\begin{equation*}
	 = \int\limits_{[\varepsilon, R]} \left[(R-y) u_x(\pi, y)u(\pi,y) - (R-y) u_x(0, y)u(0,y)\right]dy = 0
	\end{equation*}
	т.к. оба подынтегральных выражения равны нулю в силу условия (2), поэтому \newline
	\begin{equation*}
		\iint\limits_{D_{R\varepsilon}} \left( \left(R - y\right) u_y u\right)_y dx dy = \int\limits_{[0,\pi]} \left[\left(R - y\right) u_y u\right] \vert_\varepsilon^R dx =
		\int\limits_{[0,\pi]} \left[0 - \left(R - \varepsilon \right) u_y(x, \varepsilon) u(x, \varepsilon) \right] dx = 
	\end{equation*} 
	\begin{equation*}
		= - \int\limits_{A_\varepsilon B_\varepsilon} \left(R - \varepsilon \right) u_yu dx, 
	\end{equation*}
	\begin{equation*}
		\iint\limits_{D_{R\varepsilon}} u_y u dx dy = \iint\limits_{D_{R\varepsilon}} \left(\dfrac{u^2}{2}\right)'_ydx dy = \int\limits_{[0,\pi]} \left[\dfrac{u^2(x,R)}{2} - \dfrac{u^2(x, \varepsilon)}{2}\right] dx 
	\end{equation*}
	\newline
	В итоге получим
	\begin{equation*}
		= - \iint\limits_{D_{R\varepsilon}} \left(R - y\right) \left(u_x^2 + u_y^2\right) dx dy
		- \int\limits_{A_\varepsilon B_\varepsilon} \left(R - \varepsilon\right) u_y u dx 
		-\int\limits_{A_\varepsilon B_\varepsilon} \dfrac{u^2}{2} dx + \int\limits_{A_R B_R} \dfrac{u^2}{2} dx = 
	\end{equation*}
	Добавим и вычтем $\int\limits_{A_\varepsilon B_\varepsilon} \left(R - \varepsilon\right) u_x u dx$, тогда
	\begin{equation*}
		= - \iint\limits_{D_{R\varepsilon}} \left(R - y\right) \left(u_x^2 + u_y^2\right) dx dy - 
		\int\limits_{A_\varepsilon B_\varepsilon} \left(R - \varepsilon \right) \left(u_y - u_x\right)u dx - \int\limits_{A_\varepsilon B_\varepsilon} \left(R - \varepsilon\right) u_x u dx - \int\limits_{A_\varepsilon B_\varepsilon}\dfrac{u^2}{2} dx +
	\end{equation*}
	\begin{equation*}
		+ \int\limits_{A_R B_R} \dfrac{u^2}{2}dx
	\end{equation*}
	Отсюда следует
	\begin{equation*}
		\iint\limits_{D_{R\varepsilon}} \left(R - y\right) \left(u_x^2 + u_y^2\right) dx dy + \dfrac{1}{2}\int\limits_{A_\varepsilon B_\varepsilon} u^2 dx +\dfrac{R - \varepsilon}{2}u^2(\pi, \varepsilon)  =
	\end{equation*}
	\begin{equation*}
		= \int\limits_{A_\varepsilon B_\varepsilon} \left(R - \varepsilon \right) \left(u_x - u_y\right)u dx + \dfrac12  \int\limits_{A_R B_R} u^2 dx \leq \{ \text{Неравенство Коши-Буняковского} \}
	\end{equation*}
	\begin{equation*}
		\leq \left(R - \varepsilon\right) \left[\int\limits_{A_\varepsilon B_\varepsilon} \left( u_y - u_x\right)^2 dx \right]^{\frac12} \left[\int\limits_{A_\varepsilon B_\varepsilon} u^2 dx \right]^{\frac12} + \dfrac12 \int\limits_{A_RB_R} u^2 dx = I
	\end{equation*}
	Рассмотрим следующее неравенство: \begin{math}
		\left(2ar -b\right)^2 \geq 0 \Rightarrow 4a^2 r^2 - 4 abr + b^2 \geq 0 \Rightarrow ab \leq r a^2 + \frac{b}{4r}
	\end{math}
	\newline
	Возьмём $ a = \left[ \left(R - \varepsilon \right) \int\limits_{A_\varepsilon B_\varepsilon} \left( u_y - u_x\right)^2 dx \right]^{\frac12}$, $ b = \left[\left(R - \varepsilon\right)\int\limits_{A_\varepsilon B_\varepsilon} u^2 dx \right]^{\frac12}$, $ r = R - \varepsilon$, тогда
	
	\begin{equation*}
		I \leq \left(R - \varepsilon\right)^2 \int\limits_{A_\varepsilon B_\varepsilon} \left( u_y - u_x\right)^2 dx + \dfrac14 \int\limits_{A_\varepsilon B_\varepsilon} u^2 dx +\dfrac12 \int\limits_{A_RB_R} u^2 dx, 
	\end{equation*}
	\begin{equation*}
		\iint\limits_{D_{R\varepsilon}} \left(R - y\right) \left(u_x^2 + u_y^2\right) dx dy + \dfrac{1}{4}\int\limits_{A_\varepsilon B_\varepsilon} u^2 dx +\dfrac{R - \varepsilon}{2}u^2(\pi, \varepsilon) \leq 
	\end{equation*}
	\begin{equation*}
		\leq \left(R - \varepsilon\right)^2 \int\limits_{A_\varepsilon B_\varepsilon} \left( u_y - u_x\right)^2 dx  +\dfrac12 \int\limits_{A_RB_R} u^2 dx
	\end{equation*}
	Устремим $\varepsilon \to 0 + 0$, тогда в силу условия (3)
	\begin{equation*}
		\lim\limits_{\varepsilon \to 0 + 0} \int\limits_{A_\varepsilon B_\varepsilon} \left(u_y - u_x\right)^2 dx = 0
	\end{equation*}
	и получим соотношение
	\begin{equation*}
		\lim\limits_{\varepsilon \to 0 + 0} \iint\limits_{D_{R\varepsilon}} \left(R - y\right) \left(u_x^2 + u_y^2 \right) dx dy + \dfrac14 \int\limits_0^\pi u^2(x,0) dx + \dfrac{R}{2}u^2(\pi,0) \leq \dfrac12 \int\limits_{A_RB_R} u^2 dx
	\end{equation*}
	Устремим теперь $R \to \infty$, тогда  в силу условия (4) $\int\limits_{A_RB_R} u^2 dx \to 0$, тем самым, это возможно только в случае $u(x,y) \equiv 0$ в $\overline{D}$.
\subsection{Интегральное представление первых частных производных решения вспомогательной задачи}
\textbf{Теорема 4.} \textit{	Пусть $u(x,y)$ - решение задачи $(8)-(11)$, тогда $u_x, u_y$ представимы в виде
	\begin{equation}
		u_y(x,y) = - Im\  \dfrac{ \sqrt{1 - e^{i2z}} }{\pi} e^{\dfrac{+iz}{2}} \int\limits_0^\pi  \dfrac{\sqrt{\sin{t}}}{\left(1 - e^{i(z+t)} \right) \left(1 - e^{i(z-t)}\right)}  \varphi(t) dt
	\end{equation}
	\begin{equation}
		u_x(x,y) = Re\   \dfrac{ \sqrt{1 - e^{i2z}} }{\pi} e^{\dfrac{+iz}{2}} \int\limits_0^\pi  \dfrac{\sqrt{\sin{t}}}{\left(1 - e^{i(z+t)} \right) \left(1 - e^{i(z-t)}\right)}  \varphi(t) dt
\end{equation}}

\textbf{Доказательство.} 
	\newline
	Рассмотрим уравнение (6). Система синусов $\sin{\left[\left(n +\dfrac12\right)x + \dfrac\pi4\right]}$ образует базис в $L_2(0,\pi)$. Поэтому для коэффициентов $A_n\left(n+\dfrac12\right)$ справедливо следующее представление:
	\begin{equation*}
		A_n\left(n+\dfrac12\right) = \int\limits_0^\pi h_{n+1}(t) \dfrac{\varphi(t)}{\sqrt2} dt, 
	\end{equation*}
	где
	\begin{equation*}
		h_n(t) = \dfrac{2}{\pi}\dfrac{(2\cos{t/2})^\beta}{(\tan{t/2})^{\gamma/\pi}} \sum\limits_{k=1}^n \sin{kt} B_{n-k}
	\end{equation*}
	Пусть $u(x,y)$ - решение задачи (1)-(4), тогда
	\begin{equation*}
		u(x,y) = \sum\limits_{n=0}^{\infty} A_n e^{-\left(n + \dfrac12\right)y} \sin{\left[\left(n + \dfrac12\right)x\right]}
	\end{equation*}
	и соотвественно
	\begin{equation*}
		u_y(x,y) = -\sum\limits_{n=0}^{\infty} A_n \left(n +\dfrac12\right) e^{-\left(n + \dfrac12\right)y} \sin{\left[\left(n + \dfrac12\right)x\right]}
	\end{equation*}
	Здесь как раз возникает нужный нам коэффициент $A_n \left(n+\dfrac12\right)$, поэтому
	\begin{equation*}
		u_y(x,y)  = - \sum\limits_{n=0}^{\infty}  \int\limits_0^\pi \dfrac{\varphi(t)}{\sqrt2}  h_{n+1}(t)  e^{-\left(n + \dfrac12\right)y} \sin{\left[\left(n + \dfrac12\right)x\right]} dt
	\end{equation*}
	$\sin{\left[\left(n + \dfrac12\right)x\right]} = Im \ e^{i\left(n + \dfrac12\right)x}$, поэтому
	\begin{equation*}
		u_y(x,y)  = -  Im \ \sum\limits_{n=0}^{\infty}  \int\limits_0^\pi \dfrac{\varphi(t)}{\sqrt2}  h_{n+1}(t)  e^{-\left(n + \dfrac12\right)y} e^{i\left(n + \dfrac12\right)x} dt
	\end{equation*}
	Обозначим $z = x + iy$
	\begin{equation*}
		u_y(x,y)  = -  Im \ \sum\limits_{n=0}^{\infty}  \int\limits_0^\pi \dfrac{\varphi(t)}{\sqrt2}  h_{n+1}(t)  e^{i\left(n+\dfrac12\right) z}  dt
	\end{equation*}
	Для дальнейших операций нам было бы удобно, чтобы суммирование начинолось от 1, а не 0, поэтому сделаем замену $m = n +1$
	\begin{equation*}
		u_y(x,y)  = -  Im \ \sum\limits_{m=1}^{\infty}  \int\limits_0^\pi \dfrac{\varphi(t)}{\sqrt2}  h_{m}(t)  e^{i\left(m-\dfrac12\right) z}  dt
	\end{equation*}
	\begin{equation*}
		u_y(x,y)  = -  Im \ e^{-\dfrac{iz}{2}}\ \sum\limits_{m=1}^{\infty}  \int\limits_0^\pi \dfrac{\varphi(t)}{\sqrt2}  h_{m}(t)  e^{im z}  dt
	\end{equation*}
	Поменяем местами знаки интергирования и суммирования
	\begin{equation*}
		u_y(x,y)  = -  Im \ e^{-\dfrac{iz}{2}}\  \int\limits_0^\pi \dfrac{\varphi(t)}{\sqrt2}  \sum\limits_{m=1}^{\infty}   h_{m}(t)  e^{im z}  dt
	\end{equation*}
	Введём новое обозначение:
	\begin{equation*}
		I(t,z) = \sum\limits_{m=1}^{\infty}  h_{m}(t)  e^{im z}
	\end{equation*}
	\begin{equation*}
		I(t,z) =\dfrac{2}{\pi}\dfrac{(2\cos{t/2})^\beta}{(\tan{t/2})^{\gamma/\pi}} \sum\limits_{n=1}^{\infty}   \sum\limits_{k=1}^n \sin{kt} B_{n-k} e^{inz} = 
		\dfrac{2}{\pi}\dfrac{(2\cos{t/2})^\beta}{(\tan{t/2})^{\gamma/\pi}} \sum\limits_{k=1}^{\infty} \sin{kt} \sum\limits_{n=k}^{\infty} e^{inz} B_{n-k}
	\end{equation*}
	Введём новый индекс $m = n - k$
	\begin{equation*}
		I(t,z) = \dfrac{2}{\pi}\dfrac{(2\cos{t/2})^\beta}{(\tan{t/2})^{\gamma/\pi}} \sum\limits_{k=1}^{\infty} \sin{kt} \sum\limits_{m =0 }^{\infty} e^{i(m+k)z} B_{m} = 
		\dfrac{2}{\pi}\dfrac{(2\cos{t/2})^\beta}{(\tan{t/2})^{\gamma/\pi}} \sum\limits_{k=1}^{\infty} e^{ikz}\sin{kt} \sum\limits_{m =0 }^{\infty} e^{imz} B_{m}
	\end{equation*}
	Первый ряд можем вычислить по формуле суммы бесконечно убывающей геометрической прогрессии
	\begin{equation*}
		\sum\limits_{k=1}^{\infty} e^{ikz}\sin{kt} =  \sum\limits_{k=1}^{\infty} e^{ikz}\dfrac{1}{2i}\left(e^{ikt} - e^{-ikt}\right) = \dfrac1{2i} \left(\dfrac{1}{1 - e^{i(z+t)}} -  \dfrac{1}{1 - e^{i(z-t)}}\right) = 
	\end{equation*}
	\begin{equation*}
		= \dfrac{1}{2i}  \dfrac{e^{i(z+t)} - e^{i(z-t)}}{\left(1 - e^{i(z+t)} \right) \left(1 - e^{i(z-t)}\right)} =  \dfrac{e^{iz} \sin{t}}{\left(1 - e^{i(z+t)} \right) \left(1 - e^{i(z-t)}\right)}
	\end{equation*}
	Рассмотрим второй ряд:
	\begin{equation*}
		\sum\limits_{l =0 }^{\infty} e^{ilz} B_{l} = \sum\limits_{l =0 }^{\infty} e^{ilz} \sum\limits_{m=0}^{l} C^{l - m}_{\gamma/\pi} C^{m}_{-\gamma/\pi - \beta} (-1)^{l-m} = \sum\limits_{m=0}^{\infty} \sum\limits_{l=m}^{\infty} e^{ilz} C^{l - m}_{\gamma/\pi} C^{m}_{-\gamma/\pi - \beta} (-1)^{l-m} = 
	\end{equation*}
	Введём новый индекс суммирования $k = l -m$
	\begin{equation*}
		\sum\limits_{m=0}^{\infty} \sum\limits_{k=0}^{\infty} e^{i(m+k)z} C^{k}_{\gamma/\pi} C^{m}_{-\gamma/\pi - \beta} (-1)^{k} = \sum\limits_{m=0}^{\infty} e^{imz} C^{m}_{-\gamma/\pi - \beta} \sum\limits_{k=0}^{\infty}  C^{k}_{\gamma/\pi} (-1)^k e^{ikz} = 
	\end{equation*}
	\begin{equation*}
		= (1 + e^{iz})^{-\gamma/\pi - \beta} (1- e^{iz})^{\gamma/\pi} 
	\end{equation*}
	В нашем случае $\beta = -1, \gamma = \pi/2$, поэтому
	\begin{equation*}
		= (1 + e^{iz})^{1/2} (1- e^{iz})^{1/2} =\sqrt{1 - e^{i2z}} 
	\end{equation*}
	Собираем все решение:
	\begin{equation*}
		u_y(x,y) = - Im\ e^{\dfrac{-iz}{2}} \int\limits_0^\pi \dfrac{\varphi(t)}{\sqrt2} I(t,z) dt 
	\end{equation*}
	\begin{equation*}
		u_y(x,y) = - Im\ e^{\dfrac{-iz}{2}} \int\limits_0^\pi \dfrac{2}{\pi}\dfrac{(2\cos{t/2})^\beta}{(\tan{t/2})^{\gamma/\pi}}  \dfrac{e^{iz} \sin{t}}{\left(1 - e^{i(z+t)} \right) \left(1 - e^{i(z-t)}\right)} \sqrt{1 - e^{i2z}} \dfrac{\varphi(t)}{\sqrt2} dt
	\end{equation*}
	Подставляя $\beta$ и $\gamma$ получим
	\begin{equation*}
		u_y(x,y) = - Im\  \dfrac{2}{\pi} e^{\dfrac{-iz}{2}} \int\limits_0^\pi \dfrac{1}{2\cos{t/2} \sqrt{\tan{t/2}}}  \dfrac{e^{iz} \sin{t}}{\left(1 - e^{i(z+t)} \right) \left(1 - e^{i(z-t)}\right)} \sqrt{1 - e^{i2z}} \dfrac{\varphi(t)}{\sqrt2} dt
	\end{equation*}
	\begin{equation*}
		u_y(x,y) = - Im\  \dfrac{e^{\dfrac{+iz}{2}}}{\pi}  \int\limits_0^\pi  \dfrac{\sqrt{\sin{t}} \sqrt{1 - e^{i2z}}}{\left(1 - e^{i(z+t)} \right) \left(1 - e^{i(z-t)}\right)}  \varphi(t) dt
	\end{equation*}
	Теорема доказана.


\newpage
\begin{thebibliography}{30}

\bibitem{Watson}
{\it Ватсон Дж. Н.} Теория бесселевых функций. М.: Издательство иностранной литературы, 1949. -- 798 с.
\bibitem{Keldysh}
{\it Келдыш М. В.} О некоторых случаях вырожденных уравнений эллиптического типа на границе области // Доклады АН СССР. 1951. Т. 77, \textnumero 2. С. 181-183.

\end{thebibliography}
\end{document} 