\documentclass[10pt,pdf,hyperref={unicode}]{beamer}
\usetheme[]{Szeged}
\usecolortheme{beaver}
\setbeamercolor*{palette secondary}{bg=white}
\setbeamercolor*{palette tertiary}{bg=blue!20}
\setbeamercolor*{palette quaternary}{bg=blue!20}
\setbeamercolor{alerted text}{use=structure,fg=black}
\setbeamercolor{frametitle}{bg=white,fg=black}
\setbeamercolor{titlelike}{parent=structure, fg=black}
\setbeamertemplate{itemize item}{\color{black}$\bullet$}

\usepackage{tikz}
\newcommand{\tikzcircle}[2][red,fill=red]{\tikz[baseline=-0.5ex]\draw[#1,radius=#2] (0,0) circle ;}%
\usetikzlibrary{shapes.geometric, arrows.meta}
\usetikzlibrary{positioning}
\usetikzlibrary{calc}
\usepackage{tikz}
\usepackage[T1,T2A]{fontenc}
\usepackage[utf8]{inputenc}
\usepackage[english,russian]{babel}
\usepackage{amssymb,amsmath}
\usepackage{nicematrix}
\usefonttheme{professionalfonts}
\usefonttheme[onlymath]{serif} 
\usepackage{helvet}
\institute{МГУ им. М.В.Ломоносова}
\title[]{Корректная реализация алгоритма регуляризующего разложения для рациональных матриц на языке MATLAB}
\author[Васильченко  Д.Д.]{Васильченко Дмитрий Дмитриевич, 409 группа}



\date{2024}
\beamertemplatenavigationsymbolsempty
\setbeamertemplate{footline}{}

\usepackage{wasysym}
\usepackage{booktabs}
\usepackage{amsthm}


\providecommand\rightarrowRHD{\relbar\joinrel\mathrel\RHD}
\providecommand\rightarrowrhd{\relbar\joinrel\mathrel\rhd}
\providecommand\longrightarrowRHD{\relbar\joinrel\relbar\joinrel\mathrel\RHD}
\providecommand\longrightarrowrhd{\relbar\joinrel\relbar\joinrel\mathrel\rhd}

\makeatletter
\providecommand*\xrightarrowRHD[2][]{\ext@arrow 0055{\arrowfill@\relbar\relbar\longrightarrowRHD}{#1}{#2}}
\providecommand*\xrightarrowrhd[2][]{\ext@arrow 0055{\arrowfill@\relbar\relbar\longrightarrowrhd}{#1}{#2}}
\makeatother

\makeatletter
\newcommand\xleftrightarrow[2][]{%
  \ext@arrow 9999{\longleftrightarrowfill@}{#1}{#2}}
\newcommand\longleftrightarrowfill@{%
  \arrowfill@\leftarrow\relbar\rightarrow}
\makeatother

\begin{document}
\tikzstyle{decision} = [diamond, aspect=2, draw, text centered, minimum height=2em]
\tikzstyle{process} = [rectangle, draw, text centered, minimum height=2em]
\tikzstyle{data}=[trapezium, draw, text centered, trapezium left angle=60, trapezium right angle=120, minimum height=2em]
\tikzstyle{terminator} = [rectangle, draw, text centered, rounded corners, minimum height=2em]
\tikzstyle{arrow} = [thick,->,>=stealth]
\begin{frame}
\begin{center}%
Московский государственный университет имени М.\,В.~Ломоносова
    
    Факультет вычислительной математики и кибернетики
    
    Кафедра общей математики
  \end{center}%
  \vfill%
  \begin{center}%
    \large%
    Васильченко Дмитрий Дмитриевич
    
    \vspace{.5cm}
    
    \Large
    \textbf{О задаче Трикоми для уравнения Лаврентьева-Бицадзе с условием Франкля на линии склеивания типа.}
  \end{center}%
  \vfill
  \vfill%
  \vfill%
  \begin{flushright}%
    \begin{tabular}{@{}l@{}}
      \textbf{Научный руководитель:} \\
      \qquad
      \begin{tabular}{@{}l@{}}
        д.ф.-м.н., профессор \\
        Капустин~Н.\,Ю.
      \end{tabular} \\
    \end{tabular}
  \end{flushright}%
  \vfill%
  \vfill%
  \begin{center}%
    Москва, 2025%
  \end{center}%
\end{frame}

\section{Введение}
\begin{frame}{Постановка задачи}
		Рассматривается задача Трикоми для уравнения Лавертьева-Бицадзе
	\begin{equation}
		\left(sgn (y)\right) \dfrac{\partial^2 u}{\partial x^2}(x,y) + \dfrac{\partial^2 u}{\partial y^2}(x,y) = 0
	\end{equation}
	в области $D = D^{+} \cup D^{-}$, где $D^{+} = \left\{(x,y): \ 0 < x < \pi, \ 0 < y < + \infty \right\}$, \newline
	$D^{-} = \left\{(x,y): \ -y < x < y + \pi, \ -\pi/2 < y < 0\right\}$ в классе функций $u(x,y) \in C^2(D^{+}) \cap C^2(D^{-}) \cap C(\overline{D^{+} \cup D^{-}})$ с граничными условиями
	\begin{equation}
		u(0,y) = 0, \ \dfrac{\partial u}{\partial x}(\pi, y) = 0, \ 0 < y < + \infty,
	\end{equation}
	\begin{equation}
		u(x,-x) = f(x) , \ 0 \leq x \leq \pi/2, \ f(0) = 0, 
	\end{equation}
	\begin{equation}
		u(x,y) \rightrightarrows 0, \ y \to +\infty
	\end{equation}
	и условием склеивания Франкля 
	\begin{equation}
		\dfrac{\partial u}{\partial y}(x, +0) = \dfrac{\partial u}{\partial y} (x, -0), \ 0 < x < \pi.
	\end{equation}
\end{frame}
\begin{frame}
	\begin{tikzpicture}
		% Оси
		\draw[->] (-1,0) -- (8,0) node[right] {$x$};
		\draw[->] (0,-3.2) -- (0,5) node[above] {$y$};
		
		% Полуполоса
		\fill[gray!20] (0,0) -- (0,5) -- (6.28,5) -- (6.28,0) -- cycle;
		\fill[blue!20] (0,0) -- (3.14,-3.14) -- (6.28,0) -- cycle;
		% Подписи
		\node[rotate=90] at (-0.5,2) {$u(0, y) = 0$};
		\node[rotate=90] at (6.9,2) {$\dfrac{\partial u}{\partial x}(\pi, y) = 0$};
		\node at (-0.2, -0.3) {$0$};
		\node at (6.28, -0.3) {$\pi$};
		
		% Линии границ
		\draw[thick] (0,0) -- (0,5);
		\draw[thick] (6.28,0) -- (6.28,5);
		
		% Равнобедренный треугольник под полуполосой
		\draw[thick] (0,0) -- (3.14,-3.14) -- (6.28,0) -- cycle;
		
		% Пунктирная линия
		\draw[dashed, thick] (0,-3.14) -- (3.14,-3.14);
		\node at (0.8, -2.9) {$y= -\pi/2$};
		
		\node[rotate=315] at (1.2,-1.6) {$u(x,-x) = f(x)$};
		\node at (3.14, 5.2) {$u(x,y) \rightrightarrows 0$};
		\node at (3.14, 0.45) {$\dfrac{\partial u}{\partial y}(x, +0)$};
		\node[rotate=90, font=\large] at (3.14, 0) {$=$};
		\node at (3.14, -0.45) {$\dfrac{\partial u}{\partial y} (x, -0)$};
		
		\node[font=\LARGE] at(3.14,2.5) {$\Delta u = 0$};
		\node[font=\LARGE] at(3.14,-1.5) {$u_{xx} = u_{yy}$};
	\end{tikzpicture}
	
	
\end{frame}
\section{Общее решение}
\begin{frame}{Основные результаты}
	\textbf{Теорема 1.} \textit{Решение задачи (1) - (5) единственно.}
	\newline
	\newline
	Используя известную формулу общего решения задачи (1) - (5) в области $D^-$ 
	\begin{equation*}
			u(x,y) = F(x+y) + f(\dfrac{x-y}{2}) - F(0).
	\end{equation*}
	Продифференцируем это равенство и получим следующее соотношение
	\begin{equation*}
		\dfrac{1}{k} \dfrac{\partial u}{\partial y}(x, 0+0) - \dfrac{\partial u}{\partial x}(x, 0 + 0) = - f'\left(\dfrac{x}2\right), \ 0 < x < \pi. 
	\end{equation*}
\end{frame}
\section{Вспомогательная задача}
\begin{frame}{Постановка вспомогательной задачи в $D^+$}
	Получим в области $D^{+}$ вспомогательную задачу для оператора Лапласа 
	\begin{equation}
		\dfrac{\partial^2 u}{\partial x^2}(x,y) + \dfrac{\partial^2 u}{\partial y^2}(x,y) = 0
	\end{equation}
	с граничными условиями 
	\begin{equation}
		u(0,y) = 0, \ \dfrac{\partial u}{\partial x}(\pi, y) = 0, \ 0 < y < +\infty, 
	\end{equation}
	\begin{equation}
		\dfrac{\partial u}{\partial y}(x,0+0) - \dfrac{\partial u}{\partial x}(x,0+0) = -f'\left(\dfrac{x}{2}\right),
	\end{equation}
	\begin{equation}
		u(x,y) \rightrightarrows 0, \ y \to +\infty 
	\end{equation}
	
\end{frame}
\begin{frame}
	\begin{tikzpicture}
		% Оси
		\draw[->] (-1,0) -- (8,0) node[right] {$x$};
		\draw[->] (0,-1) -- (0,5) node[above] {$y$};
		
		% Полуполоса
		\fill[gray!20] (0,0) -- (0,5) -- (6.28,5) -- (6.28,0) -- cycle;
		% Подписи
		\node[rotate=90] at (-0.5,2) {$u(0, y) = 0$};
		\node[rotate=90] at (6.9,2) {$\dfrac{\partial u}{\partial x}(\pi, y) = 0$};
		\node at (-0.2, -0.3) {$0$};
		\node at (6.45, -0.3) {$\pi$};
		
		% Линии границ
		\draw[thick] (0,0) -- (0,5);
		\draw[thick] (6.28,0) -- (6.28,5);
		
		\node at (3.14, 5.2) {$u(x,y) \rightrightarrows 0$};
		\node at (3.14, -0.55) {$\dfrac{\partial u}{\partial y}(x,0+0) - \dfrac{\partial u}{\partial x}(x,0+0) = -f'\left(\dfrac{x}{2}\right)$};
		
		\node[font=\LARGE] at(3.14,2.5) {$\Delta u = 0$};
	\end{tikzpicture}
	
	
\end{frame}
\begin{frame}{Существование и единственность решения вспомогательной задачи}
	\textbf{Теорема 2.} \textit{	Решение задачи (8) - (11) существует, причём его можно представить в виде ряда
		\begin{equation}
			u(x,y) = \sum\limits_{n=0}^{\infty} A_n e^{-\left(n + \dfrac12\right)y} \sin{\left[\left(n + \dfrac12\right)x\right]},
		\end{equation}
		где коэффициенты $A_n, \ n =0,1,2, \dots$ находятся из разложения
		\begin{equation}
			\sum\limits_{n=0}^{\infty} A_n \left(n + \dfrac12 \right) \sin{\left[\left(n +\dfrac12\right)x + \dfrac\pi4\right]} = \dfrac{\varphi(x)}{\sqrt2}
	\end{equation}}
	
	\textbf{Теорема 3.} \textit{Решение задачи (8) - (11) единственно.}
\end{frame}
\begin{frame} {Интегральное представление производных решения}
	\textbf{Теорема 4.} \textit{	Пусть $u(x,y)$ - решение задачи $(8)-(11)$, тогда $u_x, u_y$ представимы в виде
		\begin{equation}
			u_y(x,y) = - Im\  \dfrac{ \sqrt{1 - e^{i2z}} }{\pi} e^{\dfrac{+iz}{2}} \int\limits_0^\pi  \dfrac{\sqrt{\sin{t}}}{\left(1 - e^{i(z+t)} \right) \left(1 - e^{i(z-t)}\right)}  \varphi(t) dt
		\end{equation}
		\begin{equation}
			u_x(x,y) = Re\   \dfrac{ \sqrt{1 - e^{i2z}} }{\pi} e^{\dfrac{+iz}{2}} \int\limits_0^\pi  \dfrac{\sqrt{\sin{t}}}{\left(1 - e^{i(z+t)} \right) \left(1 - e^{i(z-t)}\right)}  \varphi(t) dt,
	\end{equation}
	где $z = x + iy$.
}
	
\end{frame}

\end{document}