\documentclass[9pt]{article}
\usepackage[russian]{babel}
\usepackage{amsmath}
\title{Интегральное представление}
\author{Васильченко Д.Д.}
\begin{document}
	\section{Определения}
	\subsection{LU-разложение матрицы}
		LU-разложением матрицы A называется разложение вида $A = LU$, где L - нижнетреугольная матрица с единицами на диагонали, а U - верхнетреугольная матрица с ненулевыми диагнальными элементами.
	\subsection{Разложение Холецкого}
		Разложением Холецкого матрицы $A = A^{T} > 0$ называется разложение вида $A = L L^T$, где L - нижнетреугольная матрица с положительными числами на диагонали.
	\subsection{QR - разложение матрицы}
	    QR разложением матрицы A называется разложение вида $A = QR$, где Q - ортогональная матрица, а R - верхнетреугольная с положительными числами на диагонали.
	\subsection{Матрица вращения}
		Матрицей вращения называется матрица следующего вида:
		\newline

	, где $C_{ki}, S_{ki}$ - косинус и синус некоторого угла.
	\subsection{Матрица отражения}
		Пусть гиперплоскость описывается единичным вектором $u$, который ортогонален ей, тогда $H = I- 2uu^T$ - матрица отражений(Хаусхолдера). $H_u(x) = x - 2(x,u)u$ - оператор отражения (Хаусхолдера).\newpage
	\subsection{Ленточная матрица}
	 	Матрица $A = (a_{ij}) \in R^{n\times n}$ называется ленточной, если $a_{ij}$ = 0 при $i-j>p$, $a_{ij} =0$ при $j-i>q$, для некоторых $p,q \in \overline{0, n-1}$.  Причем:
	 	\begin{enumerate}
	 		\item Если $\exists i_1, j_1: \ i_1 - j_1 = p, a_{i_1j_1} \neq 0$, то p - нижняя ширина ленты матрицы A.
	 		\item Если $\exists i_2, j_2: \ j_1 - i_1 = q, a_{i_2j_2} \neq 0$, то q - верхняя ширина ленты матрицы A.
	 	\end{enumerate}
 	\subsection{Полуширина ленточной матрицы}
 		Пусть матрица A - ленточная и $p = q$, тогда число $p=q$  называется полушириной матрицы A.
 	\subsection{Число обусловленности матрицы}
 		Пусть $A\in R^{n \times n}, \ |A| \neq 0$ тогда число обусловленности матрицы: $cond(A) = \|A\| \| A^{-1} \|$
 	\subsection{Матрица перестановок}
 		Матрица перестановок общего вида - матрица, которая получается из единичной перестановкой некоторого количества строк. В каждой строке и каждом столбце этой матрица 1 элемент отличный от 0, этот элемент равен 1.
 	\subsection{PLU разложение матрицы}
 		PLU разложением матрицы A называется разложение вида $A = PLU$, где P - матрица перестановок, L - нижнетреугольная матрица с единицами на диагонали, а U - верхнетреугольная матица с ненулевыми диагнальными элементами
 	\subsection{Энергетическая норма}
 		$\| x\|_D = (Dx, x)^{1/2}$, где D - положительно определённый оператор.
 	\subsection{Предобуславливатель}
 		Матрица P называется предобуславливателем для A, если у $P^{-1} A$ число обусловленности меньше, чем у A. 
 	\subsection{Многочлен наилучшего равномерного приближения}
 		Многочлен наилучшего приближения - наилучшее приближение функции f(x) многочленом степени $\leq m$. Пусть $E^N$ - евклидово пространство, $L =  L(\phi_1, \dots, \phi_n)$, $n < N$, $dim L = n$. $\forall x \in E^N  \| x - \sum\limits_{k=1}^n \alpha_k \phi_k \|_E \to min$. $\exists ! p \in L , h \in L^T: \ x = p + h$ - наилучшее приближение.
 	\subsection{Многочлены Чебышева первого рода}
 		$P_0(x) = 1, \ P_1(x) = x$. $P_{n+1}(x) = 2x* P_n(x) - P_{n-1}(x)$. 
 	\subsection{Невязка}
 		$AX = B$. Вектор невязки: $R = B - AX'$, где $X'$ - приближенное решение. 
 	\subsection{A - сопряженные векторы}
 		Вектора $p^1, p^2, \dots, p^m$ называется A-сопряженными, если $(Ap^i, p^j) = \begin{cases}
 			= 0, i \neq j \\
 			\neq 0, i = j
 		\end{cases}$
 	\subsection{Пространства Крылова}
 		Пространоством Крылова, порожденным матрицей A и вектором f называют пространство $K^{(m)} = span\{f, Af, \dots , A^{m-1}f\}$
 	\subsection{Подобные матрицы}
 	Квадратные матрицы A и B одинакового порядка называются подобными, если существует невырожденная матрица P того же порядка, такая что $B = P^{-1} A P$
 	\subsection{Ортогонально подобные матрицы}
 		Квадратные матрицы A и B одинакового порядка называются подобными, если существует ортогональная матрица P того же порядка, такая что $B = P^{-1} A P$
 	\subsection{Отношение Рэлея}
 		Отношением Рэлея для матрицы A называется выражения вида \newline $R(x) = \dfrac{(Ax,x)}{(x,x)}, x \neq 0$.
 	\subsection{Матрица подобия}
 		Невырожденная матрица P называется матрицей подобия между A, B, если $B = P^{-1} A P$.
 	\subsection{Матрица Хесенберга}
 		Квадратная ленточная матрица с нижней полушириной $p_1 = 1$ и верхней пошириной $p_2 = n-1$. 
\end{document}