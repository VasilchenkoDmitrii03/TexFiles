\documentclass[9pt]{article}
\usepackage[russian]{babel}
\usepackage{amsmath, amssymb} % Пакеты для работы с математикой
\usepackage{amsthm} % Пакет для определения теорем
\usepackage[%
left=1.00in,%
right=1.00in,%
top=1.0in,%
bottom=1.0in,%
paperheight=11in,%
paperwidth=8.5in%
]{geometry}%

\newtheorem{theorem}{Теорема} % Нумерация теорем по главам
\newtheorem{lemma}{Лемма} % Нумерация лемм по главам
\newtheorem{example}{Пример} % Нумерация примеров по главам
\newtheorem{corollary}{Следствие} % Нумерация следствий по главам
\newtheorem{notabene}{Замечание}
\title{Изучение функциональных методов решения задач математической физички с обобщёнными решениями}

\author{Васильченко Д.Д.}
\date{}
\begin{document}
	\maketitle
	\section{Свойства пространства финитных бесконечно дифференцируемых функций и определение обобщённой производной произвольного порядка}
	
	Пусть задано открытое ограниченное множество $\Omega \subset \mathbb{R}^n$. Пространство $D(\Omega)$ состоит из всех бесконечное число раз дифференцируемых функций $v: \ \Omega \to \mathbb{R}^1$ с компактными носителями. Будем говорить, что последовательность функций $v_1, v_2, \dots$ из$D(\Omega)$ сходится к функции $V \in D(\Omega)$, если при каждом $\alpha = (\alpha_1, \dots, \alpha_n)$ последовательность $D^\alpha v_k(x)$  равномерно на $\Omega$ сходится к функции $D^\alpha v(x)$.
	
	Рассмотрим множество $C^m(\overline{\Omega})$ всех непрерывных в $\overline{\Omega}$ функций, имеющиъ в области $\Omega$ все производные до порядка m, непрерывные в $\overline{\Omega}$. В множесте $C^m(\overline{\Omega})$ можно ввести норму
	\begin{equation*}
		\|f\|_{C^m(\overline{\Omega})} = \max\limits_{|\alpha| \leq m} \max\limits_{x \in \overline{\Omega}} |D^\alpha f(x)| = \max\limits_{0 \leq k \leq m} \|f\|_{C^k(\overline{\Omega})}.
	\end{equation*}

	Множество $C^m(\overline{\Omega})$ является сепарабельным банаховым пространством.
	
	Функция $D^\alpha f(x) \in L_p(\Omega)$ называется обобщённой производной порядка $\alpha$ в области $\Omega$ функции $f(x) \in L_p(\Omega)$, если для любой функции $\varphi(x) \in D(\Omega)$ выполняется тождество
	\begin{equation*}
		\int\limits_{\Omega} D^\alpha f(x) \varphi(x) dx = (-1)^{|\alpha|} \int\limits_{\Omega} f(x) D^\alpha \varphi(x) dx
	\end{equation*}
	\section{Примеры классических функций с обобщёнными производными и без них}
	\begin{example}
		Функция $f(x) = |x_1|$ в шаре $\Omega = \{x \in \mathbb{R}^n \vert \ |x| < 1\}$ имеет первые обобщённые производные $\dfrac{\partial f}{\partial x_1 } = \mathrm{sign} x_1, \ \dfrac{\partial f}{\partial x_i } = 0, \ i = 2, 3, \dots, n$ из $L_p(\Omega)$ при любом p таком, что $1 \leq p \leq \infty$. В самом деле, расписывая  ($\alpha  = (1, 0, \dots, 0)$)
		$$
			\int\limits_{\Omega}  f(x) D^\alpha \varphi(x) dx =  +  \int\limits_{\Omega \cap \{x_1 \geq 0\}} x_1  D^\alpha\varphi(x) dx - \int\limits_{\Omega \cap \{x_1 \leq 0\}} x_1 D^\alpha\varphi(x) dx = +  \int\limits_{\Omega \cap \{x_1 \geq 0\}} \varphi(x) dx - \int\limits_{\Omega \cap \{x_1 \leq 0\}}\varphi(x) dx =
		$$
		$$
		= \int\limits_{\Omega} \mathrm{sign} x_1 \varphi(x) dx
		$$
	\end{example}
	\begin{example}
		Функция $f(x) = \mathrm{sign}x_1$ в шаре $\Omega = \{x \in \mathbb{R}^n \vert \  |x| < 1\}$ не имеет обобщённой производной $\dfrac{\partial f}{\partial x_1}$. Это следует из того, что функция $V(x)$ , удовлетворяющая 
		$$
		\int\limits_{\Omega} \mathrm{sign}x_1 D^\alpha \varphi(x) dx=  \int\limits_{\Omega} V(x) \varphi(x) dx
		$$
		является сингулярной. 
	\end{example}
	\section{Определение и свойства пространств Соболева}
	Для всякого целого $m \geq 0$ и действительного числа $p$, такого, что $1 \leq p \leq \infty$, определим пространство Соболева $W_p^m(\Omega)$, состоящее из тех функций $u \in L_p(\Omega)$, для которых все частные производные $D^\alpha u(x)$ (в смысле обобщённых производных) при $|\alpha | \leq m$ принадлежат пространству $L_p(\Omega)$.
	
	Можно ввести норму
	\begin{equation*}
		\|u\|_{m,p,\Omega} = \|u\|_{W^m_p(\Omega)} = \left(\sum\limits_{s = 0}^m |u|_{s,p,\Omega}^p\right)^{1/p}
	\end{equation*}
	\begin{equation*}
		|u|_{m,p,\Omega} \equiv |u|_{W^m_p(\Omega)} = \left(\sum\limits_{|\alpha| = m} \|D^\alpha u\|^p_{0,p,\Omega}\right)^{1/p}.
	\end{equation*}
	Относительно такой нормы пространство $W_p^m(\Omega)$ является банаховым пространством.
	
	Есть и другой способ получения пространства Соболева. Пространство $\stackrel{0}{W_p^m}(\Omega)$ является замыканием множества $D(\Omega)$ в норме пространства $W_p^m(\Omega)$. Очевидно, что $\stackrel{0}{W_p^m}(\Omega)\subset W_p^m(\Omega)$. Докажем, что в пространстве $\stackrel{0}{W_p^m}(\Omega)$ полунорма $|*|_{m, \Omega}$ эквивалентна норме $\|* \|_{m, \Omega}$.
	
	В случае $m=1$ доказательство опирается на широко известное неравенство Фридрихса
	\begin{equation*}
		|u|^2{1,\Omega} = \int\limits_{\Omega} |\mathrm{grad} u|^2 dx \geq \gamma \int\limits_{\Omega} u^2 dx  = \gamma \|u\|^2_{0,\Omega}, 
	\end{equation*}
	где $\gamma = n a^{-1}$, причем $a$ - длина стороны n-мерного куба, содержащего область $\Omega$. Докажем это неравенство для случая, когда $\Omega \subset \mathbb{R}^2$. Предположим, что область $\Omega$ можно заключить в квадрат $\Pi = \{x: \ 0 < x_1 < a, \ 0 < x_2 < a\}$. Так как пространство $\stackrel{0}{W_p^m}(\Omega)$ есть замыкание множества $D(\Omega)$ в норме $W^1_2$, то достаточно доказать неравенство лишь для функций из класса $D(\Omega)$.
	
	Предположим, что $u(x) \in D(\Omega)$. Продолжим функцию $u(x)$ на всю плоскость $\mathbb{R}^2$ нулём. При таком продолжении фукнция будет принадлежать $D(\Pi)$. Пусть $(x_1, x_2) \in \Pi$; тогда, учитывая, что $u(0, x_2) = 0$, имеем
	$$
	u(x_1, x_2) = \int\limits_0^{x_1} \dfrac{\partial u}{\partial \xi_1}(\xi_1, x_2) d\xi
	$$.
	Применяя неравенство Коши-Буняковского, получим
	$$
	u^2(x_1, x_2) \leq x_1 \int\limits_0^{x_2} \left(\dfrac{\partial u}{\partial \xi_1}(\xi_1, x_2) d\xi_1\right)^2 \leq a \int\limits_0^a  \left(\dfrac{\partial u}{\partial \xi_1}(\xi_1, x_2) d\xi_1\right)^2.
	$$
	Проинтегрировав это неравенство в пределах $0 < x_1 < a, \ 0 < x_2 < a$, имеем 
	$$
		\int\limits_{\Pi} u^2(x) dx \leq a^2 \int\limits_{\Pi} \left(\dfrac{\partial u}{\partial x_1}(x)\right)^2dx.
	$$
	Аналогично можем получить неравенство для производной по второй переменной, далее просуммируем эти неравенства  и получим требуемое.
	
	Таким образом получили, что для любой $u \in \stackrel{0}{W_p^m}(\Omega)$ выполнены оценки
	\begin{equation*}
		(1 + \gamma)^{1/2} \|u\|_{1, \Omega} \leq |u|_{1, \Omega} \leq \|u\|_{1, \Omega},
	\end{equation*}
	которые показывают эквивалентность $|u|_{1, \Omega}$ норме $\|u\|_{1,\Omega}$.
	
	\begin{example}
		Пусть $n=1$ и $\Omega = \{x: 0 < x < 1\}$. В $\Omega$ рассмотрим функцию 
		\begin{equation*}
			f(x) = \begin{cases}
				1, \ x < 0.5, \\
				0, \ x \geq 0.5.
			\end{cases}
		\end{equation*}
		 Покажем, что $f(x) \in W^\lambda_p(\Omega)$ при любом $\lambda $, таком, что $0 < \lambda < 1/p, \ 1 < p < \infty$. Действительно, 
		 $$
		 \int\limits_0^1 \int \limits_0^1 \dfrac{|f(x) - f(y)|^p}{|x-y|^{1+\lambda p}}dx dy = 2 \int\limits_{0.5}^1 \int\limits_{0}^{0.5} \dfrac{dy}{(x-y)^{1 + \lambda p}} dx \leq \dfrac{2}{\lambda p} \int\limits_{0.5}^1 (x - 0.5)^{-\lambda p} dx < \infty, \text{если } \lambda p < 1.
		 $$
	\end{example}
	\begin{example}
		Пусть $\Omega = \{x \in \mathbb{R}^n\vert \ |x_i| < 1, i = 1, \dots, n\}$ и 
		\begin{equation*}
			f(x) = \begin{cases}
				1, \ |x_i| < 0.5, i = 1, \dots, n \\
				0, \text{в остальных точках } \Omega.
			\end{cases}
		\end{equation*}
		Нетрудно показать, что $f(x) \in W^\lambda_2(\Omega)$ при любом $\lambda$ , удовлетворяющем условию $0 < \lambda < 0.5$.
		
	\end{example}
	\section{Теоремы вложения}
	\begin{theorem}
		Пусть граница $\Gamma$ области $\Omega$ принадлежит классу $C^m$.Если $u \in W_p^m(\Omega)$, то след $v = u\vert_\Gamma$ принадлежит пространству $W^{m-1/p}_p(\Gamma)$ и выполняется оценка
		\begin{equation*}
			\|v\|_{m-1/p, p, \Gamma} \leq K_1 \|u\|_{m,p,\Omega}.
		\end{equation*}
		Обратно, если $v \in W^{m- 1/p}_p(\Gamma)$, то существует функция $u \in W_p^m(\Omega)$ такая, что $v = u\vert_\Gamma$, и выполнена оценка
		\begin{equation*}
			\|u\|_{m-1, 2, \Omega} \leq K_2 \|v\|_{m-1/p, p, \Gamma}
		\end{equation*}
	\end{theorem}
	\begin{theorem}
		(теорема вложения Соболева)\newline
		Пусть $\Omega$ - открытая область в $\mathbb{R}^n$ с непрерывной по Липшицу границей и пусть $\Omega^k$ - k-мерная область, полученная пересечением $\Omega$ с k-мерной гиперплоскотсью в $\mathbb{R}^n$, $1 \leq k \leq n$. Пусть, далее , $m$ - неотрицательное действительное число и $1 \leq p \leq \infty$. Тогда выполнены следующие вложения:
		\begin{enumerate}
			\item если $mp < n$ и $n - mp < k \leq n$, то 
			$$
				W^m_p \subset L_q(\Omega^k), \ p \leq q \leq kp/(n-mp);
			$$
			В частности при $k = n$ справедливо
			$$
			\|u\|_{0,1,\Omega} \leq C \|u\|_{m,p,\Omega}, \ p \leq q \leq np /(n-mp);
			$$
			\item если $mp = n$, то для любого $k: \ 1 \leq k \leq n$
			$$
				W^m_p(\Omega) \subset L_q(\Omega^k), \ p \leq q < \infty;
			$$
			В частности при $k = n$ справедливо
			$$
				\|u\|_{0,q,\Omega} \leq C \|u\|_{m,p,\Omega}, \ p \leq q < \infty;
			$$
			Если $p=1$ и $m=n$, то вложение имеет место и для $q= \infty$.
			\item Если $mp > n$, то 
			$$
				W^m_p(\Omega) \subset L_\infty(\Omega);
			$$
			если $mp > n > (m-1)p$, то 
			$$
				W^m_p(\Omega) \subset C^{0,\lambda}(\overline{\Omega}), \ 0 < \lambda \leq m - \dfrac{n}{p};
			$$
			если $n =(m-1)p$, то
			$$
			W^m_p(\Omega) \subset C^{0,\lambda} (\overline{\Omega}), \ 0 < \lambda < 1.
			$$
		\end{enumerate}
	\end{theorem}
	\begin{theorem} (Упрощённый вариант теоремы вложения) \newline
		Пусть \( u \in W^1_2(a,b) \). Тогда \( u \in C[a,b] \).
	\end{theorem}
	\begin{proof}
 Рассмотрим функцию \( u \in W^1_2(a,b) \). Это означает, что \( u \) и её первая производная \( u' \) принадлежат \( L^2(a,b) \).
Для любых \( x, y \in [a,b] \) с \( x < y \) имеем:
	\[
	u(y) - u(x) = \int_x^y u'(t) \, dt.
	\]
 Применяя неравенство Коши-Буняковского, получаем:
	\[
	|u(y) - u(x)| = \left| \int_x^y u'(t) \, dt \right| \leq \int_x^y |u'(t)| \, dt.
	\]
Используя неравенство Коши-Буняковского для интеграла, получаем:
	\[
	\int_x^y |u'(t)| \, dt \leq \left( \int_x^y 1^2 \, dt \right)^{1/2} \left( \int_x^y |u'(t)|^2 \, dt \right)^{1/2}.
	\] Учитывая, что \( \int_x^y 1^2 \, dt = y - x \), получаем:
	\[
	|u(y) - u(x)| \leq (y - x)^{1/2} \left( \int_a^b |u'(t)|^2 \, dt \right)^{1/2}.
	\]
	Так как \( u' \in L^2(a,b) \), интеграл \( \int_a^b |u'(t)|^2 \, dt \) конечен. Пусть \( C = \left( \int_a^b |u'(t)|^2 \, dt \right)^{1/2} \). Тогда:
	\[
	|u(y) - u(x)| \leq C (y - x)^{1/2}.
	\]
	Это показывает, что \( u \) является равномерно непрерывной функцией на \([a,b]\). Следовательно, \( u \in C[a,b] \).
	\end{proof}
	\section{Лемма Брэмбла-Гильберта}
	\begin{lemma} (Брэмбла-Гильберта)\newline
		Пусть $\Omega$ - открытая выпуклая ограниченная область в $\mathbb{R}^n$ с диаметром $d$.Пусть, далее, линейный функционал $l(u)$ ограничен в пространстве $W^m_2(\Omega)$, где $0 < m = \overline{m} + \lambda$, $\overline{m}$ - целое неотрицательное чилсло, $0 < \lambda  \leq 1$, т.е.
		\begin{equation*}
			|l(u)| \leq M\left(\sum\limits_{j=0}^{\overline{m}} d^{2j}|u|^2_{j,\Omega} + d^{2m}|u|^2_{m,\Omega}\right)^{1/2}.
		\end{equation*}
		Если $l(u)$ обращается в нуль на многочленах степени $\overline{m}$ по переменным $x_1, \dots, x_n$, то существует постоянная $\overline{M}$, зависящая от $\Omega$, но не зависящая от $u(x)$, такая, что выполнено неравенство
		$$
		|l(u)| \leq M \overline{M} d^m |u|_{m,\Omega}.
		$$
	\end{lemma}
\begin{example}

	Применим лемму Брэмбла-Гильберта для оценки погрешности приближенного интегрирования
	$$
		l(u) = \int\limits_0^h u(x) dx - \dfrac{h}{2} (u(0) + u(h)), \ u \in W^2_2.
	$$
	Чтобы применить оценки теоремы вложения сделаем замену переменной $ t = x/h$ и положим $\tilde{u}(t) = u(th)$, тогда
	$$
		l(\tilde{u}) = \dfrac{h}2 \left[2 \int\limits_0^1 \tilde{u}(t) dt - \tilde{u}(0) - \tilde{u}(1)\right].
	$$
	Так как
	$$
		\max\limits_{t \in [0,1]} |\tilde{u}(t)| \leq \sqrt{2} \left(\int\limits_0^1 \left(\tilde{u}^2  + \tilde{u'}^2\right) dt\right)^{1/2} \leq \sqrt2 \|\tilde{u}\|_{W^2_2(0,1)},
	$$
	то
	$$
	|l(\tilde{u})| \leq \dfrac52 h \|\tilde{u}\|_{W^2_2(0,1)}.
	$$
	$l(\tilde{u}) = 0$  на многочленах первой степени, поэтому по лемме Брэмбла-Гильберта 
	$$
		|l(u)| \leq Mh\|\tilde{u}\|_{W^2_2(0,1)} = Mh^{5/2}\|u\|_{W^2_2(0,h)}.
	$$
	Теперь оценим ошибку квадратурной формулы трапеций на $[0,1]$ с шагом $h = 1/N$ на функциях из класса $W^2_2([0,1])$:
	$$
		l(f) = \int\limits_0^1 f(x)dx - \dfrac{h}{2} \sum\limits_{i=1}^{N}(f(x_{i-1}) + f(x_i)), \ x_i = ih.
	$$
	Перепишем в виде
	$$
		l(f) = \sum\limits_{i=1}^{N} \left\{\int\limits_{x_{i-1}}^{x_i} f(x)dx - \dfrac{h}{2} \left[f(x_{i-1}) + f(x_i)\right] \right\}
	$$
	для каждого слагаемого в скобке применим оценку, полученную выше.
	$$
	\int\limits_{x_{i-1}}^{x_i} f(x)dx - \dfrac{h}{2} \left[f(x_{i-1}) + f(x_i)\right] \leq M h^{5/2} \|f\|_{W^2_2(x_{i-1}, x_i)},
	$$
	суммируя получим
	$$
	|l(f)| \leq Mh^{5/2} \sum\limits_{i=1}^N \|f\|_{W^2_2 (x_{i-1}, x_i)} \leq Mh^2\|f\|_{W^2_2(0,1)}.
	$$
\end{example}
	\section{Определение обобщённого решения задачи Дирихле для эллиптического оператора}
	Пусть в конечной области $\Omega$ пространства $\mathbb{R}^n$  c границей $\Gamma$ задано самосопряженное эллиптическое уравнение второго порядка
	$$
		Lu = - \sum\limits_{i,j = 1}^n \dfrac{\partial}{\partial x_j}\left(a_{ij}(x) \dfrac{\partial u}{\partial x_j}\right) + q(x)u = f(x), \ x \in Q, \eqno{(1)} 
	$$
	коэффициенты которого удовлетворяют следующим условиям:
	$$
		a_{ij} = a_{ji} \in C^1(\overline{\Omega}), \ f(x) \in  C(\overline{\Omega}), q(x) \in C(\overline{\Omega}), \eqno{(2)}
	$$
	$$
		\sum\limits_{i,j =1}^n a_{ij}(x) \xi_i \xi_j \geq \gamma \sum\limits_{i=1}^{n} \xi_i^2, \ \gamma = \mathrm{const} > 0 \eqno{(3)}
	$$
	для любого $\xi = (\xi_1, \dots, \xi_n) \in \mathbb{R}^n$ (равномерная эллиптичность).
	
	Функция $u(x) \in C^2(\Omega) \cap C^1(\overline{\Omega})$ называется решением (классическим решением) первой краевой задачи (или задачи Дирихле) для выше укзанного уравнения, если в $\Omega$ она удовлетворяет этому уравнению, а на границе $\Gamma$ - условию 
	$$
		u(x) = \mu(x), \ x \in \Gamma, \eqno{(4)}
	$$
	где $\mu(x)$ - заданная функция.
	\begin{theorem}
		Пусть коэффициенты $a_ij(x)$ и $q(x)$ оператора $L$ принадлежат $C^{m-1, \alpha}(\overline{\Omega})$, $a_{ij}$ удовлетворяют неравенству (3), $q(x) \geq 0$, а граница $\Gamma$ принадлежит классу $C^{m, \alpha}$. Тогда для любых $f \in C^{m-2, \alpha}(\overline{\Omega})$ и $\mu \in C^{m, \alpha} (\Gamma)$ задача (1)-(4) имеет единственное решение из класса $C^{m,\alpha} (\overline{\Omega}), \ m \geq 2$.
	\end{theorem}
	\section{Точная интегро-дифференциальная схема}
	\section{Точная разностная схема для задачи Штурма-Лиувилля}
		Рассмотрим краевую задачу
		$$
		L^(k,q)u \equiv \dfrac{d}{d x}(k(x)\dfrac{du}{dx}) - q(x) u(x) = -f(x), \ x \in (0,1)  \equiv  \Omega \eqno{(1)}
		$$
		$$
			u(0) = u(1) = 0. \eqno{(2)}
		$$
		$$
		\text{Пусть } 0 < M_1 \leq k(x) \leq M_2 < \infty, k(x) \text{ - измеримая функция}; \eqno{(3)}
		$$
		$$
			q(x) = Q'(x), Q(x) \in W^\lambda_p(\Omega), p \geq 2, 0 < \lambda \leq 1, \eqno{(4)}
		$$
		причем $\int\limits_0^1 Q(x)v'(x) dx \geq 0$ для любой функции $v \in \stackrel{0}{W_1^2}(\Omega)$ такой, что $v(x) \geq 0$.
		$$
			f(x) = F'(x), F(x) \in W^\theta_r(\Omega), r \geq 2, 0 < \theta \leq 1. \eqno{(5)}.
		$$
		
		На отрезке $[0,1]$ введем равномерную сетку 
		$$
			\omega  = \{x_ = i/h, i = 1, \dots, N -1 , h = 1/N\}, x_0 = 0, x_N = 1
		$$
		и построим разностную схему, заменяющую задачу (1)-(2)
		$$
			y(x_i) = A_iy(x_{i+1}) + B_iy(x_{i-1}) + F_i \ (i = 1, 2, \dots, N-1), \eqno{(6)}
		$$
		$$
			y(0) = y(1) = 0,
		$$
		где $A_i$, $B_i$ и $F_i$ - некоторые функционалы от коэффициентов $k(x)$, $q(x)$ и $f(x)$ исходного уравнения.
		
		Трехточечную разностую схему вида (6) назовем точной для задачи
		$$
			\int\limits_0^1 \left[k(\xi) u'(\xi)\eta'(\xi) - Q(\xi)(u(\xi)\eta(\xi))'\right]d\xi = -\int\limits_0^1 F(\xi) \eta'(\xi) d\xi, \eqno{(7)}
		$$
		если выполняются условия
		\begin{enumerate}
			\item $A_i = A_i(k(*), q(*)), \ B_i = B_i(k(*), q(*)), F_i = F_i(k(*), q(*), f(*))$, где $F_i$ - линейный функционал от 3 переменной.
			\item $y(x) = u(x), x \in \omega$.
		\end{enumerate}
		\begin{lemma}
			Пусть выполнены условия (3)-(5). Тогда для задачи (1)-(2) существует хотя бы одна точная трехточечная разностная схема.
		\end{lemma}
\end{document}
