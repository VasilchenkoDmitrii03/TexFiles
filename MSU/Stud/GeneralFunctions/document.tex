\documentclass[9pt, a4paper]{article}

\usepackage[russian]{babel}
\usepackage{amsfonts, amssymb, amsmath, mathabx, dsfont}

% theorems, lemmas, etc.
\usepackage{amsthm}
\newtheorem*{theorem}{Теорема}
\newtheorem*{lemma}{Лемма}
\newtheorem*{corollary}{Следствие}
\newtheorem*{notabene}{Замечание}
\newtheorem*{definition}{Определение}
\newtheorem*{sample}{Пример}

% enumerating settings
\DeclareMathOperator*\lowlim{\underline{lim}}
\DeclareMathOperator*\uplim{\overline{lim}}

\usepackage[left=10mm, top=20mm, right=10mm, bottom=20mm, nohead, footskip=10mm]{geometry}
\title{Обобщённые функции}
\author{Ломов И.С.}
\date{}
\begin{document}
\maketitle
\section*{Элементарная теория обобщённых функций}
	\subsection*{Основные функции в $\mathbb{R}^1$}
		\begin{definition}
			Под основной функцией понимают любую вещественную функцию,  финитную на $\mathbb{R}$ и определенную на $\mathbb{R}$ и непрерывную вместе с любой производной конечного порядка на $\mathbb{R}$.
		\end{definition}
		Если $\varphi = 0$ вне $[a,b]$, то говорят, что $\varphi$ сосредоточена на $[a,b]$. В этом случае $[a,b]$ - носитель $\varphi(x)$. $\mathrm{supp} \varphi(x) = \overline{\{x: \varphi(x) \neq 0\}}$.
		Пространство финитных функций является линейным, проверяется тривиально.
		\subsubsection*{Предельный переход в $K$}
			Пусть $\varphi_n$ - последовательность основных функций.
			\begin{definition}
				$\varphi_n(x) \to 0$ в $K$, если все $\varphi_n(x)$ сосредоточены на одном отрезке, последовательность $\varphi_n(x) \rightrightarrows 0$ при $n \to \infty$ на этом отрезке и $\forall k \in \mathbb{N}\ \varphi_n^{(k)} \rightrightarrows 0$ при $n \to \infty$  
			\end{definition}
			Очевидно, что $\varphi_n \to \varphi$ в $K$, если $ \varphi \in K$ и  $\varphi_n - \varphi \to 0$ в $K$.
			\begin{sample}
				"Шапочка"\newline
				\begin{equation*}
					\varphi(x; a) = \begin{cases}
						e^{-\dfrac{a^2}{a^2 - x^2}}, \ |x| \leq a \\
						0, \ |x| > a
					\end{cases}
				\end{equation*}
				Пусть $\varphi_n = \dfrac1{n} \varphi(x;a) \to 0 $ в $K$, но если возьмем $\varphi_n = \dfrac{1}{n} \varphi(\dfrac{x}{n}, a)$, то сходимости не будет т.к. функции сосредоточены на разных отрезках.
			\end{sample}
			\begin{sample}
				"Срезка"\newline
				\begin{equation*}
					1_R(x) = \begin{cases}
						1, \ |x| \leq R\\
						0, \ |x| > 3R \\
						\text{монотонно убывает}, x \in [-3R, -R] \\
						\text{Монотонно возрастает}, x \in [R, 3R]
					\end{cases}
				\end{equation*}
				Пусть $g(x) \in C^\infty(\mathbb{R}) \Rightarrow g(x)1_R(x) \in K$. В этом случае на $x \in [-R, R] \Rightarrow g(x)1_R(x) = g(x)$.\newline
				Более общая срезка:
				\begin{equation*}
					\omega_\varepsilon(x) = \begin{cases}
						C_\varepsilon e^{-\dfrac{\varepsilon^2}{\varepsilon^2 - x^2}}, \ |x| \leq \varepsilon \\
						0, \ |x| > \varepsilon
					\end{cases}
				\end{equation*}
				$C_\varepsilon$ выбираем из условия $\int \omega_\varepsilon(x) dx = 1$.
			\end{sample}
			\begin{lemma}
				$\exists \eta(x) \in K$, такая, что $\forall x: \ 0 \leq \eta(x) \leq 1$; $x \in G_\varepsilon = (a- \varepsilon, b + \varepsilon)$; $G = (a,b)$ и $\eta(x) = \begin{cases}
					1, \ x \in G_\varepsilon \\
					0, \ x \notin G_\varepsilon
				\end{cases}$
			\end{lemma}
			\begin{proof}
				Пусть $\chi(x)$ - характеристическая функция множества $G_{2\varepsilon}$, то есть индикатор. Пусть $\eta(x) = \int_{\mathbb{R}} \chi(x) \omega_\varepsilon(x-y) dy$. Покажем, что эта функция принадлежит класса $C^\infty$. $\eta(x) = \int\limits_{G_{2\varepsilon}} \omega_{\varepsilon}(x-y) dy = \int\limits_{a-2\varepsilon}^{b + 2\varepsilon} \omega_{\varepsilon}(x-y) dy = \int\limits_{x - (b + 2\varepsilon)}^{x - (a - 2\varepsilon)} \omega_\varepsilon(t) dt = \{\omega_\varepsilon \in C^\infty\}$. Поэтому $\eta(x) \in C^\infty (\mathbb{R})$.\newline
				Проверим условие $0 \leq \eta(x) \leq 1$. Функция $\omega_\varepsilon$ неотрицательная, поэтому левая оценка выполнена, а правая оценка выполняется благодаря выбору константы $C_\varepsilon$ так как $\int\limits_{x - (b + 2\varepsilon)}^{x - (a - 2\varepsilon)} \omega_\varepsilon(t) dt \leq \int\limits_{\mathbb{R}} \omega_\varepsilon(t) dt  = 1$. В итоге $0 \leq \eta(x) \leq 1$.\newline
				Остаётся проверить последнее условие: Пусть $|y-x| \leq \varepsilon \Rightarrow x - \varepsilon \leq y \leq x + \varepsilon$, тогда $\eta(x) = \int\limits_{x - \varepsilon}^{x+\varepsilon} \chi(y) \omega_\varepsilon(x-y) dy = \begin{cases}
					\int\limits_{x-\varepsilon}^{x+\varepsilon} \omega_\varepsilon(x-y)dy = \int\limits_{-\varepsilon}^{\varepsilon} \omega_\varepsilon dy = 1, \ x \in G_\varepsilon \subset G_{2\varepsilon} \\
					0, \ x \notin G_\varepsilon, \text{потому что индикатор обращатеся в ноль}
				\end{cases}$
				Если $a=b=0, \varepsilon = R \Rightarrow \eta(x) = 1_R(x)$.
			\end{proof}
			\begin{theorem}
				(неметризуемость пространства $K$)\newline
				$\nexists \rho$ такой, что если $\varphi_n \to \varphi$ в $K$, то $\rho(\varphi_n, \varphi) \to 0$.
			\end{theorem}
			\begin{proof}
				Известна теорема о метрических пространствах: если есть в метрическом пространстве счётное число последовательностей 
				$\begin{matrix}
					\varphi_1^{(1)} & \varphi_2^{(1)} & \dots & \varphi_n^{(1)} & \to & \varphi^{(1)} \\
					\varphi_1^{(2)} & \varphi_2^{(2)} & \dots & \varphi_n^{(2)} & \to & \varphi^{(2)}\\
					\dots & \dots & \dots & \dots & \dots & \dots \\
					\varphi_1^{(m)} & \varphi_2^{(m)} & \dots & \varphi_n^{(m)} & \to & \varphi^{(m)}\\
					\dots & \dots & \dots & \dots & \dots & \dots \\
				\end{matrix}$
				таких, что $\varphi^{(m)} \to \varphi$ при $m\to \infty$, то $\exists \{\varphi_{n_m}^{(m)}\}$ - сходящаяся к $\varphi$ при $n \to \infty$. Рассмотрим контрпример: $\varphi_n^{(m)}(x) = \dfrac{1}{n}\varphi(\dfrac{x}{m};a )$. Для любого фиксированного $m$ $\varphi_n^{(m)} (x) \to 0$ в $K$.Но если взять последовательность $\varphi_{n_m}^{(m)}(x) \to \dfrac{1}{n_m}\varphi(\dfrac{1}{m}; a)$, то не будет общего носителя.
			\end{proof}
		\subsection*{Обобщённые функции в $\mathbb{R}^1$}
			\begin{definition}
				$E$ - множество обычных вещественных функций, определенных на $\mathbb{R}$, локально интегрируемых.
			\end{definition}
			Пусть $f(x) \in E$, ставим в соответствие функционал на множестве $K$: $(f, \varphi) = \int\limits_{\mathbb{R}} f(x) \varphi(x) dx \ (1)$.\newline
			Функционал (1) очевидно является линейным, его непрерывность следует из $\{\varphi_n(x)\} \subset K$, $\varphi_n(x) \to 0$ в $K$ $\Rightarrow (f, \varphi_n) \to 0$. 
			\begin{lemma}
				Существуют линейные непрерывные функционалы на $K$, которые не представимы в виде (1).
			\end{lemma}
			\begin{proof}
				$\delta(x)$ - дельта-функция Дирака: $\delta(x): \varphi(x) \to \varphi(0)$. Покажем, что этот функционал не представим в виде (1). Пусть $\exists f(x) \in E: \int\limits_{\mathbb{R}} f(x)\varphi(x) dx = \varphi(0), \forall \varphi \in K$. Пусть $\varphi(x) = \varphi(x; a)$, тогда $\int\limits_{\mathbb{R}} f(x) \varphi(x) dx = \int\limits_{-a}^{a} f(x) e^{-\dfrac{a^2}{a^2 - x^2}} dx \leq \int\limits_{-a}^{a} |f(x)| dx \to 0 $ при $a \to 0$. Но $\varphi(0; a) = \dfrac{1}{e}$.Поэтому данный функционал в виде (1) не представим.
			\end{proof}
			\begin{definition}
				Обобщённой функцией (распределением) назовем любой линейный непрерывный функционал на множестве $K$. Если функционал представим в виде (1), то он регулярный, иначе сингулярный.
			\end{definition}
			$K'$ - множество всех обобщённых функций над $K$. \newline
			Любой обычной функции $f(x)$ отвечает обобщённая функция, определяемая по формуле (1) $f(x) = \mathrm{const}: \ (c,\varphi) = c \int\limits_{\mathbb{R}} \varphi(x) dx$. $f(x): \forall x \to f(x)$ почти всюду. $f: \forall \varphi \to (f, \varphi)$. Не можем говорить про равенство в точке, но можем говорить об эквивалентности на $(a,b)$.
			\subsubsection*{Сингулярные функции}
				\begin{enumerate}
					\item $\delta(x)$.
					\item $\delta(x - a), \forall a \in \mathbb{R}$.
					\item $\delta'(x)$
					\item $f(x) = \dfrac{1}{x} \notin E$
				\end{enumerate}
			Пусть $f_1, f_2 \in K'$ равны, если $(f_1, \varphi) = (f_2, \varphi), \forall \varphi \in K$, не являются равными, если $\exists \varphi \in K: (f_1, \varphi) \neq (f_2, \varphi)$.
			Класс $K$ достаточно широк, чтобы различать непрерывные функции: 
			\begin{lemma}
				Пусть $f_1(x), f_2(x) \in E$ - различные непрерывные функции, тогда $f_1, f_2$ -различные обобщённые функции.
			\end{lemma}
			\begin{proof}
				Нужно показать, что $\exists \varphi_0: (f_1, \varphi_0) \neq (f_2, \varphi_0)$. Рассмотрим $f(x) = f_1(x) - f_2(x)$, тогда $\exists x_0: f(x_0) \neq 0$ и $\exists [\alpha, \beta]:\ x_0 \in [\alpha, \beta]$ на этом отрезке функция $f(x)$ сохраняет знак. Рассмотрим $\varphi_0(x) = \begin{cases}
					e^{-\dfrac{1}{(\beta-x)(x-\alpha)}}, \ x \in [\alpha, \beta] \\
					0, \ x \in \mathbb{R} \backslash [\alpha, \beta]
				\end{cases} $. Заметим, что $\varphi_0 \in K$. $\int\limits_{\mathbb{R}} f(x) \varphi_0(x) dx = \int\limits_{\alpha}^{\beta} f(x) e^{-\dfrac{1}{(\beta-x)(x-\alpha)}} dx  > 0$ т.к. $f(x)$ - сохраняет знак, а экспонента строго положительна, поэтому $f_1 \neq f_2$.
			\end{proof}
			Пусть $p \geq 0$, целое число
			\begin{definition}
				Обобщённая функция $f$ имеет порядок сингулярности  $\leq p$, если её можно представить в следующем виде: 
				\begin{equation*}
					(f,\varphi) = \sum\limits_{k=0}^{p} \int\limits_{\mathbb{R}} f_k(x) \varphi^{(k)}(x) dx = \sum\limits_{k=0}^{p} (f_k(x), \varphi^{(k)}(x)), \forall \varphi \in K,  \eqno{(1)}
				\end{equation*}
				где $f_1(x), \dots, f_p(x) \in E$
			\end{definition}
			\begin{sample}
				$f(x) \in E$, тогда регулярная $\Rightarrow p = 0$.
			\end{sample}
			\begin{sample}
				$\delta(x)$. Рассмотрим функцию Хевисайда $\theta(x)  =\begin{cases}
					1, x \geq 0 \\
					0, x < 0
				\end{cases} \in E$. $(\theta(x), \varphi(x)) = \int\limits_{-\infty}^{\infty} \theta(x) \varphi(x) dx = \int\limits_{0}^{\infty} \varphi(x) dx$ \newline
				$(\delta(x), \varphi(x)) = \varphi(0)  = - \int\limits_{0}^{\infty} \varphi'(x) dx = \int\limits_{-\infty}^{\infty} - \theta(x) \varphi'(x) dx$. Поэтому порядок сингурлярности $\delta(x)$ равен 1, а для $\delta'(x)$ $p \leq 2$.
			\end{sample}
	\subsection*{Действие с обобщёнными функциями}
		\subsubsection*{Сложение}
			Сложение и умножение на вещественное число: $\forall f_1, f_2 \in K', \ \forall \alpha_1, \alpha_2 \in \mathbb{R}: \ (\alpha_1f_1+\alpha_2f_2, \varphi) = \alpha_1(f_1, \varphi)+\alpha_2(f_2, \varphi) \Rightarrow \alpha_1f_1+\alpha_2f_2 \in k'$
		\subsubsection*{Умножение на бесконечно дифференцируемую функцию}
			$\forall f \in k', \ \forall \alpha(x) \in C^\infty (\mathbb{R})$. 
			\begin{enumerate}
				\item $f = f(x) \in E \Rightarrow (\alpha(x)f(x), \varphi(x)) = \int\limits_{\mathbb{R}} \alpha(x)f(x)\varphi(x) dx = (f(x), \alpha(x)\varphi(x))$ т.к. $\alpha(x)\varphi(x) \in K$.
				\item $f \in K'$ $(\alpha(x)f, \varphi) = (f, \alpha(x)\varphi) \Rightarrow \alpha(x) f \in K'$ т.к. функционал линейный и непрерывный.
			\end{enumerate}
		\subsubsection*{Дифференциорвание}
			$\forall f \in K': \ f':(f', \varphi) = -(f, \varphi'), \forall \varphi \in K$. Пусть $\varphi_n \to 0$ в $K$, тогда $\varphi_n' \to 0 $ в $K$ $\Rightarrow (f, \varphi_n') \to 0$ т.к. $f$-непрерывный функционал $\Rightarrow$  $(f', \varphi_n) \to 0$, то есть  $f'$ - линейный непрерывный функционал $f' \in K$.
			
			Свойства производной:
			\begin{enumerate}
				\item $(f'', \varphi) = (f, \varphi'')$, $(f^{(n)}, \varphi) = (-1)^n(f, \varphi^{(n)})$
				\item $\forall \alpha_1, \alpha_2 \in \mathbb{R}, \forall f_1, f_2 \in K'$ $((\alpha_1f_1 + \alpha_2f_2)', \varphi) = -(\alpha_1f_1 + \alpha_2f_2, \varphi') = -\alpha_1(f_1, \varphi') - \alpha_2(f_2, \varphi') = \alpha_1(f_1', \varphi) + \alpha_2(f_2', \varphi)$. То есть $(\alpha_1f_1 + \alpha_2f_2)' = \alpha_1 f_1' + \alpha_2 f_2'$
				\item $\alpha(x) \in C^\infty (\mathbb{R}), f \in K'$ $((\alpha(x) f)', \varphi) = - (\alpha(x)f, \varphi') = -(f, \alpha(x) \varphi') = - (f, \alpha(x) \varphi' + \alpha'(x)\varphi - \alpha'(x) \varphi) = -(f, (\alpha \varphi)') + (f, \alpha'\varphi) = (f', \alpha \varphi) + (\alpha' f, \varphi) = (\alpha f'+ \alpha' f, \varphi), \forall \varphi \in K$. То есть $((\alpha(x)f)', \varphi) = (\alpha'f+\alpha f', \varphi)$
			\end{enumerate}
			\begin{sample}
				$\theta(x)$: $(\theta'(x), \varphi) = -(\theta(x), \varphi'(x)) = -\int\limits_0^{\infty} \varphi'(x) dx = \varphi(0) \Rightarrow \theta'(x) = \delta(x)$.
			\end{sample}
			\begin{sample}
				$\delta(x)$: $(\delta'(x), \varphi(x)) = - (\delta(x), \varphi'(x)) = - \int\limits_{\mathbb{R}} \delta(x) \varphi'(x) = - \varphi'(0)$. Получается, что $\delta': \varphi(x)  \to -\varphi'(0)$.
			\end{sample}
			\begin{sample}
				Пусть $f(x)$ - кусочно абсолютно непрерывная функция, $x_1, \dots, x_n$ - точки разрыва. $h_1, \dots, h_n$ - скачки в точках разрыва $f(x_i+0) - f(x_i-0) = h_i$. Чему равна производная такой функции? \newline
				Введём $f_1(x) = f(x) - \sum\limits_{k=1}^{n} h_k \theta(x-x_k)$ - убрали скачки и сделали непрерывной. $f_1(x)$ -абсолютно непрерынвая функция и $\exists f_1'(x)$ п.в. совпадает с $f'(x)$.
				$f'(x) = f_1'(x) + \sum\limits_{k=1}^n h_k \delta(x-x_k)$ в $K$.
			\end{sample}
			\begin{sample}
				Рассмотрим ряд $\sum\limits_{n=1}^{\infty} \dfrac{\sin{nx}}{n} = f(x) = \begin{cases}
					\dfrac{\pi - x}{2}, \ x\in (0, \pi]\\
					0,\ x = 0\\
					-\dfrac{\pi+x}{2}, \ x \in [ - \pi, 0)
				\end{cases}$ Это $2\pi$ -периодическая функция. По полученной ранее формуле получаем, что $f' = -\dfrac{1}{2} + \pi \sum\limits_{k=-\infty}^{\infty} \delta(x - 2\pi k)$
			\end{sample}
			\begin{sample}
				Сходимость ряда. \newline
				$\sum\limits_{n=1}^{\infty} \left(\dfrac{\sin{nx}}{n}\right)' = \sum\limits_{n=1}^{\infty}  \cos{nx}$ - расходится в прострнастве $E$. Посмотрим в пространстве $K'$: $\left(\left(\sum\limits_{n=1}^{N} \dfrac{\sin{nx}}{N}\right)', \varphi\right) = \left(\sum\limits_{n=1}^{N}  \cos{nx}, \varphi\right) = -(\sum\limits_{n=1}^N \dfrac{\sin{nx}}{n}, \varphi') = - \int\limits_{\mathbb{R}} \sum\limits_{n=1}^N \dfrac{\sin{nx}}{n} \varphi' dx  \to  - \int\limits_{\mathbb{R}} \sum\limits_{n=1}^\infty \dfrac{\sin{nx}}{n} \varphi' dx = -(f(x), \varphi'(x)) = (f'(x), \varphi)$. В пространстве $K'$ ряд $\sum\limits_{n=1}^{\infty} \cos{nx} = -\dfrac12 + \pi\sum\limits_{k=-\infty}^{+\infty} \delta(x - 2\pi k)$ 
			\end{sample}
			\begin{sample}
				$y = \ln{|x|} \in E$, но $y' \notin E$, а что в $K'$? \newline
				$((\ln{|x|})', \varphi) = -(\ln{|x|}, \varphi') = - \int\limits_{-\infty}^{\infty} \ln{|x|} \varphi'(x) dx = - \lim\limits_{\varepsilon \to 0 + 0} \left(\int\limits_{-\infty}^{-\varepsilon} \ln{|x|} \varphi' dx + \int\limits_{\varepsilon}^{\infty}  \ln{|x|} \varphi'dx \right) = $ \newline
				$=  - \lim\limits_{\varepsilon \to 0 + 0} \left(\ln{|x|}\varphi(x) \vert_{-\infty}^{-\varepsilon} - \int\limits_{-\infty}^{-\varepsilon} \dfrac{1}{x} \varphi dx + \ln{|x|}\varphi\vert_{\varepsilon}^{\infty} - \int\limits_{\varepsilon}^{\infty}  \dfrac{1}{x} \varphi dx \right) 
				=  - \lim\limits_{\varepsilon \to 0 + 0} \left(\ln{\varepsilon} \varphi(-\varepsilon) - \ln{\varepsilon}\varphi(\varepsilon) - \int\limits_{|x| \geq \varepsilon} \dfrac{\varphi(x)}{x} dx\right) = $ \newline
				$= - \lim\limits_{\varepsilon \to 0 + 0} \left(\ln{\varepsilon} \varphi'(x) \left(-2\varepsilon\right) - \int\limits_{|x| \geq \varepsilon} \dfrac{\varphi(x)}{x} dx\right) = \lim\limits_{\varepsilon \to 0 + 0} \int\limits_{|x| \geq \varepsilon} \dfrac{\varphi(x)}{x} dx = \mathrm{v.p.} \int\limits_{-\infty}^{\infty} \dfrac{\varphi(x)}{x}dx = (\dfrac{1}{x}, \varphi)$. Поэтому $(\ln{|x|})' = \dfrac{1}{x}$  в $K'$. 
			\end{sample}
			\begin{sample}
				Пусть $y = x_+^\lambda, \lambda \in (-1, 0)$, $x_+^\lambda = \begin{cases}
					x^\lambda, \ x > 0 \\
					0, x \leq 0
				\end{cases}	 \in E $. Что проихсодит в $K'$? \newline
				$((x_+^\lambda)', \varphi) = - (x_+^\lambda, \varphi') = - \int\limits_{-\infty}^{\infty} x_+^\lambda \varphi'(x) dx = - \int\limits_0^\infty x^\lambda \varphi'(x) dx = -\lim\limits_{\varepsilon \to 0 +0} \int\limits_{\varepsilon}^{\infty} x^\lambda \varphi'(x) dx =  -\varepsilon^\lambda \varphi(\varepsilon) - \int\limits_{\varepsilon}^{\infty} \lambda x^{\lambda -1} \varphi(x) dx =  -\varepsilon^\lambda \varphi(\varepsilon) - \int\limits_{\varepsilon}^{\infty} \lambda x^{\lambda -1} (\varphi(x) - \varphi(0)) dx - \int\limits_{\varepsilon}^{\infty} \lambda x^{\lambda -1} \varphi(0) dx = \varepsilon^\lambda (\varphi(0) - \varphi(\varepsilon)) - \int\limits_{\varepsilon}^{\infty} \lambda x^{\lambda -1} (\varphi(x) - \varphi(0))dx \to - \int\limits_0^\infty \lambda x^{\lambda -1} (\varphi(x) - \varphi(0))dx$\newline
				В итоге $(x_+^\lambda)': \varphi(x) \to \int\limits_0^\infty \lambda x^{\lambda -1} (\varphi(x) - \varphi(0))dx$ в $K'$.
			\end{sample}
		\subsection*{Предельный переход в $K'$}
			Рассмотрим $\{f_n\}, f_n \in K', f \in K'$
			\begin{definition}
				$f_n \to f$ в  $K'$, если $(f_n, \varphi) \to (f, \varphi), \forall \varphi \in K$
			\end{definition}
			Пусть $f_n, f \in E, n \geq 1, f_n \rightrightarrows f$ в среднем на $[a,b]$ ($f_n,f$ сосредоточены на $[a,b]$). $|(f_n-f, \varphi)| = |\int\limits_{-\infty}^{\infty} (f_n-f)\varphi dx | \leq \{\text{КБШ}\} \leq \left(\int\limits_a^b (f_n-f)^2dx\right)^{1/2} \left(\int\limits_a^b \varphi^2 dx\right)^{1/2} \to 0$. То есть следует сходимость в $K'$.
			\begin{lemma}
				Пределом регулярных функций может быть сингулярная		
			\end{lemma}
			\begin{proof}
				$f_n(x) = \begin{cases}
					\dfrac{n}{2}, \ x \in [-\dfrac{1}{n}, \dfrac{1}{n}]\\
					0, \ |x| > \dfrac{1}{n}
				\end{cases}$\newline
				$(f_n(x), \varphi(x)) = \dfrac{n}{2} \int\limits_{-1/n}^{1/n} \varphi(x)dx = \{\text{формула среднего}\} = \dfrac{n}{2} \varphi(\xi) \dfrac{2}{n} \to \varphi(0) = (f(x), \varphi(x)) $ $\Rightarrow$  $f_n(x) \to \delta(x)$ в $K'$.
			\end{proof}
		\subsection*{Масса материальной точки}
		\subsection*{Плоскость электрического диполя}
		\subsection*{Первообразная обобщённых функций}
			Рассмотрим уравнение $y' = 0$ в $K'$ (1).
			\begin{lemma}
				В пространстве $K'$ уравнение (1) имеет решение $y = \mathrm{const}$.
			\end{lemma}
			\begin{proof}
				$(y', \varphi) = -(y, \varphi') =0 (2)$. (2) определяет решение уравнения на пробных функциях, которые являются производными от других пробных функций $\psi(x) \in K$, $\psi(x) \geq 0$ - не может быть пробной так как пробные функции не являются монотонными. Обозначим пространство $K_0 = \{\varphi_0(x) \in K \vert \exists \varphi_1(x) - \text{пробная}: \varphi_0(x) = \varphi_1'(x)\}, K_0\subset K$
			\end{proof}
			\begin{lemma}
				$\varphi_0(x) \in K_0 \Leftrightarrow \int\limits_{-\infty}^{\infty} \varphi_0(x) dx = 0$
			\end{lemma}
			\begin{proof}
				$\Leftarrow:$ Пусть $\varphi_1(x) = \int\limits_{-\infty}^{x}\varphi_0(t) dt \Rightarrow \varphi_1(x) \in C^\infty, \ \varphi_1'(x) = \varphi_0(x)$. Пусть $\varphi_0$ сосредоточена на $[a,b]$, тогда $\int\limits_{-\infty}^{x}\varphi_0(t) dt = \varphi_{-\infty}^{\infty} \varphi_0(t)dt  = 0$ при $x > b$. \newline
				$\Rightarrow:$ $\varphi_0 \in K_0, \ \exists \varphi_1: \varphi_0 = \varphi_1'$. $\int\limits_{-\infty}^{\infty} \varphi_0 dx = \int\limits_{-\infty}^{\infty} \varphi_1' dx = \varphi_1 \vert_a^b$. Рассмотрим $\forall \varphi_1 \in K: \int\limits_{-\infty}^{\infty} \varphi_1 dx =1, \varphi_1 \in K \backslash K_0$. Рассмотрим $\forall \varphi \in K$, представим в виде $\varphi(x) = \varphi_0(x) + \varphi_1(x) * \int\limits_{-\infty}^{\infty} \varphi(x) dx, \forall \varphi \in K$, здесь $\varphi_0$ - проекция $\varphi$ на $K_0$, а $\varphi_1(x) * \int\dots$ - проекция на $K \backslash K_0$. Получается, что $\mathrm{dim} (K \backslash K_0) = 1$.\newline
				$(y, \varphi) = (y, \varphi_0) + (y_1, \varphi_0)\int\limits_{-\infty}^{\infty} \varphi(x) dx$. Пусть $(y, \varphi_1) = c_1$ - произвольная постоянная. $(y, \varphi) = c_1 \int\limits_{-\infty}^{\infty} \varphi(x) dx = (c_1, \varphi), \forall \varphi \in K$. То есть $y = c_1$ в $K'$. 
			\end{proof}
			\begin{sample}
				$\lim\limits_{A \to \infty} \dfrac{1}{\pi} \dfrac{\sin{Ax}}{x}	 = ?$\newline
				$(\dfrac{1}{\pi} \dfrac{\sin{Ax}}{x}, \varphi(x)) = \dfrac1\pi \int\limits_{-\infty}^{\infty} \dfrac{\sin{Ax}}{x}\varphi(x) dx = \{\pm \varphi(0)\} = 
				\dfrac1\pi \int\limits_{-\infty}^{\infty} \dfrac{\sin{Ax}}{x} \varphi(0) dx + \dfrac1\pi \int\limits_{-\infty}^{\infty} \dfrac{\sin{Ax}}{x} (\varphi(x) -\varphi(0)) dx = $\newline$=\varphi(0) + \dfrac1\pi \int\limits_{-a}^{a} \dfrac{\sin{Ax}}{x} (\varphi(x) -\varphi(0)) dx + \int\limits_{|x| \geq a} \varphi(0) \dfrac{\sin{Ax}}{x}dx = \varphi(0) + \dfrac{1}{\pi} \int\limits_{-a}^{a} \varphi'(\xi) \sin{Ax} dx+\int\limits_{|x| \geq a} \varphi(0) \dfrac{\sin{Ax}}{x}dx$\newline
				$\int\limits_{|x| \geq a} \varphi(0) \dfrac{\sin{Ax}}{x}dx$ - хвост сходящегося ряда, поэтому стремится к нулю. \newline
				$\dfrac{1}{\pi} \int\limits_{-a}^{a} \varphi'(\xi) \sin{Ax} dx = -\dfrac1{A} \varphi' \cos{Ax} \vert_{-a}^a + \dfrac{1}{A} \int\limits_{-a}^{a} \varphi''(x) \cos{Ax} dx \to 0$ при $A \to + \infty$.\newline
				Получили, что $\lim\limits_{A \to \infty} \dfrac{1}{\pi} \dfrac{\sin{Ax}}{x} = \delta(x)$ в $K'$.
			\end{sample}
			\begin{sample}
				$\lim\limits_{A \to \infty} \sin{Ax} = 0$. Проверяется аналогично предыдущему пункту.
			\end{sample}
			Рассматриваем уравнение $y' = f, \ f \in K' (2)$.
			\begin{lemma}
				$\forall f \in K'$ уравнение (2) имеет решение в $K'$.
			\end{lemma}
			\begin{proof}
				$(y', \varphi) = -(y, \varphi') = (f, \varphi) = (f, \int\limits_{-\infty}^{x} \varphi'(\xi) d\xi)$\newline
				$(y, \varphi') = (f, - \int\limits_{-\infty}^{x} \varphi'(\xi) d\xi), \ \forall \varphi \in K (3)$. Пусть $\varphi_1(x) \in K, \ \int\limits_{-\infty}^{\infty} \varphi_1(x) dx = 1$, $\forall \varphi \in K: \ \varphi(x) = \varphi_1(x) \int\limits_{-\infty}^{\infty} \varphi(x) dx + \varphi_0(x)$. Определим функционал $y_0$ по действию на $\varphi_0(x)$. $(y_0, \varphi) = (y, \varphi_0) = (f, - \int\limits_{-\infty}^{\infty} \varphi_0(\xi) d\xi)$, $y_0$ - частное решение (2) или неопределенный интеграл функции $f$. $(y, \varphi) = (y_1, \varphi_1) \int\limits_{-\infty}^{\infty} \varphi dx + (y, \varphi_0) \Rightarrow$ решение уравнения (2) $\exists$ в $K'$ и записывается в виде $y = y_0 + C$. 
			\end{proof}
			\begin{sample}
				Решить уравнение $y'+y = \theta(x)$ в $K'$.\newline
				$y = z e^{-x}, y' = z'e^{-x} - z e^{-x}$, тогда наше уравнение: $z'e^{-x} = \theta(x)$, $z' = \theta(x) e^{x}$. По формуле, полученной выше: $(z_0, \varphi) = (z, \varphi_0) = (f, -\int\limits_{-\infty}^{x} \varphi_0(\xi) d\xi) = (\theta(x) e^x, - \int\limits_{-\infty}^{x} \varphi_0(\xi) d\xi) = -\int\limits_0^{\infty}e^x \int\limits_{-\infty}^{x} \varphi_0(\xi) d\xi dx = -e^x \int\limits_{\infty}^{x} \varphi_0 d\xi \vert_{x=0}^\infty + \int\limits_0^{\infty} e^x \varphi_0(x) dx = \{\varphi_0 \in K_0: \int\limits_{-\infty}^{\infty} \varphi_0 dx\xi = 0\} = \int\limits_{\infty}^{0} \varphi_0 d\xi + \int\limits_0^{\infty} e^x \varphi_0(x) dx = \int\limits_{-\infty}^{\infty} \varphi_0 d\xi + \int\limits_0^\infty (e^x - 1) \varphi_0 dx = \int\limits_{-\infty}^{\infty} \theta(x) (e^x - 1) \varphi_0(x) dx$. Поэтому $z_0(x) = \theta(x)(e^x -1) \Rightarrow z(x) = \theta(x) (e^x -1) + C$, $y = \theta(x) (1 - e^{-x}) + Ce^{-x}$
			\end{sample}
	\subsection*{Обобщённые функции в $\mathbb{R}^n$}
		\begin{definition}
			Функция называется обычной, если она определена на $\mathbb{R}^n$, принимает вещественные значения, интегрируема (по Лебегу) по любому n-мерному брусу.
		\end{definition}
		Множество всех обычных функций: $E = E_n = E(\mathbb{R}^n)$
		\begin{definition}
			Функция называется пробной (основной), если бесконечно дифференцируема в $\mathbb{R}^n$ и равна 0 вне некоторого бруса.
		\end{definition}
		Наименьшее замкнутое множество в $\mathbb{R}^n$ вне которого пробная функция равна нулю называется её носителем.  Множестов пробных функция $K = K_n$. 
			\begin{definition}
				$\{\varphi_n\}, \varphi_n \in K_n, \varphi_n \to 0$ при $n\to\infty$ в $K_n$, если все $\varphi_n$ сосредоточена на одном брусе и $\varphi_n \rightrightarrows 0 $ в $K_n$ вместе со всеми производными.		
			\end{definition}
		\begin{definition}
			Обобщённой функцией назовём любой линейный непрерывный функционал на пространстве $K_n$. Если $f \in E_n$, то она поражадает функционал $(f, \varphi) = \int\limits_{\mathbb{R}^n} f\varphi dx (1)$. Тогда $f$ - регулярная обобщённая функция.
		\end{definition}
		\begin{notabene}
			Существуют функционалы не представимые в виде (1). Например $\delta(x): (\delta, \varphi) = \varphi(0)$
		\end{notabene}
		\begin{definition}
			Обобщённая функция $f \in K_n'$ имеет порядок синуглярности $\leq p$, если она представима в виде: 
			\begin{equation*}
				(f,\varphi) = \sum\limits_{|k| \leq p} f_k(x) D^k \varphi(x) dx  = \sum\limits_{|k| \leq p} (f_k, D^k \varphi) (2), 
			\end{equation*}
			где $k = (k_1, \dots, k_n), |k| = k_1 + \dots + k_n, D^k = \dfrac{\partial^{|k|}}{\partial x_1^{k_1} \dots \partial x_n^{k_n}}$
		\end{definition}
		\subsubsection*{Действия с обобщёнными функциям}
			\begin{enumerate}
				\item $\forall \alpha_1, \alpha_2 \in \mathbb{R}, \forall f_1, f_2 \in K_n'$ $\Rightarrow$ $(\alpha_1 f_1 + \alpha_2 f_2, \varphi) = \alpha_1(f_1, \varphi) + \alpha_2(f_2, \varphi)$
				\item $\forall f \in K_n', \alpha(x) \in C^\infty(\mathbb{R}^n)$ $\Rightarrow$ $(\alpha(x)f, \varphi) = (f, \alpha(x) \varphi)$
				\item Предельный переход. $\{f_\nu\}, f_\nu \in K_n', f \in K_n'$, $f_\nu \to f$ в $K_n'$, если $(f_\nu, \varphi) \to (f_0, \varphi), \ \forall \varphi \in K_n$. 
				\item $(D^kf, \varphi) = (-1)^{|k|} (f, D^k\varphi)$
			\end{enumerate}
		\subsubsection*{Примеры на дифференциорование}
			\begin{sample}
				$(\dfrac{\partial^2f}{\partial x \partial y}, \varphi) = (f,\dfrac{\partial^2\varphi}{\partial y \partial x})= (f,\dfrac{\partial^2\varphi}{\partial x \partial y}) = (\dfrac{\partial^2f}{\partial y \partial x}, \varphi)$. То есть для обобщённых функций также смешанные производные совпадают.
			\end{sample}
			\begin{sample}
				\begin{definition}
					Обычная $g(x) \in E_n$ называется обобщённой производной по Соболеву от функции (обычной) $f$ на множестве $G$, если $\forall \varphi \in C^\infty(G)$, сосредточенной строго внутри $G$ выполнено $(-1)^{|k|} \int\limits_G g(x)\varphi(x) dx = \int\limits_G f(x) D^k \varphi(x) dx \Rightarrow g(x) = D^k f$ в $K'$.
				\end{definition}
			\end{sample}
			\begin{sample}
				$\theta(x) = \begin{cases}
					1, \ x_1, \dots, x_n \geq 0 \\
					0, \ \mathbb{R}^n \backslash \{x_1, \dots, x_n \geq 0\}
				\end{cases}$\newline
				$(\dfrac{\partial^n \theta}{\partial x_1 \dots \partial x_n}, \varphi(x)) = (-1)^n (\theta, \dfrac{\partial^n \varphi}{\partial x_1 \dots \partial x_n}) = (-1)^n \int\limits_{x_1, \dots, x_n \geq 0} \dfrac{\partial^n \varphi}{\partial x_1 \dots \partial x_n} dx_1 \dots dx_n = \varphi(0,\dots, 0) = (\delta(x), \varphi)$
			\end{sample}
			\begin{sample}
				Оператор Лапласа от сферических функций: \newline
				$\Delta f = r^{p-2} p(p+n-2)$ это верно в $K'$, если все функции из $E$.\newline
				При $p-2 > -n$ все верно в $K_n'$. Пусть $p= 2-n, n=2$
			\end{sample}
\end{document}