\documentclass[a4paper,10pt]{article}
\usepackage[T2A]{fontenc} % Кириллица
\usepackage[utf8]{inputenc} % Кодировка UTF-8
\usepackage[russian]{babel} % Поддержка русского языка
\usepackage[margin=1in]{geometry}
\usepackage{titlesec}
\usepackage{enumitem}
\usepackage{hyperref}

% Оформление разделов
\titleformat{\section}{\Large\bfseries}{}{0pt}{}[\titlerule]
\titleformat{\subsection}{\bfseries}{}{0pt}{}
\pagenumbering{gobble}

\renewcommand{\baselinestretch}{1.2}

% Настройка ссылок
\hypersetup{
	colorlinks=true,
	linkcolor=blue,
	urlcolor=blue
}

\begin{document}
	
	\begin{center}
		{\Huge \textbf{Васильченко Дмитрий Дмитриевич}} \\[0.5em]
		{Адрес, Москва, Россия} \\
		\href{mailto:dvasil.arm@gmail.com}{dvasil.arm@gmail.com} \\
		{Телефон: +7 (915) 411-19-73} \\
		{Telegram: \href{https://t.me/Demosfenn}{@Demosfenn}}
	\end{center}
	
	\vspace{1em}
	
	\section*{Образование}
	\begin{itemize}[left=0pt]
		\item \textbf{МГУ имени М.В. Ломоносова, факультет ВМК} \hfill 2021 -- наст. время \\
		Кафедра функционального анализа и применений. Средний балл: 4.41.
	\end{itemize}
	
	\section*{Опыт работы}
	\begin{itemize}[left=0pt]
		\item \textbf{Школа №1533 "ЛИТ"} \hfill 2022 -- наст. время \\
		Учитель по программированию (python, C$\#$). Проведение занятий и поддержка учебных проектов.
		\item \textbf{Центр фундаментальной и прикладной математики МГУ} \hfill 2023 -- наст. время \\
		Научная работа по темам "Функциональный анализ" и "Уравнения в частных производных".
		\item \textbf{Фриланс} \hfill ранее \\
		Выполнял небольшие заказы по программированию на языках C++, C$\#$, Assembler, Python.
	\end{itemize}
	
	\section*{Научная деятельность}
	\begin{itemize}[left=0pt]
		\item Публикации по темам системное программирование и \href{https://journals.rcsi.science/0374-0641/issue/view/17906}{уравнения в частных производных} (журнал "Дифференциальные уравнени").
		\item Призёр международных конкурсов ICYS  и \href{https://partner.projectboard.world/isef/project/soft004---search-for-algorithmic-errors}{ISEF} с проектом "поиск алгоритмических ошибок в исходном коде методами машинного обучения".
		\item Участник конференции \href{https://cs.msu.ru/sites/cmc/files/attachs/tihonovskie_chteniya_el._versiya_0.pdf}{Тихоновские чтения}.
		\item Хакатоны по тематике "Умный город".
	\end{itemize}
	
	
	\section*{Навыки}
	\begin{itemize}[left=0pt]
		\item Языки программирования: C$\#$, C, C++, Arduino, Python, SQL.
		\item Прошёл множество курсов по C++, SQL и Python на Coursera.
		\item Операционные системы: Windows, Linux.
	\end{itemize}
	
	\section*{Дополнительная информация}
	\begin{itemize}[left=0pt]
		\item Разрабатываю программу для автоматизации ведения волейбольной статистики
		\item Хобби: Спорт, волейбол, программирование.
	\end{itemize}
	
\end{document}
